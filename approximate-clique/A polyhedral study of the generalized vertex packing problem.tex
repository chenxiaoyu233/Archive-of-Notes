% Created 2019-01-23 Wed 00:08
% Intended LaTeX compiler: pdflatex
\documentclass[11pt]{article}
\usepackage[utf8]{inputenc}
\usepackage[T1]{fontenc}
\usepackage{graphicx}
\usepackage{grffile}
\usepackage{longtable}
\usepackage{wrapfig}
\usepackage{rotating}
\usepackage[normalem]{ulem}
\usepackage{amsmath}
\usepackage{textcomp}
\usepackage{amssymb}
\usepackage{capt-of}
\usepackage{hyperref}
\author{陈小羽}
\date{\today}
\title{}
\hypersetup{
 pdfauthor={陈小羽},
 pdftitle={},
 pdfkeywords={},
 pdfsubject={},
 pdfcreator={Emacs 26.1 (Org mode 9.2)}, 
 pdflang={English}}
\begin{document}

\tableofcontents

\section{What is vertex packing problem (VP)}
\label{sec:orge610274}
The traditional vertex packing problem defined on an
undirected graph identifies the largest weighted independent
set of nodes, that is, a set of nodes whose induced subgraph 
contains no edges.
\section{The generalized vertex packing problem (GVP-k) that this article cares about}
\label{sec:orgb6ab1a5}
k edges may exist within the subgraph induced by the chosen set of nodes.
\section{Some application}
\label{sec:orgf610b2e}
A particular context in which such problems arise is in the national 
airspace planning model
\section{Introduction}
\label{sec:org21c6bb3}
G = (N, E), weighted c\textsubscript{j}, for j = 1, \(\cdots{}\), n.
\subsection{Integer Programming Model (for VP)}
\label{sec:orgeb72cf9}
\begin{align*}
	\mbox{Maximize: } & cx \\
	\mbox{Subject to: } & Ax \leq e \\
				& x \in \{0, 1\}
\end{align*}
\begin{enumerate}
\item A is a p \texttimes{} n matrix a\textsubscript{hi} = 1 means vertex i \(\in\) edge h
\item e is a all-one vector.
\item \(\Rightarrow\) Ax \(\le\) e means that each edges 2 end-points should not
be in the answer x simutaneously.
\end{enumerate}
\subsection{Prefect Graph}
\label{sec:org0142686}
chromatic number = maximum clique cardinality (for each G' \(\subseteq\) G)
\subsection{Integer Programming Model (for GVP-k)}
\label{sec:org735cb32}
\begin{align*}
 \mbox{Maximize: } & cx \\
 \mbox{Subject to: } & \sum_{(i,j) \in E} z_{ij} \leq k \\
 & z_{ij} \geq x_i + x_j - 1, z_{ij} \geq 0 \\
 & x_j \in \{0, 1\}, \forall j \in N
\end{align*}
note that edge (i, j) is in the answer when \(z_{ij} = 1 = x_ix_j\)
\section{{\bfseries\sffamily TODO} Facets and partial convex hull representations for GVP-k}
\label{sec:org91dcc3f}
\textbf{Proposition 1}: Consider a graph G and a subgraph \(\hat{G}\) of G. 
If Dx \(\le\) d represents a set of valid inequalities for GVP-k defined on \(\hat{G}\), 
then Dx \(\le\) d is valid for GVP-k defined on G. 

\textbf{Proof}: Since GVP-k for \(\hat{G}\) is a relaxation of GVP-k defined on G, 
the \uline{restrictions} that govern a feasible generalized vertex packing
solution on \(\hat{G}\) are a subset of those valid for G. This completes the proof.

\textbf{Note}: You could use less restrictions on \(\hat{G}\) than on G (maybe a subset).
\end{document}