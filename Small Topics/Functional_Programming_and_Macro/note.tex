% Created 2019-09-06 Fri 16:10
% Intended LaTeX compiler: xelatex
\documentclass{article}
\usepackage{ctex}
\usepackage{tcolorbox}
\usepackage{amsmath}
\usepackage{amsthm}
\author{陈小羽}
\date{\today}
\title{函数式, 宏, lambda calculus}

\begin{document}
\maketitle
\section{Background}
当我开始使用 \LaTeX 的一些比较高级的宏包时, 我发现了一件很不可思议的事情.
在我的认知中, 像 \TeX 这样的纯粹用宏编写的系统是做不出来很复杂的逻辑的.
因为宏的执行顺序和人的思考顺序是严重不符合的, 所以通过宏编写复杂的程序是一件很复杂的工作.
并且, 宏的展开只是简单的模板匹配, 缺乏严谨性, 所以很容易写出错误的代码.

然而, 当我开始使用 tikz 这一类宏包的时候, 我被震撼了. 
tikz/pgf 包内, 实现了表达式求值和复杂的控制流语句.
这里面甚至有 \emph{foreach} 语句

\begin{verbatim}
    \foreach \x in \{1, 2, 3\} \{
        \x
    \}
\end{verbatim}

当时我在想, 如果我的认识是正确的话, 那么一定存在某种\underline{系统性}的方法, 这种方法能够支持有条理的宏编程.
而这种方法恰好是我比较感兴趣的, 所以我对此进行了一些研究.
这项研究进行了大约一周, 在这之间, 我对很多的概念进行了探索, 并找到了他们之间的一些联系.

本文将以lambda表达式为线索, 讨论宏和lambda表达式, 函数式编程和lambda表达式的关系.
在这两个基础上, 再来探究一个系统性的宏程序的实现思想, \emph{Continuation Passing Style} (一般简写为 CPS).
这种思想在多个不同的领域被反复的发现, 不仅使得系统性宏编程成为可能, 还成为了函数式编程语言中处理副作用的一种常用的方法 (又被称为 Monad).

% 这里简单介绍一下什么是lambda calculus
\section{lambda演算}
我们首先来定义lambda表达式.

\begin{tcolorbox}[title={define: lambda expression}]
\begin{verbatim}
<lambda expression> -> <constant> 
                    |  <variable> 
                    |  <lambda apply> 
                    |  <lambda abstraction> 
<constant> -> <number> 
<variable> -> x | y | z | ...
<lambda apply> -> <lambda expression> <lambda expression> 
<lambda abstraction> -> lambda x . <lambda expression>
\end{verbatim}
\end{tcolorbox}

lambda 演算建立在变量替换的基础之上, 下面我们来定义这个替换过程:
\begin{tcolorbox}[title={define: substitution}]
  \begin{align*}
                 c[y/x] &= c \\
    z[y/x] &= z \\
    x[y/x] &= y \\
    (s\ t)[y/x] &= s[y/x]\ t[y/x] \\
    (\lambda x. t)[y/x] &= \lambda x.t \\
    (\lambda z. t)[y/x] &= \lambda z. t[y/x], \mbox{if $z\not\in FV(y)$ (避免y的Free Vars被z Capture)} \\
    (\lambda z. t)[y/x] &= \lambda w .t[w/z][y/x], \hspace{0.1cm} \mbox{otherwise, find $w\not\in FV(t)\cup FV(y)$} \\
           & \mbox{$w\not\in FV(y)$: 避免y的Free Vars被w Capture} \\
    & \mbox{$w\not\in FV(t)$: 避免t的Free Vars被w Capture}
  \end{align*}
\end{tcolorbox}
可以很容易的发现, 上面的替换过程的定义, 是定义在lambda表达式递归结构的基础上, 它涵盖了在替换过程中可能遇到的所有情况.
\begin{tcolorbox}[title={define: conversions}]
  \begin{itemize}
  \item $\alpha$-conversions(变量换名): $\lambda x . t \xrightarrow{\alpha} \lambda y . t[y/x]$. (这里的$y$只能是一个变量或者常数)
  \item $\beta$-conversion(函数调用): $(\lambda x. t)\ y \xrightarrow{\beta} t[y/x]$.
  \item $\eta$-conversion(去调用): $(\lambda x. t)\ x \xrightarrow{\eta} t$ where $x\not\in FV(t)$.
  \end{itemize}
\end{tcolorbox}
不难看出,$\eta$-conversion只是$\beta$-conversion的一种特殊情况(貌似在逻辑学中常用).
这里, 从程序的角度上来说, 最值得我们关注的是$\beta$-conversion.
通过反复的对一个式子进行$\beta$-conversion, 我们便可以模拟计算的过程.

\textcolor{red}{TODO: 这里先放一下lambda演算, 将具体的比较放到后面}

\section{lambda演算和函数式编程}
lambda演算和函数式编程几乎就是同一个东西.
几乎可以讲函数式编程语言理解为一种 ``\textcolor{blue}{特殊记法}'' 下的lambda演算.
这种 ``\textcolor{blue}{特殊的记法}'' , 很多时候都可以使用宏来实现.

\subsection{lambda表达式的结果与求值顺序无关}
lambda表达式有一个很重要的性质(这个性质听说不太好证明, 是Church-Rosser theorem的一个推论)
\begin{tcolorbox}
  同一个lambda表达式, 通过顺序进行计算(conversion), 最终如果能够停机的话, 那么这些所有可能的执行结果在 $\alpha$-conversion 下面是等价的.
  因为这个性质, 我们通常使用
  \[t = s\]
  表示$t$可以通过一些列conversion转换为$s$, 因为这样的话, 他们最终执行的结果一定是相同的.
  \tcblower
  回忆: 在$\alpha$-conversion下等价表示我们只允许变量名不同, 所以这些表达式的意义最终都是一样的.
\end{tcolorbox}
其实, 通过lambda表达式, 我们可以构造函数式编程语言, 像haskell之类的函数式编程语言中的lazy evaluation的概念, 就来自与这里. 因为表达式的执行顺序可以是任意的, 编译器就可以任意的优化其求值顺序, lazy evaluation指的就是将求求值的时机尽可能的退后. 另外一门函数式语言Ocaml则走上了完全相反的道路, Ocaml采用eager evaluation, 是能求值, 就先求值的一种语言.
可以说, 不同的函数式编程语言, 因为内核都是lambda演算, 所以他们之间的不同往往体现在编译器的求值顺序上.


John Harrison的<<Introduction to Functional Programming>>这本书中, 有一章专门讲了如何通过lambda演算来构造一个函数式编程语言.
简单来说:
\begin{align*}
  \mbox{True} &:= \lambda x\ y\ .\ x \\
  \mbox{False} &:= \lambda x\ y\ .\ y \\
  (x, y) &:= \lambda f . f x y \\
  \mbox{Fst Pair} &:= \mbox{P True} \\
  \mbox{Snd Pair} &:= \mbox{P False} \\
  \mbox{If E Then E1 Else E2} &:= E\ E1\ E2 
\end{align*}
当我们有了If语句之后, 我们可以通过If语句很快的构造出And, Or, Xor之类的逻辑运算.
我们甚至可以构造出自然数
\begin{align*}
  n &:= \lambda f\ x. f^n\ x \\
  \mbox{SUC}\ n &:= \lambda n\ f\ x. n\ f\ (f\ x) \\
  \mbox{ISZERO}\ n &:= n (\lambda x. \mbox{False}) \mbox{True} \\
  m + n &:= \lambda f\ x.\ m\ f\ (n\ f\ x) \\
  m \times n &:= \lambda f\ x.\ m\ (n\ f)\ x\\
  \mbox{PREFN} &= \lambda f\ p.\ \mbox{(False, If Fst P Then Snd P Else $f$(Snd P))} \\ 
  \mbox{PRE}\ n &= \lambda f\ x.\ \mbox{Snd(n (PREFN $f$) (True, x))}
\end{align*}
我们甚至还可以构造出递归函数, 下面这个东西叫做Y组合子 (是haskell这个人发明的)
\begin{align*}
  Y := \lambda f.\ (\lambda x.\ f(x\ x))\ (\lambda x.\ f(x\ x))
\end{align*}
它有一个性质: 对任意的$F$, 都有$Y F = F(Y F)$, 于是我们可以设计递归函数.
比如, 如果我们希望计算阶乘可以这样来:
\begin{align*}
  fact(n) &:= \mbox{If ISZERO $n$ Then 1 Else $n\times fact(n-1)$} \\
  fact    &:= \lambda n.\ \mbox{If ISZERO $n$ Then 1 Else $n\times fact(n-1)$} \\
  fact    &:= (\lambda f\ n.\ \mbox{If ISZERO $n$ Then 1 Else $n\times f(n-1)$})\ fact
\end{align*}
令
\[
   F := \lambda f\ n.\ \mbox{If ISZERO $n$ Then 1 Else $n\times f(n-1)$}
\]
则 
\begin{align*}
  fact &= F\ fact \\
  Y\ F &= F\ (Y\ F)
\end{align*}
所以, 我们只需令$fact = Y\ F$即可实现这个递归函数.

最后一步, 我们还可以很简单的实现local-binding:
\begin{align*}
  \mbox{Let $x = s$ in $t$} &:= (\lambda x.\ t)\ s
\end{align*}

这样之后, 我们的lambda演算作为一个函数式编程语言, 可以说是非常完善了.
很多函数式语言, 其实就是在lambda演算的基础之上实现了类型.
我们这里讨论的lambda演算被称为untyped-lambda-calculus, 带类型的lambda演算被称为
typed-lambda-calculus

这里写的非常的简略, 细节还是要参考<<Introduction to Functional Programming>>这本书才行.
% 这里简单介绍一下什么是宏展开
\section{Macro 宏 与 lambda演算}
不难发现, lambda演算最深刻, 最核心的部分在其替换过程(substitution)的定义上, 其他东西只是形式而不太重要.
很多语言的宏, 在形式和替换过程的定义上, 都和lambda演算不同.
虽然都不一样, 但是很相似的一点是: 宏和lambda演算都使用 ``\textcolor{blue}{替换}'' 这个概念来表达计算的过程.

\subsection{C++宏}
\begin{tcolorbox}
\begin{verbatim}
#define FOO BAR // FOO --> BAR
#define ID(a) a // ID(a) --> a
#define FUN(a, b) a + b // FUN(a, b) --> a + b
#define CAT(a, b) a ## b // CAT(a, b) --> ab
#define STR(a) #a // STR(a) --> "a"
#define CDR(a, ...) __VA_ARGS__ // CDR(a, ...) --> ...
\end{verbatim}
\end{tcolorbox}
不难发现, 在形式上, C++的宏对lambda演算做了扩展, 增加了像 \verb #a , \verb a##b 还有 \verb __VA_ARGS__ 这样的语法, 为程序开发增加了便利.
但是在另外一个方面, C++的宏执行替换的机制和lambda演算不太一样.
C++的宏的替换是基于模板匹配的, 可以通过如下的规则来的描述:
\begin{tcolorbox}
  \begin{itemize}
  \item 编译器会从左往右扫描所有的TOKEN, 如果发现这个TOKEN和某个宏名相同的话 (比如 \verb CAT ), 就去匹配这个宏名剩下的括号部分(比如 \verb (a,b) )和这个TOKEN的右边部分. 匹配好之后, 编译器会将整个宏展开称宏定义的样子. 如果匹配不成功, 则编译器考虑下一个token, 以后都不会考虑这个TOKEN了.
  \item 编译器展开了一个宏之后, 会立即将扫描的指针置于展开的这个串的头部, 重新检查这个新展开的串里面是否有TOKEN可以继续展开.
  \item 在展开一个宏的时候, 编译器会先将所有括号中的式子展开, 然后再来展开这个宏.
  \end{itemize}
\end{tcolorbox}
比如:
\begin{tcolorbox}
\begin{verbatim}
#define IF_1(a, b) a
#define IF_0(a, b) b
#define IF(val) IF_ ## val

IF(1) (this, that)
==> IF_1 (this, that)  /* IF(1) --> IF_1 */
==> this  /* IF_1(this, that) --> this */

\end{verbatim}
\end{tcolorbox}
从使用的角度上来看, C++的宏比lambda演算要更加的灵活.
然而, C++的这种基于pattern matching的替换策略是不太严谨的.
\paragraph{} \emph{C++的宏的基本操作单位是字符串, 不关心数学意义, 而lambda表达式的基本单位是变量, 常数, 等, 是数学元素, 在运算的时候会考虑其数学意义.}
\begin{verbatim}
#define MUL(a, b) a * b
MUL(2 + 3, 4 + 5) ==> 2 + 3 * 4 + 5
\end{verbatim}
这种现象产生的原因是: C++的宏操作的单位是TOKEN, 也就是字符串, C++编译器不会去考虑这些东西的数学意义.
而lambda演算中, 带入的参数只能是变量, 常数, 或者另外一个lambda表达式 (相当于会自动加括号, 而添加括号不会影响lambda表达式的意义).
比如
\begin{align*}
  (\lambda x\ y . x \times y)\ (1 + 2)\ (4 + 5) \xrightarrow{\beta} (1 + 2) \times (4 + 5)
\end{align*}
这里必须加上括号, 否则不是一个合法的lambda表达式.

\paragraph{} \emph{C++的替换策略是局部的, 而lambda演算的替换策略是全局的.}

这一点可以通过下面的例子来说明
\begin{tcolorbox}
\begin{verbatim}
#define ADD(a, b) a + b + x
#define FUN(x) ADD
#define EVAL(...) __VA_ARGS__

EVAL(FUN(y) (a, b)) ==> EVAL(ADD(a, b)) 
                    ==> ADD(a, b) 
                    ==> a + b + x
/* 其实EVAL只是利用C++宏的规则, 
   让嵌套的宏能够继续展开下去的一种技巧
   便于理解的话, 上面的展开过程完全可以写成下面这样 */
FUN(y) (a, b) ==> ADD(a, b) ==> a + b + x
\end{verbatim}
\end{tcolorbox}
而这个东西用lambda表达式来表示应该是这个样子的:
\begin{align*}
  (\lambda x . \lambda a\ b. a + b + x)\ y\ a\ b &\xrightarrow{\beta} (\lambda a\ b . a + b + y)\ a\ b \\
  &\xrightarrow{\beta} a + b + y
\end{align*}
为啥会导致这个问题呢? 因为C++不允许在宏内部定义宏, 导致ADD(a, b)中的x, 不能被bind到FUN上面, 或者说, 不能被FUN(x)捕获到.

\paragraph{} \emph{C++的宏是不卫生的}
\begin{tcolorbox}
\begin{verbatim}
#define X_ADD_Y x + y
#define SUB(z) X_ADD_Y - z

SUB(x) ==> X_ADD_Y - x 
       ==> x + y - x 
       ==  y
\end{verbatim}
\end{tcolorbox}
可以看到, X\_ADD\_Y内部的x, y原本都是自由变量, 现在加到了SUB(z)下面, 因为$z\not=x, z\not=y$, 所以$x, y$理应还是自由变量才对, 但是当调用$SUB(x)$时, 因为$x = x$, 所以外部的$x$捕获了内部$x + y$中的$x$, 迫使内部的$x$变成了被bounded varialbe, 这就导致了问题. 这种问题有个专门术语, 叫做不卫生(non-hygienic).

和C++形成对比的是, 有些语言中是实现了卫生宏的, 比如scheme.
\begin{verbatim}
(let-syntax
    ((insert-binding
      (syntax-rules ()
        ((_ x)
         (let ((a 1))
           (+ x a))))))
  (let ((a 5))
    (insert-binding (+ a 3))))

(let ((a 5))
  (insert-binding (+ a 3)))
==>
(let ((a 5))
  (let ((a:1 1))
    (+ (+ a 3) a:1)))
==>
(+ (+ 5 3) 1)
==>
9
\end{verbatim}
容易注意到, 上面这段代码中, 对内部的自由变量a做了处理, 使得它没有被外部的bound-variable a 捕获, 解决了C++中宏的问题.

\subsection{宏的结果和求值顺序相关}
考虑如下例子
\begin{verbatim}
#define ADD(a, b) a + b
#define RR )

ADD (1, 2 RR
\end{verbatim}
这个例子, 如果先展开RR, 那么最后就能展开为1+2,
否则就只能展开为 ADD (1, 2) 然后停下来.
因为这个原因, 所以很多时候, 宏编程非常容易出错, 难以进行系统性的编程.

\subsection{Q: 为什么不直接把宏设计成lambda演算那样呢?}
宏和lambda演算之间各有优劣, lambda演算注重的是计算, 所以它很严谨.
宏注重的是修改自己的这种能力, 所以它在严谨这个层面上做了妥协.
在下一节里面, 我们将继续讨论这个问题.

\section{如何系统性的编写宏程序}
通常我们会发现自己无法实现复杂的宏程序, 或者很多在其他语言中能够轻易的实现的函数, 难以想到如何简洁的通过宏来实现.
\subsection{编写宏程序的时候遇到的障碍}
\paragraph{无法保存中间结果}
我们在编写C++程序的时候, 经常会有如下的操作
\begin{verbatim}
int a = func1();
int b = func2(a);
int c = func3(a, b);
int d = func4(a, b, c);
\end{verbatim}
我们想要使用之前的程序运行的中间结果.
不幸的是: 宏展开几乎不提供这样的能力.
不仅不能提供这样的能力, 宏甚至连像这样顺序执行的能力都很难提供.

如果我们想把上面的程序用宏来写, 下面是两种naive的尝试.
\begin{verbatim}
#define func1() ret_func1
#define func2(a) ret_func2
#define func3(a, b) ret_func3
#define func4(a, b, c) ret_func4

/* 写法一 */
func4(func1(), func2(func1()), func3(func1(), func2(func1())))

/* 写法二 */
#define a func1()
#define b func2(a)
#define c func3(a, b)
#define d func4(a, b, c)
\end{verbatim}
不过如果程序发生递归, 就显得不太适用了.

有一种最通用的写法是continuation passing style (CPS), 也是本文的重点.
\subsection{CPS}
CPS风格, 要求我们在写所有的程序的时候, 统一使用这样的格式来写:
\begin{align*}
  <func, args, cont>
\end{align*}
其中, $func$ 表示宏名, 或者函数名, $args$ 表示参数列表, $cont$是一个回调函数, 表示$func$执行完之后要执行的内容.
CPS要求$cont$也具有 $<func, args, cont>$ 的形式.
有点像尾递归的写法.

宏编程中, 很多时候cont不是一开始就在的, 而是在展开func的时候动态构造出来的.
通过在执行上一条宏(语句)的时候, 生成下一跳需要执行的宏(语句), 这样就可以编写出能够正常运行的程序.

比如有人通过CPS, 实现了匿名函数(又叫lambda函数, 和之前我们提到的lambda演算不是同一个东西).
\clearpage
\footnotesize
\begin{verbatim}
(define-syntax ??!apply
  (syntax-rules (??!lambda)
    ((_ (??!lambda (bound-var . other-bound-vars) body) oval . other-ovals)
    (letrec-syntax
        ((subs
          (syntax-rules (??! bound-var ??!lambda)
            ((_ val k (??! bound-var)) (appl k val))
            ((_ val k (??!lambda bvars int-body))
             (subs-in-lambda val bvars (k bvars) int-body))
            ((_ val k (x))
             (subs val (recon-pair val k ()) x))
            ((_ val k (x . y))
             (subs val (subsed-cdr val k x) y))
            ((_ val k x)
             (appl k x))))
         (subsed-cdr
          (syntax-rules ()
            ((_ val k x new-y)
             (subs val (recon-pair val k new-y) x))))
         (recon-pair
          (syntax-rules ()
            ((_ val k new-y new-x) (appl k (new-x . new-y)))))
         (subs-in-lambda
          (syntax-rules (bound-var)
            ((_ val () kp int-body)
             (subs val (recon-l kp ()) int-body))
            ((_ val (bound-var . obvars) (k bvars) int-body)
             (appl k (??!lambda bvars int-body)))
            ((_ val (obvar . obvars) kp int-body)
             (subs-in-lambda val obvars kp int-body))))
         (recon-l
          (syntax-rules ()
            ((_ (k bvars) () result)
             (appl k (??!lambda bvars result)))))
         (appl
          (syntax-rules ()
            ((_ (a b c d) result) (a b c d result))
            ((_ (a b c) result) (a b c result))))
         (finish
          (syntax-rules ()
            ((_ () () exp) exp)
            ((_ rem-bvars rem-ovals exps)
             (??!apply (??!lambda rem-bvars exps) . rem-ovals))))
         )
         (subs oval (finish other-bound-vars other-ovals) body)))))
\end{verbatim}
\clearpage
\normalsize
实现匿名函数顺便还有一个好处, 就是可以利用函数的参数来模拟变量的赋值.
这个好处一般要通过使用CPS才能够实现.
我们可以从下面这一段伪代码中, 感受到这一点:
\begin{verbatim}
((lambda (x) 
  ((lambda (y)
    ((lambda (z) body) 
     y + 3))
   x + 2))
 6)
\end{verbatim}
上面这段代码相当于下面的代码
\begin{verbatim}
x = 6;
y = x + 2;
z = y + 3;
body;
\end{verbatim}
用宏来实现这个东西的原理很简单, 就是当运行最外侧的lambda函数的时候, 将6一层层的替换进去,
最终将body里面能够替换的x全部换成6. 通过这种思想, 我们就可以模拟函数的传参了.

上面的代码虽然能够模拟赋值语句, 但是陷入了回调地狱. 我们将其做如下改写:
\begin{verbatim}
(define-syntax CHG
  (syntax-rules ()
    ((_ args func)
     (func args))))

(CHG 6
  (lambda (x)
    (CHG x + 2
      (lambda (y)
        (CHG y + 3
          (lambda (z) body))))))
\end{verbatim}
这样就顺眼多了, 通过这种思想, 我们甚至可以直接用宏实现出let语句.
关于这个东西更加具体的细节可以参考Oleg Kiselyov的文章<<Macros that Compose: Systematic Macro Programming>>

\subsection{CPS在函数式编程中的应用 (Monad)}
同样是为了模拟赋值语句, haskell中常用一种叫做Monad的数学概念.
Monad的数学定义可以参考wikibook.
或者
\begin{verbatim}
http://adit.io/posts/2013-04-17-functors,_applicatives,_and_monads_in_pictures.html
\end{verbatim}
这篇帖子对为什么Moand可以模拟赋值语句做了清楚的阐述:
\begin{verbatim}
https://en.wikibooks.org/wiki/Haskell/Understanding_monads
\end{verbatim}
Haskell选用Moand处理副作用, 还有一个重要的原因是Monad中的(>>=)算符中,
有一个从 m a 到 a 的过渡过程, 所以在实现编译器的时候可以实现顺序.
比如一个用了monad的程序长这样
\begin{verbatim}
type Maybe = Just x | Nothing

father: a -> Maybe a
mather: a -> Maybe a

bothGrandfathers p =
       father p >>=
           (\dad -> father dad >>=
               (\gf1 -> mother p >>=   -- gf1 is only used in the final return
                   (\mom -> father mom >>=
                       (\gf2 -> return (gf1,gf2) ))))
\end{verbatim}
这个式子中的(>>=)算符是编译器内部实现的(可能是用C).
所以从lambda演算的角度来说, 这个式子已经没办法化简了.
但是从编译器的角度来说, 要想要执行
\begin{verbatim}
(\dad -> ...)
\end{verbatim}
这个函数, 必须先求出dad这个变量的值.
这里面有一个m a到 a 的过程是蕴含在\verb (>>=) 中的, 而m a到 a的这个过程中很多时候被编译器隐藏了很多有状态的, 不纯洁的代码在其中.
所以Monad在这个时候还起到了限制执行顺序的作用. 保证有副作用的代码的执行顺序.
haskell还专门给monad提供了一个do-syntax来简化monad的使用.
\end{document}
%%% Local Variables:
%%% mode: latex
%%% TeX-master: t
%%% End:
