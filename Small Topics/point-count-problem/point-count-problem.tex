% Created 2020-08-12 Wed 13:13
% Intended LaTeX compiler: xelatex
\documentclass[11pt]{article}
\usepackage{graphicx}
\usepackage{grffile}
\usepackage{longtable}
\usepackage{wrapfig}
\usepackage{rotating}
\usepackage[normalem]{ulem}
\usepackage{amsmath}
\usepackage{textcomp}
\usepackage{amssymb}
\usepackage{capt-of}
\usepackage{hyperref}
\usepackage{ctex}
\usepackage{ctex}
\author{陈小羽}
\date{\today}
\title{Review For Point Count Problem And A Devide And Conquer Trick}
\hypersetup{
 pdfauthor={陈小羽},
 pdftitle={Review For Point Count Problem And A Devide And Conquer Trick},
 pdfkeywords={},
 pdfsubject={},
 pdfcreator={Emacs 26.3 (Org mode 9.1.9)}, 
 pdflang={English}}
\begin{document}

\maketitle
\tableofcontents


\section{Preliminary}
\label{sec:org9919e1b}
\subsection{点/向量 (point/vector)}
\label{sec:org91114b1}
向量空间中的点和向量其实是相同的概念.
\(n\) 维空间中的点 \(v\) 可以用 \(n\) 元组 \((v_1, v_2, v_3, \cdots, v_n)\) 表示.
特别的, 为了简化, 一维空间中的点 \(x\) 直接使用 \(x\) 表示而省略括号.
我们可以将高维度的点看成许多一维空间中的点的元组.
\begin{verbatim}
struct Point {int w[d], id; bool inS;}; // d 是维度
\end{verbatim}
\subsection{点集 (point set)}
\label{sec:orgd7b728b}
\(n\) 维向量空间是包含无限个点的集合, 任意多个 \(n\) 维向量组成的集合都是 \(n\) 维向量空间的子集.
为了方便描述 \(n\) 维向量空间的子集, 引入记号 \(\{ x | P(x)\}\) 来表示所有满足条件 \(P\) 的点组成的集合.
\(|S|\) 表示集合 \(S\) 中的元素个数. 本文中用到的集合为可重复集, 有特殊情况会专门说明.
\begin{verbatim}
Point S[n];// 集合大小为n
\end{verbatim}
\subsection{n维数点问题}
\label{sec:org6717686}
\subsubsection{n维数点问题的一般形式 (ND Point Count Problem \(\Rightarrow\) PC)}
\label{sec:org6759e65}
\begin{itemize}
\item input: n维空间中的两个点集, \(S\), \(Q\).
\item output: 对 \(\forall a \in Q\), 计算出 \(\mbox{PC}(a) = |\{ b\in S| \bigwedge_{i=1}^n b_i\leq a_i\}|\).
\end{itemize}
\subsection{n维偏序问题}
\label{sec:orge3451bb}
\subsubsection{n维偏序问题的一般形式 (ND Partial Order Count Problem \(\Rightarrow\) POC)}
\label{sec:orgaf7d76b}
\begin{itemize}
\item input: n维空间中的两个点集, \(S\), \(Q\).
\item output: 计算出 \(\mbox{POC} = |\{ (a, b) | a\in S\land b\in Q\land\bigwedge_{i=1}^n b_i\leq a_i\}|\).
\end{itemize}
\subsection{数点问题和偏序问题的关系}
\label{sec:org486e246}
很容易发现, 数点问题和偏序问题存在如下关系:
\[
    \mbox{POC} = \sum_{x \in Q} \mbox{PC}(x)
   \]
略线性的部分, 这篇文章中给出的方法, 在这两个问题上均能取得相同的复杂度.
\subsection{树状数组 (ArrayTree)}
\label{sec:orgf9255e1}
树状数组是处理上面两类问题时常用的数据结构.
这里只需要知道树状数组可以用来维护一个一维点的集合 \(S\), 并且具有如下能力:
\begin{itemize}
\item \(\mbox{Insert}(x)\): 将一个一维点 \(x\) 加入其维护的点集中, \(\mathcal{O}(\log \mbox{ub})\).
\item \(\mbox{Query}(x)\): 返回 \(|\{y\in S| y\leq x\}|\), \(\mathcal{O}(\log \mbox{ub})\).
\item 其中, \(\mbox{ub}\) 表示 \(S\) 中点的坐标的变化范围的上界.
\end{itemize}
\begin{verbatim}
// const int n = |S|;
int lowbit (int x) { return x & (-x); }
void Init(int *w, int ub) {memset(w, 0, sizeof(int) * ub);}
void Insert (int *w, int x, int ub) { for (; x <= ub; x += lowbit(x)) ++w[x]; }
int Query (int *w, int x) { 
    int ret = 0;
    for (; x >= 1; x -= lowbit(x)) ret += w[x];
    return ret;
}
\end{verbatim}

\section{Solution}
\label{sec:orgb782e7c}
\subsection{一些数据表示的约定}
\label{sec:org5d30920}
\begin{itemize}
\item P[i].inS 表示P[i]\(\in S\), 否则P[i]\(\in Q\).
\item P[i].id 表示输入时给出的点的顺序 (因为之后会对这些点进行排序, 输出的时候会以这个顺序为参照)
\end{itemize}

\subsection{1维数点问题}
\label{sec:orgb8c3ad4}
\begin{verbatim}
int cmp1 (const Point &a, const Point &b) { return a.w[1] < b.w[1]; }
int cmpid (const Point &a, const Point &b) { return a.id < b.id; }
// P: point set, PC: answer
void PC_1D(Point *P, int *PC, int l, int r) {
    sort(P+l, P+r+1, cmp1); // 按第一维排序
    for (int i = l, cnt = 0; i <= r; i++) 
        if (P[i].inS) ++cnt;
        else PC[P[i].id] = cnt;
}
\end{verbatim}
complexity: \(\mathcal{O}(n\log n)}\).
\subsection{2维数点问题}
\label{sec:org8715413}
\begin{verbatim}
int cmp2 (const Point &a, const Point &b) { return a.w[2] < b.w[2]; }
// P: point set, PC: answer, w: ArrayTree, ub: upperbound for x
void PC_2D(Point *P, int *PC, int *w, int ub, int l, int r) {
    sort(P+l, P+r+1, cmp2); // 按第二维排序
    for (int i = l; i <= r; i++) {
        if (P[i].inS) Insert(w, P[i].w[1], ub);
        else PC[P[i].id] = Query(w, P[i].w[1]); // 存在相同元素时需要修改
    }
}
\end{verbatim}
complexity: \(\mathcal{O}(n\log n)}\).
\subsection{3维数点问题}
\label{sec:org96f5cfe}
\begin{itemize}
\item 按第三维排序
\item 递归解决 \([l, m]\).
\item 计算 \([l, m]\) 的 \(S\) 中的点对 \([m+1, r]\) 的 \(Q\) 中的点的贡献.
因为第三维的相对顺序固定了, 所以问题退化为了一个二维的数点问题 (\(S\), \(Q\) 和原问题不一样).
\item 递归解决 \([m+1, r]\).
\end{itemize}
\begin{verbatim}
int cmp3 (const Point &a, const Point &b) { return a.w[3] < b.w[3]; }
void
PC_3D(Point *P, Point *P_aux, int *PC, int *PC_aux, int *w, int ub, int l, int r) {
    if (l == r) return;
    Init(w, ub);
    sort(P+l, P+r+1, cmp3); // 按第三维排序
    int mid = (l + r) >> 1;
    PC_3D(P, P_aux, PC, PC_aux, w, ub, l, mid);
    for (int i = l; i <= r; i++) {
        P_aux[i] = P[i];
        /* 保证右侧的点不会贡献答案 */
        if (i >= mid+1) P_aux[i].inS = false;
        PC_aux[P_aux[i].id] = 0;
    }
    PC_2D(P_aux, PC_aux, w, ub, l, r);
    for (int i = mid+1; i <= r; i++)
        if (!P[i].inS) PC[P[i].id] += PC_aux[P[i].id];
    PC_3D(P, P_aux, PC, PC_aux, w, ub, mid+1, r);
}
\end{verbatim}
complexity:
\(T(n) = 2T(\lfloor\frac{n}{2}\rfloor) + \mathcal{O}(n\log n) = \mathcal{O}(n\log^2 n)\).
\subsection{n维数点问题}
\label{sec:orge56b6de}
按3维数点问题的思路, 可以不停的利用分治策略, 来将 \(d\) 维的问题转化为 \(d-1\) 维上的问题 (对2维一样成立, 这样可以不用树状数组).
\begin{verbatim}
int d; // 最外侧维数
struct Point { int w[maxd], id; bool inS; };
int cmp(const Point &a, const Point &b) { 
    if (a.w[d] == b.w[d]) return a.inS > b.inS;
    return a.w[d] < b.w[d]; 
}

Point P[maxd][maxn];
int PC[maxd][maxn];
void PCD(int l, int r, int &d, bool needSort = true) {
    if (l == r) return;
    if (needSort) sort (P[d]+l, P[d]+r+1, cmp);
    if (d == 1) {
        for (int i = l, cnt = 0; i <= r; i++) 
            if (P[d][i].inS) ++cnt;
            else PC[d][P[d][i].id] = cnt;
        return;
    }
    int mid = (l + r) >> 1;
    PCD(l, mid, d, false);
    for (int i = l; i <= r; i++) {
        P[d-1][i] = P[d][i];
        /* 防止右侧的点对答案贡献 */
        if (i >= mid+1) P[d-1][i].inS = false;
        PC[d-1][P[d][i].id] = 0;
    }
    --d; PCD(l, r, d); ++d;
    for (int i = mid+1; i <= r; i++) 
        if (!P[d][i].inS) PC[d][P[d][i].id] += PC[d-1][P[d][i].id];
    PCD(mid+1, r, d, false);
}

\end{verbatim}
根据主定理, \(T(n, d) = T(n, d-1)\log n\).
complexity: \(T(n, d) = \mathcal{O}(n\log^{d-1}n)\).

\section{Summary}
\label{sec:org6c08a6d}
通过这篇 Review, 给出了一种解决 \(n\) 个点的 \(d\) 维 数点/偏序 问题的一般方法 (\(\mathcal{O}(n\log^{d-1}n)\)). 
并且这种方法不会用到任何复杂的数据结构(包括树状数组).
\end{document}