\documentclass{article}
\usepackage[usenames]{xcolor}
\usepackage{amsmath, amssymb, amsthm, unicode-math}
\usepackage{fontspec, tgpagella, tgheros, tgcursor}
\setmathfont{texgyrepagella-math.otf}
\setmathfont[range=\setminus]{Asana-Math.otf}
\setmainfont{TeX Gyre Pagella}
\setsansfont{TeX Gyre Heros}
\setmonofont{TeX Gyre Cursor}
\usepackage[utf8]{inputenc}
\usepackage{algorithm2e}
\usepackage{tikz}
\usepackage{hyperref}
\hypersetup{
  colorlinks=true,
  linkcolor=blue,
  filecolor=magenta,
  urlcolor=cyan,
}
\usepackage[letterpaper]{geometry}
\newgeometry{
  textheight=9in,
  textwidth=6.5in,
  top=1in,
  headheight=14pt,
  headsep=25pt,
  footskip=30pt
}

\title{Introduction to the Holographic Transformation}
\author{Xiaoyu Chen}
\date{}

\newtheorem{definition}{Definitin}[section]
\newtheorem{fact}{Fact}[section]
\newtheorem{corollary}{Corollary}[section]
\newtheorem{remark}{Remark}[section]
\newtheorem{theorem}{Theorem}[section]

\def\<{\left\langle}
\def\>{\right\rangle}

\begin{document}
\maketitle

\section{Background for Kronecker Products}
\begin{definition}
  Let $A \in \mathbb{R}^{m\times n}$, $B \in \mathbb{R}^{p\times q}$,
  the \textbf{Kronecker Product} of $A$ and $B$ is defined as
  \begin{align*}
    A \otimes B \triangleq \left[
    \begin{array}{ccc}
      a_{11}B & \cdots & a_{1n}B \\
      \vdots & \ddots & \vdots \\
      a_{m1}B & \cdots & a_{mn}B
    \end{array}
    \right] \in \mathbb{R}^{mp \times nq}
  \end{align*}
\end{definition}

\begin{fact}
  Some basic properties:
  \begin{itemize}
  \item $A\otimes (B + C) = A\otimes B + A\otimes C$
  \item $(B+C)\otimes A = B\otimes A + C\otimes A$
  \item $(kA)\otimes B = A\otimes (kB) = k(A\otimes B)$
  \item $(A\otimes B)\otimes C = A \otimes (B\otimes C)$
  \item $A\otimes 0 = 0\otimes A = 0$
  \end{itemize}
\end{fact}

For convenience, sometimes people use the following notation.
\begin{definition}
  Let $A \in \mathbb{R}^{m\times n}$.
  For any $k \in \mathbb{Z}^+$, let $A^{\otimes k} \triangleq \underbrace{A\otimes A \otimes \cdots \otimes A}_{k} \in \mathbb{R}^{m^k \times n^k}$.
\end{definition}

\begin{fact}
  $(A\otimes B)^T = A^T \otimes B^T$.
\end{fact}

\begin{fact}
  Let $A \in \mathbb{R}^{m \times n}$, $B \in \mathbb{R}^{r\times s}$, $C \in \mathbb{R}^{n\times p}$, $D\in\mathbb{R}^{s\times t}$, then
  \[(A\otimes B)(C\otimes D) = AC\otimes BD (\in \mathbb{R}^{mr \times pt})\]
\end{fact}
\begin{proof}
  One could verify this by:
  \begin{align*}
    (A\otimes B)(C\otimes D)
    &= \left[
      \begin{array}{ccc}
        a_{11}B & \cdots & a_{1n}B \\
        \vdots & \ddots & \vdots \\
        a_{m1}B & \cdots & a_{mn}B
      \end{array} \right] 
      \left[
      \begin{array}{ccc}
          c_{11}D & \cdots & c_{1p}D \\
          \vdots & \ddots & \vdots \\
          c_{n1}D & \cdots & c_{np}D
      \end{array} \right]  \\
    &= \left[
      \begin{array}{ccc}
        \sum_{k=1}^n a_{1k}c_{k1}BD & \cdots & \sum_{k=1}^na_{1k}c_{kp}BD \\
        \vdots & \ddots & \vdots \\
        \sum_{k=1}^n a_{mk}c_{k1}BD & \cdots & \sum_{k=1}^n a_{mk}c_{kp}BD
      \end{array}
      \right] \\
    &= AC \otimes BD . \qedhere
  \end{align*}
\end{proof}

\begin{corollary}
  If $A$ and $B$ are non-singular, then $(A\otimes B)^{-1} = A^{-1}\otimes B^{-1}$.
\end{corollary}
\begin{proof}
  $(A\otimes B)(A^{-1}\otimes B^{-1}) = I\otimes I = I$.
\end{proof}

\begin{corollary}
  Let $A \in \mathbb{R}^{m\times n}$, $B \in \mathbb{R}^{p\times q}$ and $k \in \mathbb{Z}^+$ be any number, then $A^{\otimes k} B^{\otimes k} = (AB)^{\otimes k}$.
\end{corollary}
\begin{proof}
  When $k = 1$ , $AB = AB$ holds trivially.
  When $k > 1$ , we could use the induction:
  \begin{align*}
    A^{\otimes k}B^{\otimes k} &= (A \otimes A^{\otimes k-1})(B \otimes B^{\otimes k-1}) \\
    &= AB \otimes A^{\otimes k-1}B^{\otimes k-1}. \qedhere
  \end{align*}
\end{proof}

\begin{corollary}
  If $A$ is non-singular, then for any $k \in \mathbb{Z}^+$, we have $\left(A^{\otimes k}\right)^{-1} = \left(A^{-1}\right)^{\otimes k}$.
\end{corollary}

\section{Holant Problem and Holographic Transformation}
Let $G = (V, E)$ be any graph.
The incident graph $B = ((V, E), H)$ of $G$ is a bipartite graph, where for any $v\in V, e \in E$, there is an edge (in $H$) between them iff $v$ is an end-point of $e$ in $G$.
The Holant problem is defined on the incident graph $B$.

\begin{definition}[Holant Problem]
  Let $B = ((V, E), H)$ be a bipartite graph.
  For any vertex $x \in V\cup E$, there is a function $f_x: \{0, 1\}^{d(x)} \to \mathbb{R}$, where $d(x)$ is the degree of $x$.
  Then, the Holant problem is to calculate the following formula:
  \begin{align*}
    \mathrm{Holant}_B
    &\triangleq \<\bigoplus_{v\in V}f_v, \bigoplus_{e\in E} f_e\>_{\sigma \in \{0, 1\}^H} \\
    &= \sum_{\sigma \in \{0, 1\}^H} \left(\prod_{v\in V}f_v(\sigma |_{H(v)})\right) \left(\prod_{e\in E}f_e(\sigma |_{H(e)})\right),
  \end{align*}
  where $\sigma |_{H(v)}$ and $\sigma |_{H(e)}$ are generated by restricting the configuration $\sigma$ into $H(v)$ and $H(e)$, respectively.
  Here, we note that $H(v)$ and $H(e)$ are the edges incident to $v$ and $e$ in $H$, respectively.
\end{definition}

\begin{remark}
  In the bipartite graph $B$, any edge $h \in H$ is actually a half edge in the original graph $G$.
  And obviously, for any $u, v\in V$, we have $H(u) \cap H(v) = \emptyset$.
  Similarlly, for any $e, f \in E$, we have $H(e) \cap H(f) = \emptyset$.
  Finally, using the fact that $\bigcup_{v\in V}H(v) = H$ and $\bigcup_{e\in E}H(e) = H$, we know that $\bigotimes_{v\in V}f_v$ and $\bigotimes_{e\in E}f_e$ are functions defined on $\{0, 1\}^H$.
\end{remark}

Here is an example of holographic transformation.
Let $M \in \mathbb{R}^{2\times 2}$ be a non-singular matrix.
Then 
\begin{align*}
  \mathrm{Holant}_B
  &= \left(\bigotimes_{v\in V} f_v\right)^T\left(\bigotimes_{e\in E} f_e\right) \\
  &= \left(\bigotimes_{v\in V} f_v^T\right) M^{\otimes |H|} (M^{-1})^{\otimes |H|}\left(\bigotimes_{e\in E} f_e\right) \\
  &= \left(\bigotimes_{v \in V} f_v^T M^{\otimes d(v)}\right) \left(\bigotimes_{e\in E} (M^{-1})^{\otimes d(e)}f_e\right)
\end{align*}
So, the Holant problem $\<\bigotimes_{v\in V}f_v, \bigotimes_{e\in E} f_e\>$ is equivalent to the Holant problem $\<\bigotimes_{v\in V} (M^T)^{\otimes d(v)}f_v, \bigotimes_{e\in E} (M^{-1})^{\otimes d(e)} f_e\>$.
This kind of transformation is called holographic transformation, they really do nothing but create equivalent problems.

\end{document}

%%% Local Variables:
%%% mode: latex
%%% TeX-master: t
%%% End:
