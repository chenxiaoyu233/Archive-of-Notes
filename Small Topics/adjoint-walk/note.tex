\documentclass{article}
\usepackage[usenames]{xcolor}
\usepackage{amsmath, amssymb, amsthm, unicode-math}
\usepackage{fontspec, tgpagella, tgheros, tgcursor}
\setmathfont{texgyrepagella-math.otf}
\setmathfont[range=\setminus]{Asana-Math.otf}
\setmainfont{TeX Gyre Pagella}
\setsansfont{TeX Gyre Heros}
\setmonofont{TeX Gyre Cursor}
\usepackage[utf8]{inputenc}
\usepackage{algorithm2e}
\usepackage{tikz}
\usepackage{hyperref}
\hypersetup{
  colorlinks=true,
  linkcolor=blue,
  filecolor=magenta,
  urlcolor=cyan,
}
\usepackage[letterpaper]{geometry}
\newgeometry{
  textheight=9in,
  textwidth=6.5in,
  top=1in,
  headheight=14pt,
  headsep=25pt,
  footskip=30pt
}

\title{Adjoint Walks}
\author{Xiaoyu Chen}
\date{}

\newtheorem{define}{Definition}
\newtheorem{fact}{Fact}
\def\<{\langle}
\def\>{\rangle}
\def\Var{\mathrm{Var}}
\def\E{\mathbb{E}}

\begin{document}
\maketitle

\section{Adjoint walk on two disjoint spaces}

Let $\Omega_1$ and $\Omega_2$ be two disjoint state spaces.
Let $\pi_1$ and $\pi_2$ be two distribution on $\Omega_1$ and $\Omega_2$, respectively.
Let $P_{12} \in \mathbb{R}^{\Omega_1 \times \Omega_2}$ and $P_{21} \in \mathbb{R}^{\Omega_2 \times \Omega_1}$ be two stochastic matrices.

\begin{define}
  If for any $x\in \Omega_1$ and $y \in \Omega_2$, we have $\pi_1(x)P_{12}(x, y) = \pi_2(y)P_{21}(y, x)$, then we say $P_{12}$ and $P_{21}$ are \emph{adjoint random walk} over $\pi_1$ and $\pi_2$.
\end{define}

\begin{fact}
  $P_{12}$ and $P_{21}$ are \emph{adjoint random walk} over $\pi_1$ and $\pi_2$ iff
  \[ \forall f: \Omega_1 \to \mathbb{R}, \forall g: \Omega_2 \to \mathbb{R} \;,\; \<f, P_{12}g\>_{\pi_1} = \< P_{21}f, g \>_{\pi_2}.  \]
\end{fact}
\begin{proof}
  $\Rightarrow$:
  \begin{align*}
    \< f, P_{12} g\>_{\pi_1}
    = \sum_{x\in\Omega_1} \pi_1(x) f(x) \left[P_{12}g\right](x)
    = \sum_{x\in\Omega_1}\sum_{y\in\Omega_2} \pi_1(x)&P_{12}(x, y) f(x) g(y) \\
    &\|\\
    \< P_{21}f, g \>_{\pi_2}
    = \sum_{y\in\Omega_2} \pi_2(y) \left[P_{21}f\right](y) g(y)
    = \sum_{x\in\Omega_1}\sum_{y\in\Omega_2} \pi_2(y)&P_{21}(y, x) f(x) g(y)
  \end{align*}
  $\Leftarrow$: Let $x \in \Omega_1$ and $y \in \Omega_2$ be arbitrary elements in $\Omega_1$ and $\Omega_2$, respectively.
  Let $\delta_x : \Omega_1 \to \{0, 1\}$ and $\delta_y : \Omega_2 \to \{0, 1\}$, be the indicator function of $x$ and $y$, respectively.
  Then,
  \begin{align*}
    \<\delta_x, P_{12}\delta_y\>_{\pi_1} &= \<P_{21}\delta_x, \delta_y\>_{\pi_2} \Rightarrow \pi_1(x)P_{12}(x, y) = \pi_2(y)P_{21}(y, x) \qedhere
  \end{align*}
\end{proof}

\begin{fact}
  If $P_{12}$ and $P_{21}$ are \emph{adjoint random walk} over $\pi_1$ and $\pi_2$, then
  \[\pi_1P_{12} = \pi_2 \mbox{ and } \pi_2P_{21} = \pi_1\]
\end{fact}

\begin{fact}
  Let $P_{12}$ and $P_{21}$ be \emph{adjoint random walk} over $\pi_1$ and $\pi_2$.
  Then for any distribution $\nu_1$ over $\Omega_1$, we have,
  \begin{align} \label{eq:1->2}
    \frac{\nu_1 P_{12}}{\pi_1 P_{12}} = P_{21} \frac{\nu_1}{\pi_1}.
  \end{align}
  Symmetrically, for any distribution $\nu_2$ over $\Omega_2$, we have,
  \begin{align}
    \frac{\nu_2 P_{21}}{\pi_2 P_{21}} = P_{12} \frac{\nu_2}{\pi_2}.
  \end{align}
\end{fact}
\begin{proof}
  We only proof \eqref{eq:1->2}.
  For any $y \in \Omega_2$, we have
  \begin{align*}
    \frac{\nu_1 P_{12}}{\pi_1 P_{12}} (y) = \frac{\left[\nu_1P_{12}\right](y)}{\pi_2(y)}
  \end{align*}
  On the other hand, we have
  \begin{align*}
    P_{21}\frac{\nu_1}{\pi_1} (y)
    &= \sum_{x\in\Omega_1} P_{21}(y, x) \frac{\nu_1(x)}{\pi_1(x)} \\
    &= \sum_{x\in\Omega_1} \pi_2(y)P_{21}(y, x) \frac{\nu_1(x)}{\pi_2(y)\pi_1(x)} \\
    &= \sum_{x\in\Omega_1} \pi_1(x)P_{12}(x, y) \frac{\nu_1(x)}{\pi_2(y)\pi_1(x)} \\
    &= \sum_{x\in\Omega_1} \pi_1(x)P_{12}(x, y) \frac{\nu_1(x)}{\pi_2(y)\pi_1(x)} \\
    &= \frac{\left[\nu_1P_{12}\right](y)}{\pi_2(y)} \qedhere
  \end{align*}
\end{proof}

\section{Understand variance and Dirichlet form}
For any distribution $\pi: \Omega \to \mathbb{R}$, and any function $f: \Omega \to \mathbb{R}$, the variance is defined as
\[\Var_\pi[f] \triangleq \E_\pi[f^2] - \E_\pi[f]^2.\]
We could represent the variance in a more general way:
\[\Var_\pi[f] = \< f - \pi(f), f - \pi(f)\>_\pi,\]
where, for conveniene, we use $\pi(f)$ to stand for $\pi^T f$, which is equivalent to $\E_\pi[f]$.

Then, we define the projection operator as follows.
\begin{define}
  We define $J_\pi \triangleq \mathbb{1}\pi^T$, which is a matrix with each row equals to $\pi^T$.
\end{define}
\begin{fact}
  For any $f: \Omega \to \mathbb{R}$, it is easy to verify that $J_\pi f = \<f, \mathbb{1}\>_\pi\mathbb{1}$.
  So, $J_\pi$ is a projection matrix that maps $f$ to $f^{\mathbb{1}}$.
\end{fact}
Then, for the variance, we have
\begin{align*}
  \Var_\pi[f]
  &= \<f - J_\pi f, f - J_\pi f\>_\pi \\
  &= \<f^{\perp_\pi \mathbb{1}}, f^{\perp_\pi \mathbb{1}}\>_\pi \\
  &= \<f, (I - J_\pi) f\>_\pi \\
  &= \mathcal{E}_{J_\pi}(f, f)
\end{align*}
\end{document}

%%% Local Variables:
%%% mode: latex
%%% TeX-master: t
%%% End:
