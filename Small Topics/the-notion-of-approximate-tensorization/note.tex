\documentclass{article}
\usepackage[usenames]{xcolor}
\usepackage{amsmath, amssymb, amsthm, unicode-math}
\usepackage{fontspec, tgpagella, tgheros, tgcursor}
\setmathfont{texgyrepagella-math.otf}
\setmathfont[range=\setminus]{Asana-Math.otf}
\setmainfont{TeX Gyre Pagella}
\setsansfont{TeX Gyre Heros}
\setmonofont{TeX Gyre Cursor}
\usepackage[utf8]{inputenc}
\usepackage{algorithm2e}
\usepackage{tikz}
\usepackage{hyperref}
\hypersetup{
  colorlinks=true,
  linkcolor=blue,
  filecolor=magenta,
  urlcolor=cyan,
}
\usepackage[letterpaper]{geometry}
\newgeometry{
  textheight=9in,
  textwidth=6.5in,
  top=1in,
  headheight=14pt,
  headsep=25pt,
  footskip=30pt
}

\title{The Notion of Approximate Tensorization}
\author{Xiaoyu Chen}
\date{}

\newtheorem{define}{Definition}[section]
\newtheorem{theorem}{Theorem}[section]
\newtheorem{fact}{Fact}[section]

\def\<{\langle}
\def\>{\rangle}
\def\Var{\mathrm{Var}}
\def\E{\mathbb{E}}

\begin{document}
\maketitle
\section{Background}
Recently, a novel notation called \textbf{Approximate Tensorization} was used to build the rapid mixing of Glauber dynamics for general spin systems in the uniqueness region\cite{CLV20-1}.
They use this notation to establish a (constant factor) connection between the modified log-sobolev constant of the block dynamics and the single site dynamics on general spin systems.

For spin system, it turns out that this notation is equivalant to the well-known poincar{\'e} inequality and log-sobolev inequality, while it contains more intuition for spin systems such that one could use the inner properties of spin systems more easily.

For more information of this notation, one could refer to \cite{caputo2015approximate, caputo2020block}.

\section{Approximate Tensorization of Variance}
Lets start this notation from the easier part, namely, variance and the corresponding constant i.e. spectra gap $1 - \lambda_2$.

\begin{fact}[Variance]
  For $\forall f: \Omega \to \mathbb{R}$, we have
  \[\Var_\mu f = \frac{1}{2}\sum_{\sigma,\tau\in\Omega} \mu(\sigma)\mu(\tau)\left(f(\sigma) - f(\tau)\right)^2\]
\end{fact}

\begin{fact}[Dirichlet Form]
  For $\forall f: \Omega \to \mathbb{R}$, we have
  \[\<f, (I - P) f\>_\mu = \frac{1}{2} \sum_{\sigma, \tau\in\Omega}\mu(\sigma)P(\sigma, \tau)\left(f(\sigma) - f(\tau)\right)^2\]
\end{fact}

\subsection{Single Site Dynamics}
The equivalance between approximate tensorization and the poincar{\'e} inequality relies on a careful manipulation on Dirichlet form, which is stated as follows.

\begin{align*}
  \<f, (I - P)f\>_\mu
  &= \frac{1}{2} \sum_{\sigma, \tau \in\Omega} \mu(\sigma)P(\sigma, \tau)\left(f(\sigma) - f(\tau)\right)^2 \\
  &= \frac{1}{2} \sum_{\sigma\in\Omega} \mu(\sigma) \left(\sum_{u\in V} \sum_{\substack{\tau_u\in\Omega^{\sigma_{V\setminus u}}_u \\ \tau_{V\setminus u} = \sigma_{V\setminus u}}} \frac{1}{n} \mu^{\sigma_{V\setminus u}}_u(\tau_u)\right)\left(f(\sigma) - f(\tau)\right)^2 \\
  &= \frac{1}{n} \sum_{u\in V} \sum_{\gamma\in \Omega_{V\setminus u}} \mu_{V\setminus u} (\gamma) \cdot \frac{1}{2} \sum_{x\in \Omega^{\gamma}_u} \mu^\gamma_u(x) \sum_{y\in\Omega^\gamma_u}\mu^\gamma_u(y) (f_\gamma(x) - f_\gamma(y))^2 \\
  &= \frac{1}{n}\sum_{u\in V} \sum_{\gamma\in\Omega_{V\setminus u}} \mu_{V\setminus u}(\gamma) \cdot \Var_{\mu^\gamma_u} (f_\gamma) \\
  &= \frac{1}{n} \sum_{u\in V} \mu\left[\Var_u(f)\right]
\end{align*}

\begin{fact}[Equivalence for Single Site Dynamics]
  Recall that the poincar{\'e} inequality is
  \[\forall f, (1 - \lambda_2) \cdot \Var_\mu (f) \leq \<f, (I - P)f\>_\mu\]
  And we could restate it as follows
  \[\forall f, (1 - \lambda_2) \cdot \Var_\mu (f) \leq \frac{1}{n}\sum_{u} \mu [\Var_u (f)]\]
\end{fact}

\subsection{Block  Dynamics}
Let $P^\lor_{n, n - \ell}$ denote the block dynamics where we choose $\ell$ vertices uniformly at random, and then sample a configuration on this $\ell$ vertices according to the conditional distribution.

Note that the above approach for single site dynamics could also be applied to block dynamics.

\begin{align*}
  \<f, (I - P^{\lor}_{n,n-\ell})f\>_\mu
  &= \frac{1}{2} \sum_{\sigma, \tau \in\Omega} \mu(\sigma)P^\lor_{n,n-\ell}(\sigma, \tau)\left(f(\sigma) - f(\tau)\right)^2 \\
  &= \frac{1}{2} \sum_{\sigma\in\Omega} \mu(\sigma) \left(\sum_{S \in \binom{V}{\ell}} \sum_{\substack{\tau_S\in\Omega^{\sigma_{V\setminus S}}_u \\ \tau_{V\setminus S} = \sigma_{V\setminus S}}} \frac{1}{\binom{n}{\ell}} \mu^{\sigma_{V\setminus S}}_S(\tau_S)\right)\left(f(\sigma) - f(\tau)\right)^2 \\
  &= \frac{1}{\binom{n}{\ell}} \sum_{S\in \binom{V}{\ell}} \sum_{\gamma\in \Omega_{V\setminus S}} \mu_{V\setminus S} (\gamma) \cdot \frac{1}{2} \sum_{\alpha\in \Omega^{\gamma}_S} \mu^\gamma_S(\alpha) \sum_{\beta\in\Omega^\gamma_S}\mu^\gamma_S(\beta) (f_\gamma(\alpha) - f_\gamma(\beta))^2 \\
  &= \frac{1}{\binom{n}{\ell}}\sum_{S\in \binom{V}{\ell}} \sum_{\gamma\in\Omega_{V\setminus S}} \mu_{V\setminus S}(\gamma) \cdot \Var_{\mu^\gamma_S} (f_\gamma) \\
  &= \frac{1}{\binom{n}{\ell}} \sum_{S\in \binom{V}{\ell}} \mu\left[\Var_S(f)\right]
\end{align*}

\begin{fact}[Equivalence for Block Dynamics]
  Recall that the poincar{\'e} inequality for the block dynamics is
  \[\forall f, (1 - \lambda_2) \Var_\mu f \leq \<f, (I - P^\lor_{n,n-\ell})f\>_\mu\]
  And we could restate it as follows
  \[\forall f, (1 - \lambda_2) \Var_\mu f \leq \frac{1}{\binom{n}{\ell}} \sum_{S\in\binom{V}{\ell}} \mu[\Var_S(f)]\]
\end{fact}

In the block case, the \textbf{Approximate Tensorization} is some times called \textbf{Uniform Block Factorization} (see \cite{CLV20-1}) and it is a special case of \textbf{Block Factorization} given in \cite{caputo2020block}.

\subsection{Connection with Local-to-Global Argument}

\begin{define}
  For any function $f^{(s)}: X_s \to \mathbb{R}$, and $r < s$, we could define another function as
  $f^{(r)} \triangleq P_r^\uparrow P_{r+1}^\uparrow\cdots P_{s-1}^\uparrow f^{(s)}$
  Which could also be denoted as
  \[f^{(r)} = [\pi_r \leftrightarrow \pi_{r+1}] \cdots [\pi_{s-1} \leftrightarrow \pi_s]f^{(s)}\]
\end{define}
\begin{fact}
  $\pi_s f^{(s)} = \pi_r f^{(r)}$
\end{fact}

\begin{define}
  Let $f^{(s)}$ be a function defined on $X_s$.
  Let $\gamma \in X_r$, $\alpha \in X^\gamma_{s-r}$, then we denote $f^{(s)}(\gamma\cup \alpha)$ as $f^{(s - r)}_\gamma(\alpha)$
\end{define}

\begin{fact}
  $\E_{\pi^\gamma_\ell}[f^{(\ell)}_\gamma] = f^{(\ell)}(\gamma)$
\end{fact}

\begin{fact}
  \[\frac{1}{\binom{n}{\ell}} \sum_{S\in\binom{V}{\ell}} \mu[\Var_S(f)] = \Var_{\pi_n}(f^{(n)}) - \Var_{\pi_{n-\ell}}(f^{(n-\ell)})\]
\end{fact}

\clearpage
\bibliographystyle{alpha}
\bibliography{note}

\end{document}

%%% Local Variables:
%%% mode: latex
%%% TeX-master: t
%%% End:
