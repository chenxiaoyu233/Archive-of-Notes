\documentclass{article}
\usepackage{geometry}
\geometry{a4paper, scale=0.8}
\usepackage{ctex}
\usepackage{amsmath, amssymb, amsthm}
\everymath{\displaystyle}
\newtheorem{define}{Define}
\newtheorem{fact}{Fact}
\newtheorem{property}{Property}
\newtheorem{theorem}{Theorem}
\newtheorem{lemma}{Lemma}
\newtheorem{remark}{Remark}

\usepackage{tabularx}
\usepackage{hyperref}
\hypersetup{
    colorlinks=true,
    linkcolor=blue,
    filecolor=magenta,      
    urlcolor=cyan,
}

\usepackage{color}

\title{Notes for Convex Functions}
\author{Xiaoyu Chen}
\date{}

\begin{document}
\maketitle
这份笔记是在阅读 <<Log-Concave Polynomial II>> 时的一个衍生产物.
为了能够更好了的理解 strongly log-concave polynomial 为什么能够在采样问题中带来 rapix mixing, 所以我专门去学习了一下 Convex functions.
主要是想在 concave 和 Hessian matrix 的半负定之间建立起比较严谨的联系
这篇文章的主要内容来自于知乎大佬的\href{http://yangzhou301.xyli.me/2016/03/14/826442654/}{博客}

\section{定义}
在本文中,  我们谈论的是 convex funcion (凸函数), concave function (凹函数) 其实区别不大, 只是方向不一样而已.
\begin{define}[convex funcion]
  对于 $f: \mathbb{R}^n \to \mathbb{R}$, 如果 对于 $f$ 定义域中的任意两个自变量 $x_1, x_2$ 和 任意 $\theta \in [0, 1]$, 都有
  \[f(\theta x_1 + (1 - \theta)x_2) \leq \theta f(x_1) + (1 - \theta)f(x_2)\]
  则说, $f$ 是\textbf{凸函数}. (凹函数直接改变不等号的方向即可)
\end{define}

我们希望有方便的工具来判断一个函数是否是凸函数, 这就需要找到一些充要条件.
下面给出两个重要的判别条件, 我们比较关心的是后一个, 不过我目前找到的资料中, 这两个判据的证明都是有依赖关系的.
\section{一阶判据}
\begin{theorem}[First Order Condition]
  函数 $f$ 是凸函数, 当且仅当, $f$ 的定义域上的任意两点 $x_1, x_2$, 都有:
  \[f(x_1) \geq f(x_2) + \nabla f(x_2)(x_1 - x_2)\]

  $\triangle$ 注: 这里 $x_1 - x_2$ 是向量相减, 图形上看, 是一个从 点 $x_2$ 指向 点 $x_1$ 的一个向量.
\end{theorem}
\begin{remark}[几何意义]
  上面的不等式中, 如果$x_1$ 是一个自由变量, 则$f(x_2) + \nabla f(x_2) (x_1 - x_2)$描述了一个$n+1$ 维空间中的 超平面.
  这个不等式的意思就是\textbf{整个函数 $f$ 都在这个超平面上方}. 这一点还是比较直观的.
\end{remark}
\begin{proof}[proof]
  \textcolor{blue}{\emph{凸函数 $\Rightarrow$ $f(x_1) \geq f(x_2) + \nabla f(x_2)(x_1 - x_2), \forall x_1, x_2 \in\mathrm{dom}\,f$}:}\\
  \begin{align*}
    f(\theta x_1 + (1 - \theta)x_2) &\leq \theta f(x_1) + (1 - \theta)f(x_2) \\
    f(x_2 + \theta(x_1 - x_2)) &\leq f(x_2) + \theta(f(x_1) - f(x_2)) \\
    f(x_2 + \theta(x_1 - x_2)) - f(x_2) &\leq  \theta(f(x_1) - f(x_2)) \\
    \frac{f(x_2 + \theta(x_1 - x_2)) - f(x_2)}{\theta} &\leq f(x_1) - f(x_2) \\
    f(x_1) &\geq f(x_2) + \frac{f(x_2 + \theta(x_1 - x_2)) - f(x_2)}{\theta} 
  \end{align*}
  因为这个式子需要对所有的 $x_1, x_2, \theta$ 成立, 我们不妨让 $\theta$ 逼近 $0$.
  来考虑下面的极限:
  \[\lim_{\theta\to 0}\frac{f(x_2 + \theta(x_1 - x_2)) - f(x_2)}{\theta}\]
  令 $g(\theta) := f(x_2 + \theta(x_1 - x_2))$ (也就是说 $x(\theta) = x_2 + \theta(x_1 - x_2), g(\theta) = f(x(\theta))\,$), 则上述极限相当于在对 $g$ 求导.
  \begin{align*}
    \lim_{\theta\to 0}\frac{f(x_2 + \theta(x_1 - x_2)) - f(x_2)}{\theta} = \frac{g(\theta) - g(0)}{\theta} = \left.\frac{\mathrm{d}g}{\mathrm{d}\theta}\right|_{\theta = 0}
                                                                         = \frac{\mathrm{d}f}{\mathrm{d}x}\frac{\mathrm{d}x}{\mathrm{d}\theta} 
    &= \left.(x_1 - x_2) \frac{\mathrm{d}f(x)}{\mathrm{d}x}\right|_{x=x_2} \\
    &= \nabla f(x_2)(x_1 - x_2)
  \end{align*}

  $\triangle$ 注: 其实, $\frac{\mathrm{d}f(x_2)}{\mathrm{d}x_2} = \nabla f(x_2)$ 这一步是比较迷的. 不过从全微分的角度, 还是能够解释的, 期待以后补完这个东西的详细定义.
  
  所以, 和上面结合起来就是:
  \[f(x_1) \geq f(x_2) + \nabla f(x_2)(x_1 - x_2)\]

  下面来说明:
  \textcolor{blue}{\emph{$f(x_1) \geq f(x_2) + \nabla f(x_2)(x_1 - x_2)$ $\Rightarrow$ 凸函数}:}\\
  因为左边的不等式对于任何的 $x_1, x_2$ 都是适用的, 所以对于任意的 $\theta$, 令  $x = \theta x_1 + (1 - \theta) x_2$, 我们有如下结论成立:
  \begin{align*}
    \left\{
    \begin{array}{l}
      f(x_1) \geq f(x) + \nabla f(x) (x_1 - x) \\
      f(x_2) \geq f(x) + \nabla f(x) (x_2 - x)
    \end{array}
    \right.
  \end{align*}
  把这两个式子按照 $x_1, x_2$的比例相加, 就有:
  \begin{align*}
    \theta f(x_1) + (1 - \theta)f(x_2) &\geq f(x) + \nabla f(x) (\theta x_1 + (1 - \theta) x_2)  - \nabla f(x) x \\
    \theta f(x_1) + (1 - \theta)f(x_2) &\geq f(x) = f(\theta x_1 + (1-\theta)x_2) \qedhere
  \end{align*}
\end{proof}

\section{二阶判据}
\begin{theorem}[Second-Order Condition]
  函数$f$为凸函数, 当且仅当 对于任意 自变量 $x_0$, 其 Hessian 矩阵 $H(x_0)$ 都是半正定的.
\end{theorem}
\begin{proof}[proof]
  \textcolor{blue}{凸函数 $\Rightarrow$ $\forall x_0, H(x_0)$ 半正定}:\\
  $f$ 在 $x_0$ 点的泰勒展开式(Peano余项)如下:
  \[f\left({x}\right) = f\left( {x}_0 \right) + \nabla f\left( {x}_0 \right)\left( {x} - {x}_0 \right) + \frac{1}{2} \left( {x} - {x}_0 \right)^T {H}\left( {x}_0 \right) \left({x} - {x}_0 \right) + o\left( \left({x} - {x}_0 \right)^T \left( {x} - {x}_0 \right) \right)\]
  我们令 $x = x_0 + d$ ($d$ 为一个微小的向量), 可以将上面的式子改写成下面这样:
  \[f\left( {x}_0 + {d} \right) = f\left( {x}_0 \right) + \nabla f\left( {x}_0 \right){d} + \frac{1}{2} {d}^T {H}\left( {x}_0 \right) {d} + o\left( d^Td \right)\]
  因为$f$是凸函数, 所以由第一判据
  \[f(x_0 + d) \geq f(x_0) + \nabla f(x_0) d\]
  上面两个不等式联合起来可以发现:
  \begin{align*}
    \frac{1}{2} {d}^T {H}\left( {x}_0 \right) {d} + o\left( d^Td \right) &\geq 0 \\
    \lim_{||d||_2 \to 0}\frac{1}{2} \frac{{d}^T {H}\left( {x}_0 \right) {d}}{d^Td} + \frac{o\left( d^Td \right)}{d^Td} &\geq 0 \\
    \lim_{||d||_2 \to 0}\frac{{d}^T {H}\left( {x}_0 \right) {d}}{d^Td} &\geq 0
  \end{align*}
  所以 $H(x_0)$ 半正定.

  \textcolor{blue}{$\forall x_0, H(x_0)$半正定 $\Rightarrow$ 凸函数}:\\
  $f$ 在 $x_0$ 点的泰勒展开式(Lagrange余项余项)如下
  \[f\left( {x} \right) = f\left( {x}_0 \right) + \nabla f\left( {x}_0 \right)\left( {x} - {x}_0 \right) + \frac{1}{2}{\left( {x} - {x}_0 \right)^T}{H}\left( {\bf{\xi }} \right)\left( {x} - {x}_0 \right)\]
  因为 $H(\xi)$ 正定, 所以我们可以立即判断出
  \[f(x) \geq f(x_0) + \nabla f(x_0) (x - x_0), \forall x, x_0 \qedhere\]
\end{proof}

不难发现, concave function 的 Hessian 阵一定是负定的.
\end{document}
%%% Local Variables:
%%% mode: latex
%%% TeX-master: t
%%% End:
