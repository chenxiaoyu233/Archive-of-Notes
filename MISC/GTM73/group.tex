% Created 2019-03-28 Thu 21:54
% Intended LaTeX compiler: pdflatex
\documentclass[11pt]{article}
\usepackage[utf8]{inputenc}
\usepackage[T1]{fontenc}
\usepackage{graphicx}
\usepackage{grffile}
\usepackage{longtable}
\usepackage{wrapfig}
\usepackage{rotating}
\usepackage[normalem]{ulem}
\usepackage{amsmath}
\usepackage{textcomp}
\usepackage{amssymb}
\usepackage{capt-of}
\usepackage{hyperref}
\author{陈小羽}
\date{\today}
\title{Group}
\hypersetup{
 pdfauthor={陈小羽},
 pdftitle={Group},
 pdfkeywords={},
 pdfsubject={},
 pdfcreator={Emacs 26.1 (Org mode 9.1.9)}, 
 pdflang={English}}
\begin{document}

\maketitle
\tableofcontents


\section{Semigroups, Monoids And Groups}
\label{sec:org4701788}
\subsection{Exercise 7 (P29)}
\label{sec:org7a8adbf}
it is easy to see that \((\forall \bar{a}, \bar{b} \in Z_p - \bar{0})\to \bar{a}\bar{b}\in Z_p - \bar{0}\).
And \((\forall \bar{a} \not= 0) \to (a, p) = 1\). so \((\exists x, y \in Z)\land xa + yp = 1\), i.e. \(xa = y^{'}p + 1\).
which means that \(a^{-1} = x\) exists.
\subsection{Exercise 11 (P30)}
\label{sec:org87d9076}
if \((ab)^n = a^nb^n\) for three consequtive integers \(n\).
then, \((ab)^{n-1} = a^{n-1}b^{n-1}\), \((ab)^n = a^nb^n\), \((ab)^{n+1} = a^{n+1}b^{n+1}\).
split \((ab)^{n+1}\), we get:
\begin{align}
     & (ab)^{n+1}         = a^{n+1}b^{n+1} \\
    =& (ab)(ab)^n         = (ab)a^nb^b \\
    =& (ab)^n(ab)         = a^nb^n(ab) \\
    =& (ab)(ab)^{n-1}(ab) = (ab)a^{n-1}b^{n-1}(ab) \\
    =& (ab)^2(ab)^{n-1}   = (ab)^2a^{n-1}b^{n-1} \\
    =& (ab)^{n-1}(ab)^2   = a^{n-1}b^{n-1}(ab)^2
\end{align}
from (1) and (3) we could get:
\begin{align}
     a^nb^n(ab) =& a^{n+1}b^{n+1} \\
         b^nab  =& ab^{n+1} \\
           b^na =& ab^n
\end{align}
from (1) and (2) we could get:
\begin{align}
     (ab)a^nb^n =& a^{n+1}b^{n+1} \\
     ba^n       =& a^nb
\end{align}
from (1) and (4) we could get:
\begin{align}
     (ab)(ab)^{n-1}(ab) =& a^{n+1}b^{n+1} \\
     (ab)a^{n-1}b^{n-1}(ab) =& a^{n+1}b^{n+1} \\
       ba^{n-1}b^{n-1}a =& a^nb^n \\
       ba^na^{-1}b^{-1}b^na =& a^nb^b
\end{align}
from (9), (11) and (15) we could get:
\begin{align}
     a^nba^{-1}b^{-1}ab^n =& a^nb^n \\
        ba^{-1}b^{-1}a    =& e \\
         a^{-1}b^{-1}     =& b^{-1}a^{-1}
\end{align}
\subsection{Exercise 15 (P30)}
\label{sec:org47714cb}
for all \(a\in G\), we could define \(G' = \{ab| \forall b\in G\}\).
because \((\forall a, b, c \in G) ab = ac \to b = c\), we could define a injection function
\(f: G\to G'\) by leting \(f(x) = ax\). It is easy to verify that \(\mbox{Img } f = G'\).
so \(f\) is a bijection and \(G = G'\).
we could look the \(f(x)\) as an edge from \(x\) to \(f(x)\). then we could get
a graph, such that all the verteces in the graph are in a cycle.
so we could get a circle like this: \(aa = f, af = c, ac = \cdots ax = a\).
and we could get \(a^k = a\) from this circle. so \((\forall b\in G) a^kb = ab \to a^{k-1}b = b\).
so we could let \(a^{k-1} = e\) to be the left unit of \(G\).
and by knowing that \((\forall x\in G)(\exists y\in G)\land xy = e\).
we've proved that all the elements in \(G\) has a inverse.
\section{Homomorphism And Subgroups}
\label{sec:orgee29e08}
\subsection{Exercise 1 (P33)}
\label{sec:orgca0baab}
construct a monoid \(G\) as follow:
\begin{center}
\begin{tabular}{llll}
 & a & b & e\\
a & a & a & a\\
b & b & b & b\\
e & a & b & e\\
\end{tabular}
\end{center}
it is easy to find that \(G\) is a monoid.
And we construct a homomorphism \(f: G\to G\) as follow:
\(f(a) = a, f(b) = a, f(e) = a\).
\subsection{Exercise 2 (P33)}
\label{sec:org9825f9f}
\(f(x)f(y) = x^{-1}y^{-1} = (yx)^{-1} \not= (xy)^{-1} = f(xy)\)
\subsection{Exercise 3 (P33)}
\label{sec:org98b28d0}
\subsubsection{non-abelian}
\label{sec:org7409d46}
\(AB = \left(\begin{array}{cc} i & 0 \\ 0 & -i \end{array}\right)\)

\(BA = \left(\begin{array}{cc} -i & 0 \\ 0 & i \end{array} \right)\)
so, \(Q_8\) is not abelian.
\subsubsection{it is a group}
\label{sec:orge4cdea0}
And we know that \(A^4 = B^4 = I\).
Because jBB\((\forall i,j \in Z) A^iB^jB^{4-j}A^{4-i} = I \Rightarrow (A^iB^j)^{-1} = B^{4-j}A^{4-i}\), we claim that all the elements
in this set have a inverse.
\subsubsection{why order 8}
\label{sec:org74c66e6}
you could write a program to check this.
\begin{verbatim}
template <class Type>
struct Matrix {
#define FOR(i, l, r) for (int (i) = (l); (i) <= (r); (i)++)
    int row, col;
    complex<Type> w[maxn][maxn];
    Matrix (int row, int col):row(row), col(col) {
        FOR(i, 1, row) FOR(j, 1, col) w[i][j] = 0; // initalize
    }
    Matrix (const Matrix &other) {
        row = other.row, col = other.col;
        FOR(i, 1, row) FOR(j, 1, col) w[i][j] = other.w[i][j];
    }
    friend Matrix operator * (const Matrix &a, const Matrix &b) {
        assert(a.col == b.row);
        Matrix c(a.row, b.col);
        FOR(i, 1, a.row) FOR(j, 1, b.col) {
            c.w[i][j] = 0;
            FOR(k, 1, a.col) c.w[i][j] += a.w[i][k] * b.w[k][j];
        }
        return c;
    }
    friend bool operator < (const Matrix &a, const Matrix &b) {
        assert(a.row == b.row);
        assert(a.col == b.col);
        FOR(i, 1, a.row) FOR(j, 1, a.col) {
            if (a.w[i][j] != b.w[i][j]) {
                if (a.w[i][j].real() == b.w[i][j].real()) 
                    return a.w[i][j].imag() < b.w[i][j].imag();
                return a.w[i][j].real() < b.w[i][j].real();
            }
        }
        return false;
    }
    friend bool operator == (const Matrix &a, const Matrix &b) {
        if (a.row != b.row || a.col != b.col) return false;
        FOR(i, 1, a.row) FOR(j, 1, a.col) {
            if (a.w[i][j] != b.w[i][j]) return false;
        }
        return true;
    }
    void Log() {
        FOR(i, 1, row) {
            FOR(j, 1, col) {
                printf("%d + %di ", w[i][j].real(), w[i][j].imag());
            } printf("\n");
        }
    }
#undef FOR
};
\end{verbatim}

\begin{verbatim}
void search() {
    initI(); initA(); initB();
    set<Matrix<int> > st; st.clear();
    queue<Matrix<int> > q;
    q.push(I);
    st.insert(I);
    while(!q.empty()) {
        Matrix<int> tt = q.front(); q.pop();
        tt.Log(); puts("");
        Matrix<int> ta = tt * A;
        Matrix<int> tb = tt * B;
        Matrix<int> at = A * tt;
        Matrix<int> bt = B * tt;
        if (!st.count(ta)) {
            st.insert(ta);
            q.push(ta);
        }
        if (!st.count(tb)) {
            st.insert(tb);
            q.push(tb);
        }
        if (!st.count(at)) {
            st.insert(at);
            q.push(at);
        }
        if (!st.count(bt)) {
            st.insert(bt);
            q.push(bt);
        }
    }
    cout << st.size() << endl;
}
\end{verbatim}

\begin{verbatim}
#include <iostream>
#include <cstring>
#include <cstdio>
#include <complex>
#include <map>
#include <set>
#include <queue>
using namespace std;

const int maxn = 10;

// load the Matrix Class
template <class Type>
struct Matrix {
#define FOR(i, l, r) for (int (i) = (l); (i) <= (r); (i)++)
    int row, col;
    complex<Type> w[maxn][maxn];
    Matrix (int row, int col):row(row), col(col) {
        FOR(i, 1, row) FOR(j, 1, col) w[i][j] = 0; // initalize
    }
    Matrix (const Matrix &other) {
        row = other.row, col = other.col;
        FOR(i, 1, row) FOR(j, 1, col) w[i][j] = other.w[i][j];
    }
    friend Matrix operator * (const Matrix &a, const Matrix &b) {
        assert(a.col == b.row);
        Matrix c(a.row, b.col);
        FOR(i, 1, a.row) FOR(j, 1, b.col) {
            c.w[i][j] = 0;
            FOR(k, 1, a.col) c.w[i][j] += a.w[i][k] * b.w[k][j];
        }
        return c;
    }
    friend bool operator < (const Matrix &a, const Matrix &b) {
        assert(a.row == b.row);
        assert(a.col == b.col);
        FOR(i, 1, a.row) FOR(j, 1, a.col) {
            if (a.w[i][j] != b.w[i][j]) {
                if (a.w[i][j].real() == b.w[i][j].real()) 
                    return a.w[i][j].imag() < b.w[i][j].imag();
                return a.w[i][j].real() < b.w[i][j].real();
            }
        }
        return false;
    }
    friend bool operator == (const Matrix &a, const Matrix &b) {
        if (a.row != b.row || a.col != b.col) return false;
        FOR(i, 1, a.row) FOR(j, 1, a.col) {
            if (a.w[i][j] != b.w[i][j]) return false;
        }
        return true;
    }
    void Log() {
        FOR(i, 1, row) {
            FOR(j, 1, col) {
                printf("%d + %di ", w[i][j].real(), w[i][j].imag());
            } printf("\n");
        }
    }
#undef FOR
};

Matrix<int> I(2, 2), A(2, 2), B(2, 2);
void initI() {
    I.w[1][1] = 1;
    I.w[2][2] = 1;
}
void initA() {
    A.w[1][2] = 1;
    A.w[2][1] = -1;
}
void initB() {
    B.w[1][2] = complex<int>(0, 1);
    B.w[2][1] = complex<int>(0, 1);
}

void search() {
    initI(); initA(); initB();
    set<Matrix<int> > st; st.clear();
    queue<Matrix<int> > q;
    q.push(I);
    st.insert(I);
    while(!q.empty()) {
        Matrix<int> tt = q.front(); q.pop();
        tt.Log(); puts("");
        Matrix<int> ta = tt * A;
        Matrix<int> tb = tt * B;
        Matrix<int> at = A * tt;
        Matrix<int> bt = B * tt;
        if (!st.count(ta)) {
            st.insert(ta);
            q.push(ta);
        }
        if (!st.count(tb)) {
            st.insert(tb);
            q.push(tb);
        }
        if (!st.count(at)) {
            st.insert(at);
            q.push(at);
        }
        if (!st.count(bt)) {
            st.insert(bt);
            q.push(bt);
        }
    }
    cout << st.size() << endl;
}

int main() {
    search();
    return 0;
}
\end{verbatim}

\subsection{Exercise 4 (P33)}
\label{sec:org0f200e3}
就感觉做这种题非常浪费时间
\begin{verbatim}
#include <iostream>
#include <cstring>
#include <cstdio>
#include <complex>
#include <map>
#include <set>
#include <queue>
using namespace std;

const int maxn = 10;

// load the Matrix Class
template <class Type>
struct Matrix {
#define FOR(i, l, r) for (int (i) = (l); (i) <= (r); (i)++)
    int row, col;
    complex<Type> w[maxn][maxn];
    Matrix (int row, int col):row(row), col(col) {
        FOR(i, 1, row) FOR(j, 1, col) w[i][j] = 0; // initalize
    }
    Matrix (const Matrix &other) {
        row = other.row, col = other.col;
        FOR(i, 1, row) FOR(j, 1, col) w[i][j] = other.w[i][j];
    }
    friend Matrix operator * (const Matrix &a, const Matrix &b) {
        assert(a.col == b.row);
        Matrix c(a.row, b.col);
        FOR(i, 1, a.row) FOR(j, 1, b.col) {
            c.w[i][j] = 0;
            FOR(k, 1, a.col) c.w[i][j] += a.w[i][k] * b.w[k][j];
        }
        return c;
    }
    friend bool operator < (const Matrix &a, const Matrix &b) {
        assert(a.row == b.row);
        assert(a.col == b.col);
        FOR(i, 1, a.row) FOR(j, 1, a.col) {
            if (a.w[i][j] != b.w[i][j]) {
                if (a.w[i][j].real() == b.w[i][j].real()) 
                    return a.w[i][j].imag() < b.w[i][j].imag();
                return a.w[i][j].real() < b.w[i][j].real();
            }
        }
        return false;
    }
    friend bool operator == (const Matrix &a, const Matrix &b) {
        if (a.row != b.row || a.col != b.col) return false;
        FOR(i, 1, a.row) FOR(j, 1, a.col) {
            if (a.w[i][j] != b.w[i][j]) return false;
        }
        return true;
    }
    void Log() {
        FOR(i, 1, row) {
            FOR(j, 1, col) {
                printf("%d + %di ", w[i][j].real(), w[i][j].imag());
            } printf("\n");
        }
    }
#undef FOR
};

Matrix<int> I(2, 2), A(2, 2), B(2, 2);
void initI() {
    I.w[1][1] = 1;
    I.w[2][2] = 1;
}
void initA() {
    A.w[1][2] = 1;
    A.w[2][1] = -1;
}
void initB() {
    B.w[1][2] = complex<int>(1, 0);
    B.w[2][1] = complex<int>(1, 0);
}

void search() {
    initI(); initA(); initB();
    set<Matrix<int> > st; st.clear();
    queue<Matrix<int> > q;
    q.push(I);
    st.insert(I);
    while(!q.empty()) {
        Matrix<int> tt = q.front(); q.pop();
        tt.Log(); puts("");
        Matrix<int> ta = tt * A;
        Matrix<int> tb = tt * B;
        Matrix<int> at = A * tt;
        Matrix<int> bt = B * tt;
        if (!st.count(ta)) {
            st.insert(ta);
            q.push(ta);
        }
        if (!st.count(tb)) {
            st.insert(tb);
            q.push(tb);
        }
        if (!st.count(at)) {
            st.insert(at);
            q.push(at);
        }
        if (!st.count(bt)) {
            st.insert(bt);
            q.push(bt);
        }
    }
    cout << st.size() << endl;
}

int main() {
    search();
    return 0;
}
\end{verbatim}
\subsection{Exercise 5 (P33)}
\label{sec:org93b6be4}
Reflexivity: \(aa^{-1}\in S \Leftrightarrow e\in S\).
Symmetry: \(ab^{-1}\in S \to ba^{-1}\in S \Leftrightarrow\) every elements in \(S\) has an inverse.
Transitivity: \(ab^{-1} \in S \land bc^{-1}\in S \to ac^{-1}\in S \Leftrightarrow\) operation on \(S\) is close.
\subsection{Exercise 6 (P34)}
\label{sec:org116c354}
name the subset as \(S\).
\(\forall a\in S\), let \(S_a = \{a^n | n\in Z\} \subset S\).
then we could find that \((\exists n, m \in Z) a^n = a^m\) (because of finite).
we could use this infomation to construct \(e\) and the inverse.
\subsection{Exercise 7 (P34)}
\label{sec:orgdf065c8}
let \(S = \{kn | k \in Z\}\).
construct a homomorphism \(f: S \to Z\) by letting \(f(x) = x/n\).
\(f(xnyn) = \frac{xn}{n}\frac{yn}{n} = f(xn)f(yn)\).
\begin{enumerate}
\item \(f\) is a surjection.
\item \(f(x) = f(y) \Rightarrow x/n = y/n \Rightarrow x = y\), so \(f\) is a injectoin.
\end{enumerate}
so \(f\) is a bijection.
so \(S\cong Z\).
\end{document}