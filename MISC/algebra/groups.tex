% Created 2019-03-01 Fri 10:18
% Intended LaTeX compiler: pdflatex
\documentclass[11pt]{article}
\usepackage[utf8]{inputenc}
\usepackage[T1]{fontenc}
\usepackage{graphicx}
\usepackage{grffile}
\usepackage{longtable}
\usepackage{wrapfig}
\usepackage{rotating}
\usepackage[normalem]{ulem}
\usepackage{amsmath}
\usepackage{textcomp}
\usepackage{amssymb}
\usepackage{capt-of}
\usepackage{hyperref}
\author{陈小羽}
\date{\today}
\title{Algebra/Group}
\hypersetup{
 pdfauthor={陈小羽},
 pdftitle={Algebra/Group},
 pdfkeywords={},
 pdfsubject={},
 pdfcreator={Emacs 26.1 (Org mode 9.1.9)}, 
 pdflang={English}}
\begin{document}

\maketitle
\tableofcontents

\section{Summary}
\label{sec:orgac572bd}
When I am reading this book, I find it is so easy for me to forget the concepts
that I was learn before. Because the amount of them are incredible.
So, I record some important concepts and their definition here.

\section{Logical Prerequisites}
\label{sec:org8970e84}
\subsection{map}
\label{sec:org0d10647}
If \(f: A\mapsto B\) is a mapping of one set into another, we write
\[x \mapsto f(x)\] to denote the effect of \(f\) on an element \(x\) of \(A\).
\subsection{injective}
\label{sec:orgb7d02a2}
We say that \(f\) is \textbf{injective} if \(x\not=y\) implies \(f(x)\not=f(y)\).
\subsection{surjective}
\label{sec:orgee86f6c}
We say \(f\) is \textbf{surjective} if given \(b\in B\) there exists \(a\in A\) such that \(f(a) = b\).
\subsection{bijective}
\label{sec:orgf578fa6}
We say \(f\) is \textbf{bijective} if it is both surjective and injective.
\subsection{restriction of map}
\label{sec:org72c85fb}
We say the restriction of \(f\) to \(A'\) is a map of \(A'\) into \(B\) denoted by \(f|A'\).
\subsection{composite map}
\label{sec:orge085e03}
If \(f: A\mapsto B\) and \(g: B\mapsto C\) are maps, then we have a composite map \(g\circ f\)
such that \((g\circ f)(x) = g(f(x))\) for all \(x\in A\).
\subsection{image of \(f\)}
\label{sec:org735ef5c}
Let \(f:A\mapsto B\) be a map, and \(B'\) a subset of \(B\). By \(f^{-1}(B')\) we mean the subset
of \(A\) consisting of all \(x\in A\) such that \(f(x)\in B'\). We call it the \textbf{inverse image} of 
\(B'\). We call \(f(A)\) the image of \(f\).
\subsection{commutative}
\label{sec:orgd95ffef}
\href{./commutative.png}{说明 (图片)}
\section{Monoids}
\label{sec:org01803fb}
\subsection{law of composition}
\label{sec:org5745603}
A mapping \(S\times S \to S\) is sometimes called a \textbf{law of composition} (of \(S\) into itself).
\subsection{associative}
\label{sec:orgf827124}
\$(xy)z = x(yz)
\subsection{unit element}
\label{sec:orga2597c1}
\(e\in S\), and \(ex = x = xe\) for all \(x\in S\). Then \(e\) is a unit element.
\subsection{monoid}
\label{sec:org40edaf2}
a \textbf{monoid} is a set \(G\), with a law of composition which is associative, and having a unit element.
\subsection{commutativity}
\label{sec:org6f579e7}
\(xy = yx\).
\subsection{commutative (or abelian)}
\label{sec:org0c3708e}
If the law of compositive of \(G\) is commutative, we also say that \(G\) is \textbf{commutative (or abelian)}.
\subsection{almost all}
\label{sec:org058a1b8}
All but a finite number.
\subsection{submonoid}
\label{sec:org8968f60}
By a \textbf{submonoid} of \(G\), we shall mean a subset \(H\) of \(G\) containing the unit element \(e\),
and such that, if \(x, y \in H\) then \(xy \in H\) (we say that \(H\) is \textbf{closed} under the law of 
composition). It is then clear that \(H\) is itself a monoid, under the law of composition indeuced
by that of \(G\).
\section{Group}
\label{sec:orgc1cc2b8}
\subsection{group}
\label{sec:org4952431}
A \textbf{group} \(G\) is a monoid, such that for every element \(x\in G\), there exists an element \(y\in G\)
such that \(xy = yx = e\).
\subsection{inverse}
\label{sec:orga55d864}
An \textbf{inverse} for \(x\) is an element \(y\in G\) such that \(xy = yx = e\).
Sometimes \(y\) is denoted by \(x^{-1}\).
\subsection{direct product}
\label{sec:orgcf14a1d}
\(G_1 \times G_2 = \{(x, y) | \forall x\in G_1, \forall y\in G_2\}\)
\subsection{generators}
\label{sec:orga5d9b6d}
Let \(G\) be a group and \(S\) a subset of \(G\). We shall say that \(S\) \textbf{generates} \(G\), or \(S\) is 
a set of \(generators\) for \(G\), if every element of \(G\) can be expressed as a product of
element in \(S\) or a inverse of element in \(S\).
\subsection{homomorphism}
\label{sec:org6e44ee4}
A (monoid/group)-\textbf{homomorphism} of \(G\) into \(G'\) is a mapping \(f: G\mapsto G'\) such that \(f(xy) = f(x)f(y)\).
\subsection{isomorphism}
\label{sec:orgef9434a}
A homomorphism \(f:G\mapsto G'\) is called an \textbf{isomorphism} if there exists a homomorphism
\(g: G'\mapsto G\) such that \(f\circ g\) and \(g\circ f\) are the identity mappings.
This is sometimes denoted by \(G' \approx G\).
\subsection{automorphism}
\label{sec:org9a980ed}
A isomorphism from \(G\) to \(G\).
\subsection{endomorphism}
\label{sec:org85808f8}
A homomorphism from \(G\) to \(G\).
\subsection{power map}
\label{sec:orgd449ad8}
The map \(x\mapsto x^n\) is called the n-th \textbf{power map}.
\subsection{kernel}
\label{sec:orgdbad547}
Let \(f: G\mapsto G'\) be a group-homomorphism.
We define the \textbf{kernel} of \(f\) to be the subset of \(G\) consisting of all \(x\) such that \(f(x) = e'\).
\subsection{embedding}
\label{sec:org801c961}
A homomorphism \(f: G\mapsto G'\) which establish an isomorphism between \(G\) and its image in \(G'\)
will also be called an embedding.
\subsection{left (right) coset}
\label{sec:orgfc1836a}
A \textbf{left coset} of \(H\) in \(G\) is a subset of \(G\) of type \(aH\), for some element \(a\in G\).
\subsection{coset representative}
\label{sec:orga95b1a6}
An element of \(aH\) is called a \textbf{coset representative} of \(aH\).
\subsection{index}
\label{sec:org94de3cd}
The number of (left) coset of \(H\) in \(G\) is called the (left) \textbf{index} of \(H\) in \(G\),
and is denoted by \((G:H)\).
The index of the trivial subgroup \(\{e\}\) is called the \textbf{order} of \(G\) (\((G:1)\)).
\section{Normal Subgroups}
\label{sec:orgd8180c1}
\subsection{normal subgroup}
\label{sec:orgecf6700}
Let \(G\) be a group and \(H\) is a subgroup of \(G\).
If \(H\) satisfies \(H \subset xHx^{-1}\) for all the \(x\in G\).
Then \(H\) is a normal subgroup of \(G\).
\subsection{canonical map}
\label{sec:orga3cbc1f}
If \(H\) is a normal subgroup of \(G\).
Then, the constructed map \(f: G\to G/H\) is called \textbf{canonical map}.
And \(G/H\) is called the \textbf{factor group} of \(G\) by \(H\).
canonical map is often denoted by \(\varphi\).
\subsection{normalizer}
\label{sec:org2271f26}
Let \(S\) be a subset of \(G\) and let \(N = N_s\) be the set of 
all elements \(x\in G\) such that \(xSx^{-1} = S\). Then N
is called the \textbf{normalizer} of \(S\).
\subsection{centralizer}
\label{sec:org1ea247c}
Let \(S\) be a subset of \(G\).
\textbf{Centralizer} \(C\) is the set of all the elements in \(G\) such that
\(\forall x\in S, \forall y\in C, yxy^{-1} = x\).
The centralizer of \(G\) itself is called \textbf{center} of \(G\).
\subsection{tower}
\label{sec:org4e81365}
Let \(G\) be a group. A sequence of subgroups
\(G = G_0 \supset G_1 \supset G_2 \supset \cdots \supset G_m\)
is called a \textbf{tower} of subgroups.
\subsection{normal tower}
\label{sec:orgb71d5fb}
a tower is said to be normal if each \(G_{i+1}\) is normal in \(G_i\).
\subsection{abelian tower}
\label{sec:orgf85b401}
a tower is said to be abelian if it is normal and if each factor 
group \(G_i/G_{i+1}\) is abelian.
\subsection{cyclic tower}
\label{sec:orgb565ce9}
a tower is said to be cyclic if it is normal and if each factor 
group \(G_i/G_{i+1}\) is cyclic.
\subsection{refinement of a tower}
\label{sec:orgf02ee04}
a refinement of a tower is a tower which can be obtained by inserting
a finite number of subgroups in the given tower.
\subsection{solvable group}
\label{sec:org9db09f4}
a group is said to be solvable if it has abelian tower, whose
last element is the trivial subgroup(\{e\}).
\subsection{commutator}
\label{sec:orgdf96784}
a \textbf{commutator} in \(G\) is a group element os the form \(xyx^{-1}y^{-1}\).
with \(x, y\in G\).
\subsection{commutator subgroup}
\label{sec:org52bc16d}
a commutator subgroup \(C^C\) of \(G\) is a subgroups generates by the 
commutators.
\subsection{simple group}
\label{sec:orga1bde89}
a group is said to be \textbf{simple} if it is non-trivial.

\section{Cyclic Groups}
\label{sec:org5322a3f}
\subsection{cycilc group}
\label{sec:org895eb81}
Let \(G\) be a group. \(G\) is cyclic if there exists an element
\(a\) of \(G\) such that every element \(x\) of \(G\) can be written
in the form \(a^n\) for some \(n\in Z\).
\(a\) is called the generator of \(G\).
\subsection{exponent of an element}
\label{sec:orgc6ab9b5}
Let \(G\) be a group.
For \(a\in G\), if \(a^m = e\), then we say \(m\) is a exponent of \(a\).
If all the elements in \(G\) has the exponent \(m\), then we say that \(G\) as exponent \(m\).
\subsection{period of an element}
\label{sec:org12a29e0}
\(d\) is the smallest positive exponent of \(a\).
Then \(d\) is called the period of \(a\).
\section{Operations Of A Group On A Set}
\label{sec:orgdc3f612}
\subsection{operation/action of group \(G\) on set \(S\).}
\label{sec:org412bd80}
An operation of \(G\) on \(S\) is a homomorphism:
\[\pi: G\to Perm(S) \]. 
We then call \(S\) a \$G\$-set
\subsection{conjugation}
\label{sec:org3338240}
\(G\to Aut(G)\) is called a conjugation.
its kernel is called the \textbf{center} of \(G\).
\subsection{conjugate}
\label{sec:orgde221a4}
\(A, B\) are two subsets of \(G\). we say that they are
\textbf{conjugate} if there exists \(x\in G\) such that \(B = xAx^{-1}\).
\subsection{morphism of \$G\$-set / \$G\$-map}
\label{sec:orgf496e80}
Let \(S, S'\) be two \$G\$-sets, and \(f:S\to S'\) a map.
We say that \(f\) is a \textbf{morphism} of \$G\$-sets, or a \$G\$-map
if \(f(xs) = xf(s)\) for \(x\in G\) and \(s\in S\).
\subsection{iostropy}
\label{sec:orgc58e481}
The set of elements \(x\in G\) such that \(xs = s\) is obviously
a subgroup of \(G\), called the \(isotropy\).
(denoted by \(G_s\)).
\subsection{faithful / fixed point (P28)}
\label{sec:orga714738}
\subsection{orbit}
\label{sec:org4abdb87}
Let \(G\) operate on a set \(S\). Let \(s\in S\). The subset
of \(S\) consisting of all elements \(xs\) (with \(x\in G\)) is
denoted by \(G_S\), and is called the \textbf{orbit} of \(s\) under \(G\).
\subsection{transitive}
\label{sec:orgd37bdce}
an operation of \(G\) on \(S\) is said to be \textbf{transitive}
if there is only one orbit.
\section{Sylow Subgroups}
\label{sec:org4fd7db4}
\subsection{p-group}
\label{sec:org74896a6}
a finite group whose order is a power of \(p\).
\subsection{p-subgroup}
\label{sec:org1f1c931}
Let \(G\) be a finite group and \(H\) is a subgroup of it.
we call \(h\) a p-subgroup os \(G\) if \(H\) is a p-group.
\subsection{p-Sylow subgroup}
\label{sec:org121c4cd}
We call \(H\) a p-Sylow subgroup if the order of \(H\) is \(p^n\) and if \(p^n\) is the hightest
power of \(p\) deviding the order of \(G\).
\section{Direct Sums and Free Abelian Groups}
\label{sec:orgfdf007b}
\subsection{direct sum}
\label{sec:org161cffe}
Let \(\{a_i\}_{i\in i}\) be a family of abelian groups. we define their \textbf{direct sum}
\[A = \bigoplus_{i\in I} A_i\] to be the subset of the direct product \(\prod a_i\) consisting of all 
families \((x_i)_{i\in i}\) with \(x_i\in a_i\) such that \(x_i = 0\) for all but a 
finite nunber of indices \(i\).
\subsection{basis}
\label{sec:org2880a1e}
Let \(A\) be an abelian group. Let \(\{e_i\}(i\in I)\) be a family of elements of
\(A\). We say that this family is a \textbf{basis} for \(A\) if the family
and if every element of \(A\) has a unique expression as a linear combination
\[ x = \sum x_ie_i \] with \(x_i\in Z\) and almost all \(x_i = 0\).
\subsection{free abelian group}
\label{sec:org02c603a}
An abelian group is said to be \textbf{free} if it has a basis.
\subsection{free abelian group generated by \(S\)}
\label{sec:org8e5f4ff}
We shall denote \(Z\langle S \rangle\) also by \(F_{ab}(S)\), 
and call \(F_{ab}(S)\) the \textbf{free abelian group generated by \(S\)}.
we call elements of \(S\) its \textbf{free generators}.
\end{document}