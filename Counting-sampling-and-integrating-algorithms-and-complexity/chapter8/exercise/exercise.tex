\documentclass{article}
\usepackage{amsmath, amssymb, amsthm}
\usepackage{tikz}
\usepackage{hyperref}
\hypersetup{
    colorlinks=true,
    linkcolor=blue,
    filecolor=magenta,      
    urlcolor=cyan,
}

\newtheorem{claim}{Claim}

\title{Exercise of Chapter 8}
\author{Xiaoyu Chen}

\begin{document}
\maketitle
\section{Exercise 8.2.}
Prove Jensen's inequality.

\flushleft Recall The Jesen's inequality as follow: \\
Let $K\subset \mathbb{R}^k$ be a compact convex set, $X$ a r.v. taking values in $K$,
and $g: K\to \mathbb{R}$ a convex real-valued function. Then $g(\mathbb{E} X) \leq \mathbb{E} g(X)$
\subsection{solution}
Following the hint, we first construct the graph
\[G(g) = \{(x, y) : \mbox{$x\in k$ and $y\geq g(x)$}\} \subset \mathbb{R}^{k+1}\]
\begin{claim}
  $G(g)$ is a convex set
\end{claim}
\begin{proof}
For any two vertex $x = (a, b), y = (c, d) \in G(g)$, and some $\lambda \in [0, 1]$, we have
\begin{align*}
  \lambda x + (1-\lambda y) &= \lambda(a, b) + (1-\lambda)(c,d) \\
  &= (\lambda a + (1-\lambda)c, \lambda b + (1-\lambda) d)
\end{align*}
Since $a\in K$ and $c\in K$, and $K$ is a convex set, we have $\lambda a + (1-\lambda) c \in K$.
Moreover, we have:
\begin{align*}
  \lambda b + (1-\lambda) d &\geq \lambda g(a) + (1-\lambda) g(c) \\
  &\geq g(\lambda a + (1-\lambda) c),\hspace{0.5cm} \mbox{since $g$ is convex}
\end{align*}
Combining these together we coud find that $\lambda x + (1-\lambda) y \in G(g)$.
Then, by the definition of convex set, $G(g)$ is a convex set.
\end{proof}
\begin{claim}
  we have a supporting hyperplane for $G(g)$ at $(\mathbb{E} x, g(\mathbb{E} x))$
\end{claim}
\begin{proof}
  Since $G(g)$ is a convex set, and $(\mathbb{E} x, g(\mathbb{E} x))$ is on the boundary of $G(g)$, there is a supporting hyperplane for $G(g)$.
  (this is followed by the
  \href{https://en.wikipedia.org/wiki/Supporting_hyperplane}
       {supporting hyperplane theorem})
\end{proof}
For convenience, we denote this hyperplane by
\[P: a^T (x - b) = 0, \hspace{0.5cm} \mbox{where $b = (\mathbb{E} X, g(\mathbb{E} X))$}\]
For all $v = (x, y)\in P$ and $x\in K$ we have $y \leq g(x)$, since $P$ is the supporting hyperplane of $G(g)$.
This is the key observation of the whole solution.
\begin{claim}
  For any vertex $a$, let $a_{n-1} \in \mathbb{R}^{k-1}$ be the sub-vertex of $a$, and $a_n$ be the last dimension of $a$. Then for a hyperplane $P: a^T(X - b) = 0$ and a point $x\in P$, we have:
  \[x_n = \frac{1}{a_n}(a_{n-1}^T (b_{n-1} - x_{n-1})) + b_n\]
\end{claim}
\begin{proof}
  \begin{align*}
    a^T(x - b) &= 0 \\
    a^Tx &= a^Tb \\
    a_{n-1}^Tx_{n-1} + a_nx_n &= a_{n-1}^Tb_{n-1} + a_nb_n \\
    a_nx_n &= a_{n-1}^Tb_{n-1} - a_{n-1}^Tx_{n-1} + a_nb_n \\
    x_n &= \frac{1}{a_n}(a_{n-1}^T(b_{n-1} - x_{n-1})) + b_n \qedhere
  \end{align*}
\end{proof}
So, If the hyperplane $P$ is the one at $(\mathbb{E} X, g(\mathbb{E} X))$, we have
\[f(x) = \frac{1}{a_n}(a_{n-1}^T(\mathbb{E} X - x)) + g(\mathbb{E} X), \hspace{0.5cm} \mbox{where $(x, f(x)) \in P$ and $x\in K$}\]
\begin{claim}
$\mathbb{E}f(Y) = g(\mathbb{E}X)$
\end{claim}
\begin{proof}
  \begin{align*}
    \mathbb{E}f(X) &= \frac{\int_K f(x) \mathrm{d}x}{\int_K \mathrm{d}x} \\
    &= \frac{
    \int_K [\frac{1}{a_n}(a_{n-1}^T(\mathbb{E}X - x)) + g(\mathbb{E} X)] \mathrm{d}x
     } {\mathrm{Vol}(K)} \\
    &= \frac{
\frac{a_{n-1}^T}{a_n}\int_K(\mathbb{E}X - x)\mathrm{d}x + g(\mathbb{E} X)\mathrm{Vol}(K)}{\mathrm{Vol}(K)} \\
    &= g(\mathbb{E} X) \qedhere
  \end{align*}
\end{proof}
So we have $g(\mathbb{E} X) = \mathbb{E}f(X) \leq \mathbb{E}g(X)$, since $f(x) \leq g(x)$ on every $x\in K$.

\section{Exercise 8.3}
Verify (8.1) and (8.2). (One of these identities is actually trivial)
\subsection{verify (8.2)}
Lets start from the trivial one.
\begin{align*}
  \mathcal{E}_P(\varphi, \varphi) = \frac{1}{2} \sum_{x, y\in\Omega}\pi(x)P(x, y)(\varphi(x) - \varphi(y))^2
\end{align*}
Let
\begin{align*}
  G[\Omega] &= \frac{1}{2} \sum_{x, y\in\Omega} \pi(x)P(x, y)(\varphi(x) - \varphi(y))^2 \\
  G[\Omega_1, \Omega_2] &= \sum_{x\in\Omega_1,y\in\Omega_2} \pi(x)P(x,y)(\varphi(x) - \varphi(y))^2
\end{align*}
Then we have,
\begin{align*}
  \pi(\Omega_0)&\mathcal{E}_{P_0} (\varphi, \varphi) + \pi(\Omega_1)\mathcal{E}_{P_1}(\varphi, \varphi) + C \\
  &= G[\Omega_0] + G[\Omega_1] + G[\Omega_0, \Omega_1]
\end{align*}
Since in this definition, we have assumed time reversibility of $(\Omega, p)$.
So, $G[\Omega_0]$ enumerates all unordered pairs $\{x, y\}$ where $x, y\in\Omega_0$.
Similarly, $G[\Omega_1]$ enumerates all unordered pairs $\{x, y\}$ where $x, y\in\Omega_1$.
Moreover, $G[\Omega_0, \Omega_1]$ enumerate all the unordered pairs $\{x, y\}$ where $x\in\Omega_0$ and $y\in\Omega_1$.
And hence
\[\pi(\Omega_0)\mathcal{E}_{P_0}(\varphi, \varphi) + \pi(\Omega_1)\mathcal{E}_{P_1}(\varphi, \varphi) + C\]
enumerates all the unordered pairs $\{x, y\}$ where $x, y\in\Omega$, which is the same as $\mathcal{E}_{P}(\varphi, \varphi)$.

\newcommand{\E}{\mathbb{E}}
\newcommand{\Var}{\mathrm{Var}}
\subsection{verify (8.1)}
First, for convenience, lets define some notations.
\begin{align*}
  [\E\varphi]_{\Omega}^\pi &:= \sum_{x\in\Omega} \pi(x)\varphi(x) \\
  \varphi &:= \varphi(x), \hspace{0.5cm} \mbox{this is an abbreviate}
\end{align*}
First, we know that
\begin{align*}
  \Var_\pi \varphi &= \E\varphi^2 - (\E\varphi)^2 \\
  \Var_{\pi_0}\varphi &= [\E\varphi^2]_{\Omega_0}^{\pi_0} - ( [\E\varphi]_{\Omega_0}^{\pi_0})^2 \\
  \Var_{\pi_1}\varphi &= [\E\varphi^2]_{\Omega_1}^{\pi_1} - ( [\E\varphi]_{\Omega_1}^{\pi_1})^2 
\end{align*}
And we could rewrite $(\E\varphi)^2$ as
\begin{align*}
  (\E\varphi)^2 &= (\E\varphi)(\E\varphi) \\
  &= [\E\varphi]_{\Omega_a}^\pi\E\varphi + [\E\varphi]_{\Omega_b}^\pi \E\varphi \\
&= ([\E\varphi]_{\Omega_a}^\pi)^2 + 2[\E\varphi]_{\Omega_a}^\pi[\E\varphi]_{\Omega_b}^\pi + ([\E\varphi]_{\Omega_b}^\pi)^2
\end{align*}
Then
\begin{align*}
  \Var_\pi\varphi &- \pi(\Omega_0)\Var_{\pi_0}\varphi - \pi(\Omega_1)\Var_{\pi_1} \\
  &= \pi(\Omega_0)([\E\varphi]_{\Omega_0}^{\pi_0})^2 + \pi(\Omega_1)([\E\varphi]_{\Omega_1}^{\pi_1})^2 - (\E\varphi)^2 \\
  &= \frac{1 - \pi(\Omega_0)}{\pi(\Omega_0)}([\E\varphi]_{\Omega_0}^{\pi})^2 + \frac{1-\pi(\Omega_1)}{\pi(\Omega_1)}([\E\varphi]_{\Omega_1}^{\pi})^2 - 2[\E\varphi]_{\Omega_0}^\pi[\E\varphi]_{\Omega_1}^\pi \\
  &= \frac{\pi(\Omega_1)}{\pi(\Omega_0)}([\E\varphi]_{\Omega_0}^{\pi})^2 + \frac{\pi(\Omega_0)}{\pi(\Omega_1)}([\E\varphi]_{\Omega_1}^{\pi})^2 - 2[\E\varphi]_{\Omega_0}^\pi[\E\varphi]_{\Omega_1}^\pi , \hspace{0.5cm} \mbox{since $\pi(\Omega_0) + \pi(\Omega_1) = 1$}\\
  &= \pi(\Omega_0)\pi(\Omega_0)([\E\varphi]_{\Omega_0}^{\pi_0})^2 + \pi(\Omega_0)\pi(\Omega_1)([\E\varphi]_{\Omega_1}^{\pi_1})^2 - 2\pi(\Omega_0)\pi(\Omega_1)[\E\varphi]_{\Omega_0}^{\pi_0}[\E\varphi]_{\Omega_1}^{\pi_1} \\
  &= \pi(\Omega_0)\pi(\Omega_0)\left\{([\E\varphi]_{\Omega_0}^{\pi_0})^2 + ([\E\varphi]_{\Omega_1}^{\pi_1})^2 - 2[\E\varphi]_{\Omega_0}^{\pi_0}[\E\varphi]_{\Omega_1}^{\pi_1} \right\} \\
  &= \pi(\Omega_0)\pi(\Omega_0)(\E_{\pi_a}\varphi - \E_{\pi_b}\varphi)^2 \\
  &= \Var_\pi \overline{\varphi}
\end{align*}
\end{document}

%%% Local Variables:
%%% mode: latex
%%% TeX-master: t
%%% End:
