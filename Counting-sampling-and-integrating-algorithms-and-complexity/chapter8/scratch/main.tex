\documentclass{article}
\usepackage{amsmath, amssymb, amsthm}

\newtheorem{define}{Define}
\newtheorem{fact}{Fact}
\newtheorem{theorem}{Theorem}

\begin{document}

\section{basic}
\begin{fact}
  If we have a matroid $\mathcal{B}(M)$, and an element $e\in M$.
  Then any basis in $\mathcal{B}_e$ has at least one neighbor in $\mathcal{B}_{\overline{e}}$ and vice versa.
\end{fact}
\begin{proof}
  Actually, this is a corollary of \underline{strong base exchange theorem}, which is quite difficult to prove.
\end{proof}

\section{deleting}
\begin{fact}
  If $C$ is a cycle in $\mathcal{B}(M\setminus e)$ and $e\not\in C$,
  then $C$ is a cycle in $\mathcal{B}(M)$.
\end{fact}
\begin{proof}
  Say we have a cycle $C$ of $\mathcal{B}(M\setminus e)$.
  More specificly,
  \begin{align*}
    B_0 &\in \mathcal{B}(M\setminus e) \\
    B_1 &\in \mathcal{B}(M) \\
    B_1 &= B_0 \cup \{e\} \setminus \{g\}, \hspace{0.5cm} g\not= e
  \end{align*}
  By the definition of $C$, we know that $C$ could not extend to $B_0$,
  which means there is an element $f\in C$ and $f\not\in B_0$.
  Then, by the definition of $B_1$, we know that $f\not\in B_1$, so $C$ could not extend to $B_1$.
\end{proof}

\begin{fact}
  If $D$ is a cut in $\mathcal{B}(M\setminus e)$,
  then $D$ is a cut in $\mathcal{B}(M)$ or $D\cup \{e\}$ is a cut in $\mathcal{B}(M)$.
\end{fact}
\begin{proof}
  Now, we have
  \begin{align*}
    B_0 &\in \mathcal{B}(M\setminus e) \\
    B_1 &\in \mathcal{B}(M) \\
    B_1 &= B_0 \cup \{e\} \setminus \{g\}, \hspace{0.5cm} g\not= e
  \end{align*}
  By the definition of $D$ we know that $B_0$ is not contained in the complement of $D$.
  So there is an element $f\in B_0$ which is not contained by the complement of $D$.
  So if $f\not= e$, then $B_1$ is not contained by the complement of $D$.
  Or if $f=e$, then $B_1$ is not contained by the complement of $D\cup \{e\}$.
\end{proof}

\begin{theorem}
  $\mathcal{B}(M\setminus e)$ is an orientable matroid.
\end{theorem}

\begin{proof}
If $\mathcal{B}(M)$ is an orientable matroid,
then we have
\begin{align*}
  \gamma(C, g) &\not= 0 \mbox{ iff } g \in C \\
  \delta(D, g) &\not= 0 \mbox{ iff } g \in D \\
  \sum_{g\in E} \gamma(C, g)\delta(D, g) &= 0, \hspace{0.5cm} \mbox{$E$ is the groud set}
\end{align*}
Now, consider $\gamma': \mathcal{C}\times E\setminus\{e\}\to \{-1, 0, 1\}$, and $\delta': \mathcal{D}\times E\setminus\{e\}\to \{-1, 0, 1\}$ for $\mathcal{B}(M\setminus e)$.
Set their values by the following rules:
\begin{align*}
  \gamma'(C, g) &= \gamma(C, g) \\
  \delta'(D, g) &= \delta(D, g), \mbox{ if $D$ is a cut in $\mathcal{B}(M)$} \\
  \mbox{or } \delta'(D, g) &= \delta(D\cup\{e\}, g), \mbox{ if $D\cup\{e\}$ is a cut in $\mathcal{B}(M)$}
\end{align*}
According to Fact 1 and Fact 2, we could claim that $\gamma'$ and $\delta'$ are well defined.
Then we have:
\begin{align*}
  \sum_{g\in E\setminus\{e\}} \gamma'(C, g)\delta'(D, g) &= \sum_{g\in E\setminus\{e\}} \gamma'(C, g)\delta'(D, g) + 0 \\
                                                         &= \sum_{g\in E\setminus\{e\}} \gamma'(C, g)\delta'(D, g) + \gamma'(C, e)\delta'(D, g), \hspace{0.5cm} \mbox{since $e\not\in C$} \\
                                                         &= \left\{
                                                           \begin{array}{l}
                                                             \sum_{g\in E} \gamma(C, g)\delta(D, g), \mbox{ if $D$ is a cut in $\mathcal{B}(M)$} \\
                                                             \sum_{g\in E} \gamma(C, g)\delta(D\cup\{e\}, g), \mbox{ if $D\cup\{e\}$ is a cut in $\mathcal{B}(M)$}
                                                           \end{array}
  \right. \\
  &= 0
\end{align*}
So, we know that $\mathcal{B}(M\setminus\{e\})$ is an orientable matroid.
\end{proof}
\section{contracting}
\begin{fact}
  If $D$ is a cut in $\mathcal{B}(M/e)$ and $e\not\in D$,
  then $D$ is a cut in $\mathcal{B}(M)$.
\end{fact}
\begin{proof}
  Say we have a cut $D$ of $\mathcal{B}(M/e)$.
  More specificly,
  \begin{align*}
    B_0 &\in \mathcal{B}(M/e) \\
    B_1 &\in \mathcal{B} \\
    B_1 &= B_0 \cup \{g\}, \hspace{0.5cm} g\not= e
  \end{align*}
  By the definition of $D$, we know that $D$'s complement does not contain $B_0$.
  So, there is an element $f\in B_0$, which does not contained by $D$'s complement.
  Note that this element is also in $B_1$, so $D$'s complement also does not contain $B_1$.
\end{proof}

\begin{fact}
  If $C$ is a cycle in $\mathcal{B}(M/e)$ and $e\not\in C$, then $C$ is a cycle in $\mathcal{B}(M)$ or $C\cup \{e\}$ is a cycle in $\mathcal{B}(M)$.
\end{fact}
\begin{proof}
  Now we have
  \begin{align*}
    B_0 &\in \mathcal{B}(M/e) \\
    B_1 &\in \mathcal{B} \\
    B_1 &= B_0 \cup \{g\}, \hspace{0.5cm} g\not= e
  \end{align*}
  By the definition of $C$, we know that $C$ could not extend to $B_0$.
  So there is an element $f\in C$ and $f\not\in B_0$.
  If $f\not= g$, then $C$ could not extend to $B_1$.
  Or if $f = g$, then $C\cup\{e\}$ could not extend to $B_1$.
\end{proof}

\begin{theorem}
  $\mathcal{B}(M/e)$ is an orientable matroid.
\end{theorem}
The proof of this theorem is similar to Theorem 1.
\end{document}
%%% Local Variables:
%%% mode: latex
%%% TeX-master: t
%%% End:
