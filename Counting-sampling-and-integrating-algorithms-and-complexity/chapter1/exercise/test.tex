\documentclass{article}

\usepackage{amsmath}
\usepackage{tikz}
%\usepackage{luacode}
%\usepackage{enumitem}

%\title{Exercise of Chapter 1}
%\author{Xiaoyu Chen}
%\date{}

\begin{document}
%\maketitle
%\section{Exercise 1.15}
%In the physics community, perfect matchings are sometimes known as ``dimer covers.'' It is of some interest to know the number of dimer covers of a graph $G$ when $G$ has a regular structure that models, for example, a crystal lattice. Let $\Lambda$ to be the $L\times L$ square lattice, with vertex set $V(\Lambda) = \{(i,j) : 0 \leq i,j < L\}$ and edge set $E(\Lambda) = \{(i, j), (i', j')\} : |i - i'| + |j - j'| = 1$. Exhibit a (nicely structured!) Pfaffian orientation of $\Lambda$.
%\subsection{solution}
%\begin{luacode*}
%  function LLGrid(L)
%    for x = 1, L do
%      for y = 1, L do
%        tex.print(string.format('\\node (p%d%d) at (%d, %d) [point] {};', x, y, x, y))
%      end
%    end
%    for x = 1, L do
%      for y = 1, L-1 do
%        tex.print(string.format('\\draw [->, semithick] (p%d%d) -- (p%d%d);', x, y+1, x, y))
%      end
%    end
%    for x = 1, L-1 do
%      for y = 1, L do
%        s = '->'
%        if y % 2 == 0 then s = '<-' end
%        tex.print(string.format('\\draw [%s, semithick] (p%d%d) -- (p%d%d);', s, x, y, x+1, y))
%      end
%    end
%  end
%\end{luacode*}
%The Pfaffian orientation of $\Lambda$ could be constructed using the following rules:
%\begin{enumerate}[nosep]
%    \item Make all the edges on the column point down.
%    \item Make all the edges on the odd row point right.
%    \item Make all the edges on the even row point left.
%\end{enumerate}
%\begin{center}
%  \begin{tikzpicture}
%  [point/.style={shape=circle, minimum size=1pt, inner sep=0pt, minimum size=5pt, fill}]  
%  \luadirect{LLGrid(5)}
%  \end{tikzpicture}
%\end{center}
%It is easy to see that all the faces in this grid has odd number of edges oriented clockwise.
\end{document}
