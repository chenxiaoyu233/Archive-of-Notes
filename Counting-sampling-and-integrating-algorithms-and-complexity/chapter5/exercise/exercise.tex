\documentclass{article}

\usepackage{amsmath, amssymb, amsthm}
\usepackage{mathrsfs}
\usepackage{tikz}
\usetikzlibrary{positioning, arrows}

\usepackage{enumitem}
\usepackage{tcolorbox}

\title{Exercise of Chapter 5}
\author{Xiaoyu Chen}
\date{}

\begin{document}
\maketitle

\section{Exercise 5.4}
Verify these claims about the lazy MC.
\subsection{solution}
\subsection{irreducibility}
First we want to prove the irreducibility of the MC. To achieve this, we only need to find a method which could construct a path for any $X, Y\in\Omega$.
\begin{tcolorbox}[title = {Suppose we have $X, Y\in\Omega$}]
  We could transform $X$ to $Y$ in following steps:
  \begin{enumerate}[itemsep=0mm]
  \item Remove all the elements in $X$ by continuely select $e\in X$.
  \item Add all the elements in $Y$ by continuely select $e\in Y \land e\not\in X$.
  \end{enumerate}
\end{tcolorbox}
\subsection{aperiodicity}
Note that for all state $X\in\Omega$ of the lazy MC we have: \[P^2(X,X)>0\], and \[P^3(X,X) > 0\]. Then by $\gcd(2, 3) = 1$, we know the lazy MC is aperiodic.

Putting the irreducibility and aperiodicity together we could draw a conclusion that the lazy MC is ergodic.

\section{Exercise 5.17}
Show how to tighten the upper bound in Proposition 5.11 by a factor $\overline{\lambda}$. Since Claim 5.15 is essentially tight when t is troublesome, it is necessary to improve
somehow the inequality
\[
  \sum_{(I,F)\in cp(t)} \pi(\eta_t(I,F)) \leq 1
\], by studying carefully the range of $\eta_t$. See Jerrum and Sinclair [45], specifically the proof of their Proposition 12.4.

\subsection{solution}
This solution is amost a summary of my understanding for the proof in Jerrum and Sinclair [45].
\paragraph{Case 1:} If we are not in the troublesome case:
\begin{align*}
  \lambda^{|I|}\lambda^{|F|} &= \lambda^{|M\cup M'|}\lambda^{|C|} \\
  \pi(I)\pi(F) &= \pi(M\cup M')\pi(C) \\
  &= \frac{m\pi(M\cup M')}{\min\{\pi(M), \pi(M')\}}\pi(M)P(M, M')\pi(C) \\
  &\leq m\overline{\lambda}\pi(M)P(M, M')\pi(C) \\
\end{align*}
The last step is because $\min\{|M|, |M'|\} \geq |M\cup M'| - 1$.
Then the congestion is:
\begin{align*}
  \varrho &= \max_{t = (M, M')} \left\{\frac{1}{\pi(M)P(M, M')}\sum_{(I,F)\in cp(t)} \pi(I)\pi(F)|\gamma_{IF}|\right\} \\
  &\leq m\overline{\lambda}\sum_{(I,F)\in cp(t)} \pi(\eta_t(I, F))|\gamma_{IF}| \\
  &\leq m\overline{\lambda}n\sum_{(I,F)\in cp(t)} \pi(\eta_t(I, F)) \\
  &\leq nm\overline{\lambda}
\end{align*}

\paragraph{Case 2:} If we are in the troublesome case, which, by the definition, means $t$ is a $\leftrightarrow$ and the current component is a cycle. Then, there must be at least 2 unmatched vertices. One of them is the neighbor of start vertex in $I$, and the other is surrounded by $e$ and $e'$. 
So, we could select a path between them in $I\oplus F$ following the cycle's orientation and augment on this path. This operation could give us another matching $\eta_t'(I, F)$. And this operation is reversible because we could find these two vertices according to $I\oplus F$ and $M\cup M'$. So, $\eta_t'$ is a injection from $cp(t)$ to $\Omega$. And we have $\lambda\pi(\eta_t(I, F)) = \pi(\eta_t'(I, F))$ since $\eta_t'(I, F)$ has one more edge than $\eta_t(I, F)$. Put them together, we have:
\begin{align*}
  \sum_{(I, F)\in cp(t)} \lambda\pi(\eta_t(I, F)) &= \sum_{(I, F)\in cp(t)} \pi(\eta_t'(I, F)) \leq 1 \\
  \lambda\sum_{(I, F)\in cp(t)} \pi(\eta_t(I, F)) &\leq 1 \\
  \sum_{(I, F)\in cp(t)} \pi(\eta_t(I, F)) &\leq \frac{1}{\lambda}
\end{align*}
By using the conclusion that \[\pi(I)\pi(F)\leq m\overline(\lambda)^2\pi(M)P(M, M')\pi(C)\] from the book, we could give the congestion:
\begin{align*}
  \varrho &= \max_{t = (M, M')} \left\{\frac{1}{\pi(M)P(M, M')}\sum_{(I,F)\in cp(t)} \pi(I)\pi(F)|\gamma_{IF}|\right\} \\
  &\leq m\overline{\lambda}^2 \sum_{(I, F)\in cp(t)} \pi(\eta_t(I, F))|\gamma_{IF}| \\
  &\leq nm\overline{\lambda}^2 \frac{1}{\lambda} \\
  &\leq nm\overline{\lambda}
\end{align*}

\begin{tcolorbox}[title = {Note:}]
  The most instresting thing of this problem is that the upper bould of the congestion is achieved by two different ways according to different cases.
  However, Jerrum tends to use algebratic symbols to represent the graph structure. This is not very friendly for me to understand his idea behind those symbols. Because each time when a symbol appears, I have to do some examples so that I could get what he means. Though its not very easy to understand those symbols, the idea (e.g. how to construct the $\eta_t(I, F)$) behind them is very easy to understand.
\end{tcolorbox}

\end{document}

%%% Local Variables:
%%% mode: latex
%%% TeX-master: t
%%% End:
