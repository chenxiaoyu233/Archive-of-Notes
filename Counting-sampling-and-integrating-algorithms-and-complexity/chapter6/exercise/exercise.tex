\documentclass{article}

\usepackage{amsmath, amssymb, amsthm}
\usepackage{tcolorbox}
\usepackage{tikz}

\title{Exercise of Chapter 6}
\author{Xiaoyu Chen}
\date{}

\begin{document}
\maketitle
\section{Exercise 6.3}
Show that the uniform distribution on $K$ is an invariant measure for the Metropolis version of the ball walk.
\subsection{solution}
The most interesting thing here is that the transition distribution $P(x,\cdot)$ of the Matropolis version of the ball walk is not continuous on its range $B(x,\delta)\cap K$. More specificly, for $y\in B(x,\delta)\cap K$, we have:
\[
  P(x, \mathrm{d}y) = \left\{
    \begin{array}{lr}
      \frac{\mathrm{d}y}{\mathrm{Vol}_n B(x, \delta)}, \hspace{0.5cm} & y\not= x \\
      \frac{\mathrm{Vol}_n(B(x,\delta)\setminus K)}{\mathrm{Vol}_n B(x, \delta)}, \hspace{0.5cm} &y = x
    \end{array}
  \right.
\]
So, for any $x, y\in K$, we have:
\[
  P(x,\mathrm{d}y) = P(y, \mathrm{d}x)
\]
Suppose $X_0$ has the uniform distribution $\mu$ and let $\mu_1$ be the distribution of $X_1$, then:
\begin{align*}
  \mu_1(A) &= \int_A \mu_1(dy) \\
  &= \int_A\int_K \mu(\mathrm{d}x) P(x, \mathrm{d}y) \\
  &= \int_A\int_{B(y,\delta)\cap K} \mu(\mathrm{d}x) P(x, \mathrm{d}y), \hspace{0.5cm} \mbox{since if $x\not\in B(y,\delta)\cap K$ then $P(x,\mathrm{d}y) = 0$}\\
  &= \int_A\int_{B(y,\delta)\cap K} \mu(\mathrm{d}x) P(y, \mathrm{d}x), \hspace{0.5cm} \mbox{since $P(x,\mathrm{d}y) = P(y,\mathrm{d}x)$} \\
  &= \int_A\int_{B(y,\delta)\cap K} \mu(\mathrm{d}y) P(y, \mathrm{d}x), \hspace{0.5cm} \mbox{since $\mu(\mathrm{d}x) = \mu(\mathrm{d}y)$ when we let $\mathrm{Vol}_n(\mathrm{d}x) = \mathrm{Vol}_n(\mathrm{d}y)$} \\
  &= \int_A \mu(\mathrm{d}y)\int_{B(y,\delta)\cap K}  P(y, \mathrm{d}x), \hspace{0.5cm}\\
  &= \int_A \mu(\mathrm{d}y), \hspace{0.5cm} \mbox{since $P(y,\cdot)$ is a distribution} \\
  &= \mu(A)
\end{align*}

\iffalse
\section{Exercise 6.10}
Verify Corollary 6.8. Doing this essentially involves translating Theorem 5.6 to the setting of continuous state space. In case you skip this exercise, a full derivation may be found in §6.8.
\subsection{solution}
Lets assume that $f$ is a integrable function on $K$. Then we have:
\begin{align*}
  [P_{zz}f] (x) &= \int_K P_{zz}(x,\mathrm{d}y) f(y) \\
  &= \int_K \frac{1}{2}(I(x,\mathrm{d}y) + P(x,\mathrm{d}y)) f(y) \\
  &= \frac{1}{2}f(x) + \int_K \frac{1}{2} P(x,\mathrm{d}y) f(y) \\
  &= \frac{1}{2}f(x)\int_K P(x,\mathrm{d}y) + \int_K \frac{1}{2} P(x,\mathrm{d}y) f(y) \\
  &= \frac{1}{2} \int_K P(x,\mathrm{d}y) (f(x) + f(y))
\end{align*}
Since the lazy MC must be ergodic, we denote its stationary distribution by $\mu$. Then we have:
\begin{align*}
  \mathbb{E}_\mu f &= \int_K \mu(\mathrm{d}x)f(x) \\
  \mathbb{E}_\mu [P_{zz}f] &= \int_K \mu(\mathrm{d}x) [P_{zz}f](x) \\
   &= \int_K \mu(\mathrm{d}x) \int_K P_{zz}(x,\mathrm{d}y)f(y) \\
   &= \int_K f(y) \int_K \mu(\mathrm{d}x)P_{zz}(x,\mathrm{d}y) \\
   &= \int_K f(y) \mu(\mathrm{d}y) \\
   &= \mathbb{E}_\mu f
\end{align*}
\begin{align*}
  \mathrm{Var}_\mu (P_{zz}f) &= \int_K \mu(\mathrm{d}x) \{[P_{zz}f](x)]\}^2 - \mathbb{E}_\mu^2[P_{zz}f] \\
  &= \frac{1}{4}\int_K \mu(\mathrm{d}x) (\int_K P(x,\mathrm{d}y)(f(x) + f(y)))^2 - \mathbb{E}_\mu^2[P_{zz}f] \\
  &\leq \frac{1}{4}\int_K\mu(\mathrm{d}x) \int_K P(x,\mathrm{d}y) (f(x)+f(y))^2 - \mathbb{E}_\mu^2[P_{zz}f] \\
  &= \frac{1}{4}\int_K\int_K\mu(\mathrm{d}x) P(x,\mathrm{d}y) (f(x)+f(y))^2 - \mathbb{E}_\mu^2[P_{zz}f]
\end{align*}

\begin{align*}
  \mathrm{Var}_\mu f &= \mathbb{E}_\mu f^2 - \mathbb{E}_\mu^2f \\
  &= \int_K \mu(\mathrm{d}x) f^2(x) - \mathbb{E}_\mu^2f \\
  &= \frac{1}{2}\int_K \mu(\mathrm{d}x) f^2(x) + \frac{1}{2}\int_K f^2(y)\mu(\mathrm{d}y)  - \mathbb{E}_\mu^2f \\
  &= \frac{1}{2}\int_K \mu(\mathrm{d}x) f^2(x)\int_K P(x,\mathrm{d}y) + \frac{1}{2}\int_K f^2(y)\int_K \mu(\mathrm{d}x)P(x,\mathrm{d}y)  - \mathbb{E}_\mu^2f \\
  &= \frac{1}{2} \int_K\int_K \mu(\mathrm{d}x)P(x,\mathrm{d}y)(f^2(x) + f^2(y)) - \mathbb{E}_\mu^2f
\end{align*}

\begin{align*}
  \mathrm{Var}_\mu f - \mathrm{Var}_\mu [P_{zz}f] &\geq \frac{1}{4}\int_K\int_K \mu(\mathrm{d}x) P(x,\mathrm{d}y) (f(x) - f(y))^2 \\
  &= \frac{1}{2} \varepsilon_P(f,f)
\end{align*}

So, for any integrable function $f:K\to\mathbb{R}$, we have:
\begin{align*}
  \mathrm{Var}_\mu(P_{zz}f) &\leq \mathrm{Var}_\mu f - \frac{1}{2}\varepsilon_P(f,f) \\
  &\leq \mathrm{Var}_\mu f - \frac{\lambda}{2}\mathrm{Var}_\mu f \\
  &= (1-\frac{\lambda}{2})\mathrm{Var}_\mu f \\
\end{align*}

Now we want to achieve the following bound:
\[
  |P_{zz}^t(x,A) - \mu(A)| \leq \varepsilon,\hspace{0.5cm} \forall A\subseteq K
\]
We could rewrite as:
\begin{align*}
  |[P_{zz}^tf](x) - \mu(A)| \leq \varepsilon 
\end{align*}

\begin{align*}
  \mathrm{Var}_\mu (P_{zz}^tf) &\geq \int_K ([P_{zz}^tf](x) - \mathbb{E}_\mu (P_{zz}f))^2 \mu(\mathrm{d}x) \\
  &\geq \int_K ([P_{zz}^tf](x) - \mu(A))^2 \mu(\mathrm{d}x) \\
  &\geq (\int_K ([P_{zz}^tf](x) - \mu(A)) \mu(\mathrm{d}x))^2 \\
  &\approx \left(\frac{1}{\mathrm{Vol}_n K} \int_K ([P_{zz}^tf](x) - \mu(A))dx\right)^2
\end{align*}
\fi

\section{Exercise 6.25}
Flesh out the details of the above proof sketch. (Prove the Poincare inequality with out the curvatre condition. This proof will loose a factor $n$ in $\lambda$.)
\subsection{Solution}
\subsubsection{Flesh out the proof of Lemma 6.21.(i)}
\begin{tcolorbox}[title={Recall It Here:}]
  \centering $l(x)^{1/n}$ is concave over $K$
  \tcblower
  Where the $l(x)$ is defined as
  \[\frac{\mathrm{Vol}_n (B(x, \delta)\cap K)}{\mathrm{Vol}_n B(x, \delta)}\]
\end{tcolorbox}
\begin{proof}
  Suppose there are two points $x$ and $y$ in $K$. From (6.32) we know that
  \[(\lambda x + (1-\lambda)y + B(0, \delta)) \cap K \supseteq \lambda((x + B(0, \delta))\cap K) + (1-\lambda)((y + B(0, \delta))\cap K)\]
  Then, by Brunn-Minkowski theorem, we have
  \begin{align*}
  \mathrm{Vol}_n&[(\lambda x + (1-\lambda)y + B(0, \delta)) \cap K]^{1/n} \\
    &\geq \lambda\mathrm{Vol}_n[(x + B(0, \delta))\cap K]^{1/n} + (1-\lambda)\mathrm{Vol}_n[(y + B(0, \delta))\cap K]^{1/n}
  \end{align*}
  Which means:
  \[l(\lambda x + (1-\lambda)y)^{1/n} \geq \lambda l(x)^{1/n} + (1-\lambda)l(y)^{1/n}\]
  So, $l(x)^{1/n}$ is concave on $K$.
\end{proof}
\subsubsection{Fix the upper bound of $a_{i, j}$}
From Lemma 6.22, we know that the sequence of 
\[\frac{1}{\mu(S_0)}, \frac{1}{\mu(S_1)},\cdots, \frac{1}{\mu(S_{m-1})}\]
is convex.
So we have
\begin{align*}
  a_{i,j} &\leq 2w_iw_j\sum_{k=i}^j \frac{1}{w_k} \\
  &= w_iw_j \sum_{k=i}^j (\frac{1}{w_k} + \frac{1}{w_{i + j - k}}) \\
  &\leq w_iw_j \sum_{k=1}^j (\frac{j-k}{j-i}\frac{1}{w_i} + \frac{k-i}{j-i}\frac{1}{w_j}) + (\frac{k-i}{j-i}\frac{1}{w_i} + \frac{j-k}{j-i}\frac{1}{w_j}) \\
  &= w_iw_j \sum_{k=i}^j (\frac{1}{w_i} + \frac{1}{w_j}) \\
  &= w_iw_j (j-i+1) (\frac{1}{w_i} + \frac{1}{w_j}) \\
  &= (j-i+1)(w_i + w_j)
\end{align*}
This establishes Claim 6.16 in the absencse of the curvature condition.

\subsubsection{Establish (6.29)}
Let $\eta = c_3\delta/n$, then by Lemma 6.21.(ii) we have
\begin{align*}
  |\ln l(x) - \ln l(y)| &\leq \frac{n}{\delta} || x - y || \\
  &\leq \frac{n}{\delta} (c_3\delta/n)\\
  &= c_3
\end{align*}
and hence
\begin{align*}
  l(x)/l(y) &\leq e^{c_3} \\
  l(x) &\leq l(y)e^{c_3} \\
\end{align*}
According to symmetry
\[l(y) \leq l(x)e^{c_3}\]
And since we could set the value of $\delta$ as we want, we may cliam that:
\[\mathrm{Vol}_nB(x,\delta') \leq s_1\mathrm{Vol}_nB(y,\delta')\]
where $s_1$ is some constant.
So we have a lower bound for all $B(y,\delta')$, i.e.
\[\frac{1}{s_1}\mathrm{Vol}_nB(x,\delta') \leq \mathrm{Vol}_nB(y,\delta')\]

So, from Lemma 6.24. we have
\begin{align*}
  \mathrm{Vol}_n&(B(x,\delta') \cap B(y, \delta') \cap K) \\
  &\geq \frac{1}{1 + e}\min\{\mathrm{Vol}_n(B(x, \delta')\cap K), \mathrm{Vol}_n(B(y,\delta')\cap K)\}
\end{align*}
and hence
\begin{align*}
  \mathrm{Vol}_n I \geq \frac{s_2}{1+e} \mathrm{Vol}_n (B(x,\delta')\cap K)
\end{align*}
where $s_2 = \min\{1, \frac{1}{s_1}\}$ is some constant.

Observe that $\delta' = \delta - \varepsilon\sqrt{n}$, so if we choose $\varepsilon$ properly,
we could make $\delta' = s_3 \delta$,
where $0 < s_3 \leq 1$ is a constant.

After that, we have
\[\frac{\mathrm{Vol}_nB(x,\delta')}{\mathrm{Vol}_nB(x,\delta)} = \frac{1}{s_3^n}\]
When $\frac{1}{s_3^n} = \frac{1}{2}$, we have
\[\frac{\mathrm{Vol}_n(B(x,\delta')\cap K)}{\mathrm{Vol}_n(B(x,\delta)\cap K)} \geq \frac{1}{2}\]
\begin{tcolorbox}[title={Some Note}]
  Although this seems right here, and some material online says it is trivial. I still can not prove it.
\end{tcolorbox}
After that, we have
\begin{align*}
  \int_{K_0} h \mathrm{d}u &= \frac{1}{2}\int_{K_0}\int_{K_0} \mu(\mathrm{d}x) \frac{\mathrm{d}y}{\mathrm{Vol}_n (B(x, \delta)\cap K)}(f(x) - f(y))^2 \\
  &\geq \frac{1}{2}\int_{K_0}\int_{I} \mu(\mathrm{d}x) \frac{\mathrm{d}y}{\mathrm{Vol}_n (B(x, \delta)\cap K)}(f(x) - f(y))^2 \\
  &\geq \frac{1}{2\mathrm{Vol}_n(B(x,\delta)\cap K)} \int_{I}\mathrm{d} y\int_{K_0} \mu(\mathrm{d}x) (f(x) - f(y))^2\\
  &\geq \frac{1}{2\mathrm{Vol}_n(B(x,\delta)\cap K)} \int_{I}\mathrm{d} y\int_{K_0} \mathrm{d}\mu (f - \overline{f})^2\\
  &\geq \frac{\mathrm{Vol}_n I}{2\mathrm{Vol}_n(B(x,\delta)\cap K)} \int_{K_0} \mathrm{d}\mu (f - \overline{f})^2\\
  &\geq \frac{s_2}{4(e+1)} \int_{K_0} (f - \overline{f})^2 \mathrm{d}\mu
\end{align*}
\subsubsection{The Order of $\lambda$}
From (6.26) we know that
\begin{align*}
  \lambda \in \mathcal{O}(\frac{1}{m^2}) &=  \mathcal{O}(\frac{c_3\delta}{Dn}) \\
  &= \mathcal{O}(\frac{c_3^2\delta^2}{D^2n^2})
\end{align*}

So, we draw a contradiction here, and fix the original proof.
\end{document}

%%% Local Variables:
%%% mode: latex
%%% TeX-master: t
%%% End:
