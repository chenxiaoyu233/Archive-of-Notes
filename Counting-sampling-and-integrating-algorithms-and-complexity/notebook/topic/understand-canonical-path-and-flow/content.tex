\section{Understand Canonical Path And Flow}
After reading the topic on canoical path on Jerrum Book, I do not think that I have a very deeply understanding of this technique. Because the calculation used by it is very complex and magical, where the book does not explain the intuition behind the calculation

\begin{define}[Indicator Function]
  \index{indicator function}
  The function $f$ we use here, should be an indicator of some set $A\subseteq\Omega$. That is
  \[f(x) := [x\in A], \hspace{0.5cm} \forall x\in\Omega\]
  \tcblower
  In the latter part of this article, we may use $f_A$ instead of $f$ if we want to refer to specific set $A$.
\end{define}
\begin{define}[Inner Product On Distribution]
  \index{inner product on distribution}
  Suppose we have two function (or vector) $f$ and $g$, the we could define their inner product based on $\pi$:
  \[\langle f, g \rangle_\pi := \sum_{x\in\Omega} f(x)g(x)\pi(x)\]
\end{define}
Lets have a look at the following concept with our new tools:
\begin{fact}
  \[\mathbb{E}_\pi f = \pi(A)\]
\end{fact}
\begin{proof}
  Lets treat $f$ and $\pi$ as two vector, then:
  \begin{align*}
    \mathbb{E}_\pi f &= \pi f \\
    &= \pi([x \in A]) \\
    &= \pi(A) \qedhere
  \end{align*}
\end{proof}
\begin{corollary}
  Suppose $P$ is a transition matrix of some Markov Chain, then:
  \[\mathbb{E}_\pi [Pf] = \mathbb{E}_\pi f = \pi(A)\]
\end{corollary}
\begin{proof}
  \begin{align*}
    \mathbb{E}_\pi [Pf] &= \pi Pf \\
    &= \pi f, \hspace{0.5cm} \mbox{since $\pi$ is stationary distribtion} \\
    &= \mathbb{E}_\pi \qedhere
  \end{align*}
\end{proof}
A interesting thing here is that we could treat a Markov Chain with a stationary distribution $\pi$ as a multicommodity flow network, although the ``flow'' we use here is different from the flow in context of algorithm.
This network works under the following rules:
\begin{algorithm}
  Network Working Schedule \\
  \Begin{
    $t = 0$ \tcp{Time Counter}
    $P$ \tcp{The Transition Matrix of Our MC}
    $\pi_0$ \tcp{Some Distribution}
    \tcc{$\pi_0(v)$ means the quantity of commodities on $v$}
    \While{$t\gets t+1$} {
      $\pi_t(\cdot) \gets 0$ \\
      \For{$u, v\in\Omega$} {
        $\pi_t(v)\gets\pi_t(v) + \pi_{t-1}(u)P(u, v)$ \\
        \tcc{$\pi_t(v) = \sum_{u\in\Omega}\pi_{t-1}(u)P(u,v)m ,$}
      }
    }
  }
\end{algorithm}

It is clear that from time $t-1$ to time $t$, there is a $\pi_{t-1}(u)P(u,v)$ units of flow on edge $(u, v)$. And it is easy to verify that from time $t-1$ to time $t$, there are
\[\sum_{u\in\Omega}\pi_{t-1}(v)P(v, u) = \pi_{t-1}\]
units of commodities flow out $v$ and there are
\[\sum_{u\in\Omega}\pi_{t-1}(u)P(u, v) = \pi_{t}\]
units of commodities flow in $v$.

So, it seems that any vertex in the network will run out of all its commodities in current turn, and get all its commodities for the next turn from its neighbors. To go further in this topic, we borrow some notations from the textbook:
\begin{define}[Flow Function In One Step]
  We want a functoin $Q_1^\pi(\cdot, \cdot)$ to measure the quantity of the flow from one part to another part in one turn (start with the distribution $\pi$). Suppose we have two sets $A$ and $B$, then the flow from $A$ to $B$ is represent by:
  \[Q_1^\pi(A, B) = \sum_{a\in A, b\in B} \pi(a)P(a, b)\]
  \tcblower
  Here, $\pi$ could be any distribution.
  If we need to distinguish the transition matrix $P$ in the context, we use $\leftidx{_P}{Q}{_1^\pi}$ instead of $Q_1^\pi$.
\end{define}
Now consider flow cross many steps.
\begin{proposition}
  Suppose the commodities at each vertex will be well mixed after each turn, then
  \[\mathbb{E} [Q_t^\pi(A, B)] = \sum_{a\in A, b\in B} \pi(a)P^t(a, b)\]
  \tcblower
  Here, $\pi$ could be any distribution.
\end{proposition}
\begin{proof}
  We prove this by induction.

  In the base case: $\mathbb{E}[Q_1^\pi(A, B)] = Q_1^\pi(A, B)$. 

  In the inductive case: Suppose $t = n$, and our assumption holds in $t < n$. Then
  \begin{align*}
    \mathbb{E} [Q_n^\pi(A, B)] &= \sum_{a\in A, b\in B} \mathbb{E}[Q_n^\pi(a, b)] \\
    &= \sum_{\substack{a\in A, b\in B\\ m\in\Omega}} \mathbb{E}[Q_{n-1}^\pi(a, m)] P(m, b) \numberthis \label{eq:probability-on-edge} \\
    &= \sum_{\substack{a\in A, b\in B\\ m\in\Omega}} \pi(a)P^{n-1}(a,m) P(m, b), \hspace{0.5cm} \mbox{by assumption} \\
    &= \sum_{a\in A, b\in B} \pi(a) \sum_{m\in\Omega}P^{n-1}(a,m) P(m, b) \\
    &= \sum_{a\in A, b\in B} \pi(a) P^n(a, b) 
  \end{align*}
  \marginnote{
    Actually, we could treat the commodities as some liquid distinguishing by their start point. In each turn, we mix them together first and then push them to pipes (or edges) by the ratio ($P(u, v)$) on the pipe. (Maybe we could throw the $\mathbb{E}$ away by understanding the process like this)
  }
  Because we could treat $P(a,\cdot)$ as a distribution. So the expect quantity of flow (which has moved from $a$ to $m$) moves from $m$ to $b$ in step $n$ is
  \[\mathbb{E}[Q_{n-1}^\pi(a, m)] P(m, b)\]
  And hence we have the Equation \ref{eq:probability-on-edge}.
\end{proof}
\begin{corollary}
  \[\lim_{t\to\infty} \mathbb{E}[Q_t^\mu(A, B)] = \mu(A)\pi(B)\]
\end{corollary}
\begin{proof}
  \begin{align*}
    \lim_{t\to\infty} \mathbb{E}[Q_t^\mu(A, B)] &= \lim_{t\to\infty} \sum_{a\in A, b\in B} \mu(a)P^t(a, b) \\
    &= \sum_{a\in A, b\in B} \mu(a)\pi(b), \hspace{0.1cm} \mbox{$\pi$ is the stationary distribution} \\
    &= \mu(A)\pi(B) \qedhere
  \end{align*}
\end{proof}
For convenience, we use $Q_t$ instead of $\mathbb{E}_\pi [Q_t]$ latter in this article.
\begin{define}
  For convenience:
  \[Q^\mu(A, B) := \lim_{t\to\infty} \mathbb{E}[Q_t^\mu(A, B)] = \mu(A)\pi(B)\]
  \tcblower
  Moreover,
  \[Q(A, B) := \lim_{t\to\infty} \mathbb{E}[Q_t^\pi(A, B)] = \pi(A)\pi(B)\]
  where $\pi$ is the stationary distribution of the MC.
\end{define}
So far so good, we have a useful tool to explain the intuition behind the calculation for canonical path on Jerrum Book. Now, we need to focus on some important concepts which is used in the proof of the ``canonical path'' (e.g. $\mathrm{Var}_\pi f$, $\mathrm{Var}_\pi [P_{zz}f]$, $\mathcal{E}_P(f, f)$, etc.).
\subsection{Explaination For $\mathrm{Var}_\pi f$}
\marginnote[1cm]{
  Note that the distribution $\pi$ here is the stationary distribution for the MC.
}
First, we should note that there is a special form of variance:
\[\mathrm{Var}_\pi f = \frac{1}{2} \sum_{x, y\in\Omega} \pi(x)\pi(y)(f(x) - f(y))^2\]
where you could refer to Proposition \ref{prop:variance-flow} for how this comes.
\begin{lemma}
  \[\mathrm{Var}_\pi f = \pi(A^c)\pi(A), \hspace{0.5cm} \mbox{for $f = f_A$} \]
\end{lemma}
\begin{proof}
  Note that
  \[(f(x) - f(y))^2 = [x\in A]\oplus[y\in A]\]
  So we have
  \begin{align*}
    \mathrm{Var}_\pi f &= \frac{1}{2}\sum_{x, y\in\Omega} \pi(x)\pi(y)(f(x) - f(y))^2 \\
    &= \frac{1}{2}\sum_{x, y\in\Omega} \pi(x)\pi(y) ([x\in A]\oplus[y\in A]) \\
    &= \frac{1}{2}\sum_{x\in A, y\in\Omega\setminus A} \pi(x)\pi(y) + \frac{1}{2}\sum_{x\in\Omega\setminus A, y\in A} \pi(x)\pi(y) \\
    &= \sum_{x\in A, y\in\Omega\setminus A} \pi(x)\pi(y) \\
    &= \sum_{x\in A, y\in A^c} \pi(x)\pi(y) \\
    &= \pi(A^c)\pi(A) \qedhere
  \end{align*}
\end{proof}
\begin{corollary}
  \[\mathrm{Var}\pi f + (\mathbb{E}_\pi f)^2 = \pi(A)\]
\end{corollary}
\begin{proof}
  Note that
  \[(\mathbb{E}_\pi f)^2 = \pi(A)\pi(A)\]
  So we have
  \begin{align*}
    \mathrm{Var}\pi f + (\mathbb{E}_\pi f)^2 &= \pi(A^c)\pi(A) + \pi(A)\pi(A) \\
    &= \pi(\Omega)\pi(A) \\
    &= \pi(A) \qedhere
  \end{align*}
\end{proof}

\begin{lemma}
  \[\mathrm{Var}_\pi f + (\mathbb{E}_\pi f)^2 = \frac{1}{2} \sum_{x, y\in\Omega} \pi(x)P(x, y)(f(x)^2 + f(y)^2)\]
  \tcblower
  This equation appears on the book, we could explain it use the language of flow.
\end{lemma}
\begin{proof}
  According to the rule of the flow. $\pi(A)$ is the quantity of the commodities in the begining. Because $\pi$ is the stationary disbtribution of MC, so $\pi P = \pi$. This means the quantity of commodities in $A$ should still be $\pi(A)$ after one trun of the flow. So we have
  \begin{align*}
    \pi(A) &= \mbox{\# of commodities from $A$} + \mbox{\# of commodities from $A^c$} \\
    &= \sum_{u\in A, v\in A} \pi(u)P(u,v) + \sum_{u\in A^c, v\in A} \pi(u)P(u,v) \\
    &= \sum_{[u\in A\land v\in A]} \pi(u)P(u,v) + \sum_{[u\in A^c\land v\in A]} \pi(u)P(u,v) \\
    &= \sum_{[u\in A\land v\in A]} \pi(u)P(u,v) + \frac{1}{2}\sum_{[u\in A]\oplus[v\in A]} \pi(u)P(u,v) \\
    &= \frac{1}{2} \sum_{x,y\in\Omega} \pi(x)P(x, y) (f(x)^2 + f(y)^2)
  \end{align*}
  The last step is because $\frac{1}{2}(f(x)^2 + f(y)^2)$ could be treat as an indicator:
  \[
    \frac{1}{2}(f(x)^2 + f(y)^2) = \left\{
      \begin{array}{l}
        1, \hspace{0.5cm} x\in A\land y\in A \\
        1/2, \hspace{0.5cm} [x\in A]\oplus[y\in A] \\
        0, \hspace{0.5cm} \mbox{otherwise}
      \end{array}
    \right. \qedhere
  \]
\end{proof}
\subsection{Explaination For $\mathcal{E}_P(f ,f)$}
\begin{fact}
  \[\mathcal{E}_P(f, f) = \leftidx{_P}{Q}{_1^\pi} (A, A^c)\]
  \tcblower
  So $\mathcal{E}_P(f, f)$ means the flow from $A^c$ to $A$ in one step (start from $\pi$).
\end{fact}
\begin{proof}
  From Jerrum Book, we could know:
  \begin{align*}
    \mathcal{E}_\pi (f, f) &= \frac{1}{2} \sum_{x, y\in\Omega}\pi(x)P(x, y) (f(x) - f(y))^2 \\
    &= \frac{1}{2} \sum_{x, y\in\Omega}\pi(x)P(x, y) [x\in A]\oplus[y\in A] \\
    &= \frac{1}{2} \sum_{[x\in A]\oplus[y\in A]}\pi(x)P(x, y) \\
    &= \sum_{x\in A^c, y\in A}\pi(x)P(x, y) \\
    &= Q_1^\pi(A^c, A) \qedhere
  \end{align*}
\end{proof}

\begin{fact}
  \[\mathrm{Var}_\pi f + (\mathbb{E}_\pi f)^2 - \frac{1}{2}\mathcal{E}_P(f, f) = \frac{1}{4}\sum_{x, y\in\Omega}\pi(x)P(x,y)(f(x) + f(y))^2\]
\end{fact}
\begin{proof}
  First we have:
  \[
    \frac{1}{4}(f(x) + f(y))^2 = \left\{
      \begin{array}{lr}
        1, & x\in A, y\in A \\
        1/4, & x\in A, y\in A^c \\
        1/4, & x\in A^c, y\in A \\
        0, & \mbox{otherwise}
      \end{array}
    \right.
  \]
  So
  \begin{align*}
    \mathrm{Var}_\pi f &+ (\mathbb{E}_\pi f)^2 - \frac{1}{2}\mathcal{E}_P(f, f) \\
    &= \sum_{x, y\in A} \pi(x)P(x,y) + \frac{1}{2}\sum_{x\in A^c, y\in A} \pi(x)P(x,y) \\
    &= Q_1^\pi(A, A) + \frac{1}{2}Q_1^\pi(A^c, A) \\
    &= \pi(A) - \frac{1}{2}Q_1^\pi(A^c, A) \qedhere
  \end{align*}
\end{proof}
\begin{corollary}[\label{coro:Q1(A,A)}]
  \[\mathrm{Var}_\pi f + (\mathbb{E}_\pi f)^2 - \mathcal{E}_{P_{zz}}(f, f) = \pi(A) - \leftidx{_{P_{zz}}}{Q}{_1^\pi}(A^c, A) = \leftidx{_{P_{zz}}}{Q}{_1^\pi}(A, A)\]
\end{corollary}

\subsection{Explaination For $\mathrm{Var}_\pi [P_{zz}f]$}
Since $\mathbb{E}_\pi [P_{zz}f] = \mathbb{E}_\pi f$, we only consider $\mathrm{Var}_\pi [P_{zz}f] + (\mathbb{E}_\pi f)^2$.
\begin{lemma}
  \[\mathrm{Var}_\pi [P_{zz}f] + (\mathbb{E} f)^2 = \leftidx{_{P_{zz}}}{Q}{_2^\pi}(A, A)\]
\end{lemma}
\begin{proof}
  \begin{align*}
    \mathrm{Var}_\pi [P_{zz}f] + (\mathbb{E} f)^2 &= \sum_{x\in\Omega} \pi(x)([P_{zz}f](x))^2 \\
    &= \sum_{x\in\Omega} \pi(x)(P_{zz}(x, A))^2 \\
    &= \sum_{x\in\Omega} \pi(x)\sum_{y\in A} P_{zz}(x, y) \sum_{z\in A} P_{zz}(x, z)\\
    &= \sum_{\substack{y\in A, z\in A\\x\in\Omega}} \pi(x)P_{zz}(x,y)P_{zz}(x,z) \\
    &= \sum_{\substack{y\in A, z\in A\\x\in\Omega}} \pi(y)P_{zz}(y,x)P_{zz}(x,z), \hspace{0.5cm} \mbox{assume $\pi(x)P_{zz}(x, y) = \pi(y)P_{zz}(y,x)$} \\
    &= \sum_{y\in A, z\in A} \pi(y)P_{zz}^2(y,z) \\
    &= \leftidx{_{P_{zz}}}{Q}{_2^\pi}(A, A)
  \end{align*}
\end{proof}
\begin{corollary}
  \[\mathrm{Var}_\pi [P_{zz}^tf] + (\mathbb{E}_\pi P_{zz}^tf)^2 = \leftidx{_{P_{zz}}}{Q}{_{2t}^\pi}\]
\end{corollary}
We find the connection between $\mathrm{Var}_\pi f$ and $\mathrm{Var}_\pi [P_{zz}f]$, since Corollary \ref{coro:Q1(A,A)}
\marginnote[1cm]{
  If you want to relax the restriction on $f$, you need to proof this theorem in the way the Jerrum Book does. But that is pretty hard to understand. The proof we give here only aims at gaining the intuition.
}
\begin{theorem}
  \[\mathrm{Var}_\pi (P_{zz}f) \leq \mathrm{Var}_\pi f - \mathcal{E}_{P_{zz}} (f, f)\]
  \tcblower
  This theorem is true on any $f:\Omega\to\mathbb{R}$ (i.e. not just indicator).
\end{theorem}
\begin{proof}
  Since
  \begin{align*}
    \mathrm{Var}_\pi (P_{zz}f) + (\mathbb{E}_\pi f)^2 &\leq \mathrm{Var}_\pi f + (\mathbb{E}_\pi f)^2 - \mathcal{E}_{P_{zz}} (f, f) \\
    \leftidx{_{P_{zz}}}{Q}{_2^\pi}(A,A) &\leq \leftidx{_{P_{zz}}}{Q}{_1^\pi}(A, A) \numberthis \label{eq:Q2 <= Q1}
  \end{align*}
  Thus we only need to proof Equation \ref{eq:Q2 <= Q1}.
  Actually, under the assumption of $\pi(x)P(x,y) = \pi(y)P(y,x)$, we have:
  \begin{align*}
    \leftidx{_{P_{zz}}}{Q}{_2^\pi}(A, A) &= \sum_{x\in\Omega}\pi(x)P_{zz}(x, A)P_{zz}(x, A) \\
    &\leq \sum_{x\in\Omega} \pi(x)P_{zz}(x, A), \hspace{0.5cm} \mbox{Since $P_{zz}(x, A) \leq 1$} \\
    &= \leftidx{_{P_{zz}}}{Q}{_1^\pi}(A,A) \qedhere
  \end{align*}
\end{proof}
\begin{corollary}
    \[\pi(A)\pi(A) \leq \leftidx{_{P_{zz}}}{Q}{_{t+1}^\pi}(A,A) \leq \leftidx{_{P_{zz}}}{Q}{_t^\pi}(A, A) \leq \leftidx{_{P_{zz}}}{Q}{_1^\pi}(A,A)\] 
\end{corollary}

\begin{theorem}
  For $P^t = P_{zz}^t$, we have
  \[\mathrm{Var}_\pi[P^{t+1}f] \leq \mathrm{Var}_\pi [P^tf] - \mathcal{E}_{P_{zz}}(P^tf, P^tf)\]
\end{theorem}
\begin{proof}
  First, note that
  \[\mathrm{Var}_\pi[P^{t+1}f] + (\mathbb{E}_\pi [P^{t+1}f])^2 = Q_{2t+2}^\pi(A, A)\]
  and
  \[\mathrm{Var}_\pi[P^{t}f] + (\mathbb{E}_\pi [P^{t}f])^2 = Q_{2t}^\pi(A, A)\]
  Thus, we only need to prove that
  \[\mathcal{E}_{P_{zz}}(P^tf, P^tf) \leq Q_{2t}^\pi - Q_{2t+2}^\pi\]
  Moreover
  \begin{align*}
    \mathcal{E}_{P_{zz}} (P^tf, P^tf) &= \frac{1}{2} \sum_{x,y\in\Omega} \pi(x)P(x, y) (P^tf(x) - P^tf(y))^2 \\
    &= \frac{1}{2} \sum_{x,y\in\Omega} \pi(x)P(x, y) (P^t(x, A) - P^t(y, A))^2 \\
    &= \sum_{x,y\in\Omega} \pi(x)P(x, y) (P^t(x, A))^2 - \sum_{x, y\in\Omega} \pi(x)P(x, y) P^t(x,A) P^t(y, A) \\
    &= \sum_{x,y\in\Omega} \pi(x)P(x, y) (P^t(x, A))^2 - \sum_{x,y\in\Omega} \pi(x)P(x, y) \sum_{a\in A} P^t(x, a) \sum_{b\in A} P^t(y, A) \\
    &= \sum_{x,y\in\Omega} \pi(x)P(x, y) (P^t(x, A))^2 - \sum_{x,y\in\Omega} P(x, y) \sum_{a\in A} \pi(a) P^t(a, x) \sum_{b\in A} P^t(y, A) \\
    &= \sum_{x,y\in\Omega} \pi(x)P(x, y) (P^t(x, A))^2 - \sum_{\substack{x,y\in\Omega\\a, b\in A}}  \pi(a) P^t(a, x) P(x, y) P^t(y, b) \\
    &= \sum_{x,y\in\Omega} \pi(x)P(x, y) (P^t(x, A))^2 - \sum_{a, b\in A}  \pi(a) P^{2t+1}(a, b) \\
    &= \sum_{x,y\in\Omega} \pi(x)P(x, y) (P^t(x, A))^2 - Q_{2t+1}^\pi(A, A) \\
    &\leq \sum_{x,y\in\Omega} \pi(x)(P^t(x, A))^2 - Q_{2t+1}^\pi(A, A) \\
    &= Q_{2t}^\pi(A, A) - Q_{2t+1}^\pi(A, A) 
  \end{align*}
  So, we only need to prove that
    \[Q_{2t}^\pi(A, A) - Q_{2t+1}^\pi(A, A)\leq Q_{2t}^\pi(A, A) - Q_{2t+2}^\pi(A, A)\]
  Since $Q_{2t+2}^\pi(A, A) \leq Q_{2t+1}^\pi(A, A)$, this inequality is obviously true.
\end{proof}

We won't go futher here because we have already gain some intuition from above. Next, we only need to use the lower bound in the book:
\[\mathcal{E}_P(f, f) \geq \frac{1}{\varrho}\mathrm{Var}_\pi f\]
And we have
\[\mathrm{Var}_\pi(P_{zz}f) \leq (1 - \frac{1}{2\varrho})\mathrm{Var}_\pi f\]
Its enough, and we stop here.

%%% Local Variables:
%%% mode: latex
%%% TeX-master: "../../notebook"
%%% End:
