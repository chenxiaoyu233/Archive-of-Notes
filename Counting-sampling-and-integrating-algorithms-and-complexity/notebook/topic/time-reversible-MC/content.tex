\section{Time Reversible Markov Chain}
\marginnote[5mm]{
  Recall that we usuall denote a Markov Chain by $\mathcal{M} = (\Omega, P)$.
}
\begin{define}[Reversed Markov Chain]
  \index{reversed markov chain}
  A reversed MC $\mathcal{M}^R: (\Omega, Q)$ is induced by the following rule:
  \[
    Q(i, j) = \Pr[X_m=j|X_{m+1}=i]
  \]
\end{define}
Then we have:
\begin{align*}
    Q(i, j) &= \Pr[X_m=j|X_{m+1}=i] \\
    &= \frac{\Pr[X_m=j\land X_{m+1}=i]}{\Pr[X_{m+1}=i]} \\
    &= \frac{\Pr[X_m = j] P(j, i)}{\Pr[X_{m+1} = i]} \\
    &= \frac{\pi(j)P(j,i)}{\pi(i)}
\end{align*}

\begin{define}[time-reversible Markov Chain]
  \index{time-reversible markov chain}
  A MC $\mathcal{M}: (\Omega, P)$ is time-reversible, if it equals to its reversed MC $\mathcal{M}^R: (\Omega, Q)$, i.e. $P = Q$.
\end{define}
Note that, this definition equals to saying:
\[
  \pi(i)P(i, j) = \pi(j)P(j,i),\hspace{0.5cm} \forall i, j\in\Omega
\] 
Moreover, if \[\sum_{\omega\in\Omega}\pi(\omega) = 1\], then we could infer that $\pi$ is a stationary distribution of $\mathcal{M}$.
%%% Local Variables:
%%% mode: latex
%%% TeX-master: "../../notebook"
%%% End:
