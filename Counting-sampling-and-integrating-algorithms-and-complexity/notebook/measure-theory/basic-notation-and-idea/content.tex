\section{Basic Notations and Ideas}

\subsection{Non-measurealbe Sets}
Measure is very important since the integrating and probability are built on top of it.
At the begining, the idea is very simple:
\paragraph{}\emph{
  The concept for the measure of length grows up naturally.
  For a set $(a, b]\in \mathbb{R}$, the length of it is $b - a$.
  As the interval $(a, b]$ could be seen as a subset of $\mathbb{R}$,
  we might ask, could we define a function $f:\mathcal{P}(\mathbb{R}) \to \mathbb{R}$
  to give a measure for all the subset of $\mathbb{R}$?
} 

\paragraph{}
At first, some people start to find such a function $f$ on $\mathcal{P}(\mathbb{R})$.
Ideally, if we want to define a function $\lambda$ like this, we'd like it satisfies some properties:
\begin{itemize}[itemsep=0mm]
\item (0) $\lambda: \mathcal{P}(\mathbb{R}) \to \mathbb{R}_{+}\cup\{+\infty\}$
\item (1) $\lambda((a, b]) = b - a$, since we want it to remain its nature meaning.
\item (2) $\lambda(A + x) = \lambda(A)$, since the shift should not change the volume.
\item (3) $\lambda(\cup_{j\geq 1} A_j) = \sum_{j\geq 1} \lambda(A_j)$ for coutable  disjoint $A_j$. Since we need to add the volume for calculation.
\end{itemize}

But soon, they find it impossible, as we could prove this function could not exists.
Otherwise, there must be a contradiction.
\begin{theorem}
  The function that satisfies the requirements above does not exist.
\end{theorem}
\begin{proof} 
  For $x, y\in\mathbb{R}$, let $x\sim y$ if $y - x\in\mathbb{Q}$.
  Let $[x] = \{y \in \mathbb{R} | y - x \in \mathbb{Q}\}$ be a equivalent class.
  Let $\Lambda = \mathbb{R}|\sim$, which means the set of equivalent classes.
  \marginnote{$\Lambda$ is uncountable, since $\mathbb{R}$ is uncountable, and any $[x]\in\Lambda$ is countable. We may use $\alpha, \beta$ to represent the point in $\Lambda$.}
  Now, we construct a set $\Omega\in\mathbb{R}$ be the set which contains one and only one points from each equivalent class. Thus we could assume that $\Omega\subseteq(0, 1)$.

\begin{claim}[For a set $\Omega\subseteq\mathbb{R}$, and $p, q\in\mathbb{R}$. Either we have $\Omega+p = \Omega+q$, eigher we have $\Omega+p$ disjoints with $\Omega+q$]
  \begin{proof}
    First, lets assume $(\Omega + p)\cap(\Omega+q)\not=\emptyset$.
    Then for some $x$ in it, we have
    \begin{align*}
      x &= \alpha + p, \hspace{0.5cm} \alpha\in\Omega \\
        &= \beta + q,  \hspace{0.5cm} \beta \in\Omega
    \end{align*}
    So we have $\alpha - \beta = q - p$.
    From the definition of $\Omega$, we know that $p = q$ and $\alpha = \beta$.
    So we have \[q\not= p \Rightarrow (\Omega + q)\cap(\Omega+p) = \emptyset\].
  \end{proof}
\end{claim}
  By this claim, we could define a set:
  \[ S = \sum_{\substack{q\in\mathbb{Q}\\-1 < q < 1}}(\Omega + q) \]
  Its easy to see that for all these $q$, $\Omega + q \subseteq (-1, 2)$.
  So we have $S\subseteq(-1, 2)$ and $\lambda(S) \leq \lambda((-1, 2)) = 3$.
  So by property (3), we have
  \[\sum_{\substack{q\in\mathbb{Q}\\-1<q<1}} \lambda(\Omega + q) \leq 3\]
  And by property (2), we have $\lambda(\Omega + q) = \lambda(\Omega) = 0$.
  \marginnote{
    Here, since we have infinity $q$s, $\lambda(\Omega + q) = 0$. Otherwise, the sum of them could not less than $3$.
  }
  By observing this, we have $\lambda(S) = 0$.
  \begin{claim}[$(0, 1) \subseteq S$]
    \begin{proof}
      Suppose we have $x\in(0,1)$ and $\alpha\in [x]\cap\Omega$, then we have $\alpha\in(0, 1)$ (by definition of $\Omega$). Since $\alpha$ belongs to the equivalent class of $[x]$, $\alpha - x = q\in\mathbb{Q}$. Its easy to see that $-1 < q < 1$. Which means $x=\alpha + q$ and thus $x\in\Omega + q$ for some $q\in(0,1)$. Which is exactly in $S$. So we have $(0, 1)\subseteq S$.
    \end{proof}
  \end{claim}
  So far, we have find a contradiction that there is no such function which could satisfies all the properties.
\end{proof}

What can we do now? Maybe we could relax some of these properties.
Maybe we could remove some this conditions.
Maybe we could accept the function which is not defined on all the subsets of $\mathbb{R}$. This means there are some kinds of subsets of $\mathcal{P}$ which we could not set a measure on it, these sets are called non-measurealbe sets.

\subsection{Classes of subsets, and set functions}
\begin{define}[semi-algebra]
  \index{semi-algebra}
  $\mathcal{S} \subseteq \mathcal{P}(\Omega)$, is a semi-algebra if:
  \begin{enumerate}[itemsep=0mm]
  \item $\Omega \in \mathcal{S}$.
  \item $A, B\in \mathcal{S} \Rightarrow A\cap B\in \mathcal{S}$. \\ (\emph{This means it is closed under finite intersections})
  \item $\forall A\in\mathcal{S} \Rightarrow$ $\exists E_1, E_2, \cdots, E_n\in\mathcal{S}$ such that $A^c = \cup_{j=1}^n E_j$.
  \end{enumerate}
\end{define}

\begin{define}[algebra]
  \index{algebra (in measure theory)}
  $\mathcal{A} \subseteq \mathcal{P}(\Omega)$, is an algebra if:
  \begin{enumerate}[itemsep=0mm]
  \item $\Omega \in \mathcal{A}$.
  \item $A, B\in \mathcal{A} \Rightarrow A\cap B\in \mathcal{A}$.
  \item $A\in\mathcal{A} \Rightarrow A^c\in\mathcal{A}$. \\ (\emph{Here is the difference between semi-algebra and algebra})
  \end{enumerate}
\end{define}

\begin{define}[$\sigma$-algebra]
  \index{$\sigma$-algebra}
  $\mathcal{F} \subseteq \mathcal{P}(\Omega)$, is an algebra if:
  \begin{enumerate}[itemsep=0mm]
  \item $\Omega\in\mathcal{F}$.
  \item $A_j\in\mathcal{F}$ for $j\geq 1$ $\Rightarrow$ $\cap_{j\geq 1} A_j \in \mathcal{F}$. \\ (\emph{closed under countable intersections})
  \end{enumerate}
\end{define}

\begin{remark}
  $\sigma$-algebra $\Rightarrow$ algebra $\Rightarrow$ seim-algebra.
  $\sigma$-algebra asks for more constraint than algebra, and thus it is bigger than an algebra (this means you could find some algebra in $\sigma$-algebra or $\sigma$-algebra is itself an algebra).
\end{remark}

\begin{observation}
  If $\mathcal{A}_{\alpha} \subseteq \mathcal{P}(\Omega)$ is ($\sigma$)-algebra for some $\alpha\in I$. Then we have $\mathcal{A} = \cap_{\alpha\in I} \mathcal{A}_\alpha$ is a ($\sigma$)-algebra.
\end{observation}
we can use these fact to introduce the notion of the algebra generated by the classes of sets.
\begin{define}
  The algebra generated by class $\mathcal{C}\subseteq\mathcal{P}(\Omega)$ is denoted by $\mathcal{A}(\mathcal{C})$. And $\mathcal{A}(\mathcal{C})$ satisfies:
  \begin{itemize}
  \item $\mathcal{C}\subseteq\mathcal{A}(\mathcal{C})$, $\mathcal{A}(\mathcal{C})$ is algebra.
  \item \[
      \left\{
        \begin{array}{l}
          \mathcal{C} \subseteq \mathcal{B} \\
          \mathcal{B} \mbox{ is algebra} 
        \end{array}
      \right. \Rightarrow \mathcal{A}(\mathcal{C})\subseteq \mathcal{B}
    \]
  \end{itemize}
\end{define}
Let $\mathcal{A}_\alpha$ be all the algebra that contains $\mathcal{C}$.
Then it is easy to verify that
\[\mathcal{A}(\mathcal{C}) = \bigcap_\alpha \mathcal{A}_\alpha\].

\begin{lemma}
  $\mathcal{S}\subseteq\mathcal{P}(\Omega)$ is a semi-algebra, $\mathcal{A}$ is the algebra generated by $\mathcal{S}$. Then
  \[A\in\mathcal{A} \Leftrightarrow \substack{\exists E_j, i\leq j\leq n, E_j \in \mathcal{S}\\A=\sum_{j=1}^n E_j}\]
\end{lemma}
\begin{proof}
  \emph{($\Leftarrow$):} This is obvious from the definition of $\mathcal{A}$.

  \emph{($\Rightarrow$):} Define a new class
  \[\mathcal{B} = \{\sum_{j=1}^n F_j, F_j\in\mathcal{S}\}\].
  We will prove that: \[
    \left\{
      \begin{array}{l}
        \mbox{(1) $\mathcal{B}$ is an algebra} \\
        \mbox{(2) }\mathcal{S} \subseteq \mathcal{B}
      \end{array}
    \right. \Rightarrow \mathcal{A} \subseteq \mathcal{B}
  \]
  Note that (2) is trivial since every element in $\mathcal{S}$ could be written in the form of $\sum_{j=1}^n F_j$.
  
  Now we prove the (1).
  \begin{enumerate}
  \item $\Omega\in\mathcal{B}$. (this is trivial, since $\Omega\in\mathcal{S}\subseteq\mathcal{B}$).
  \item $A,B\in\mathcal{B}\Rightarrow A\cap B\in\mathcal{B}$. (easy to verify)
  \item $A\in\mathcal{B}\Rightarrow A^c\in\mathcal{B}$. \\
    Let $A\in\mathcal{B}$ and $A = \sum_{j=1}^n E_j, E_j\in\mathcal{S}$.
    Then $A^c = (\sum_{j=1}^n E_j)^c = E_1^c \cap E_2^c \cap \cdots \cap E_n^c$.
    And $E_i^c = \sum_{k_i = 1}^{l_i} F_{i, k_i}, F_{i, j} \in \mathcal{S}$ (This is by the definition of semi-algebra).
    So,
    \begin{align*}
    A^c &= (\sum_{k_1=1}^{l_1}F_{1,k_1})\cap(\sum_{k_2=1}^{l_2}F_{2,k_2})\cap\cdots\cap(\sum_{k_n=1}^{l_n}F_{n,k_n}) \\
    &= \sum_{k_1=1}^{l_1}\sum_{k_2=1}^{l_2}\cdots\sum_{k_n=1}^{l_n} (F_{1,k_1}\cap F_{2,k_2}\cap\cdots\cap F_{n,k_n})
    \end{align*}
    So, $A^c\in\mathcal{B}$.
  \end{enumerate}
  Since $\mathcal{B}$ is an algebra, and it contains $\mathcal{S}$, so $\mathcal{A}\subseteq{S}$.
\end{proof}

Now we investigate functions defined on these sets.
\begin{define}[addtive function]
  \index{additive function}
  For some $\mathcal{C}\subseteq\mathcal{P}(\Omega)$ where $\emptyset\in\mathcal{C}$. we could define a function $\mu:\mathcal{C}\to\mathbb{R}_+\cup\{+\infty\}$.
  Then $\mu$ is additive when:
  \begin{enumerate}
  \item $\mu(\emptyset) = 0$.
  \item $\substack{E_1, E_2, \cdots E_n\in\mathcal{C}\\E = \sum_{j=1}^n E_jj} \Rightarrow \mu(E) = \sum_{j = 1}^n \mu(E_j)$ \\
    (defined on finite disjoint union) \\
  \end{enumerate}
\end{define}

\begin{observation}
  If $E\subseteq F\subseteq \mathcal{C}C$ and $\mu:\mathcal{C} \to \mathbb{R}_+\cup \{+\infty\}$ is addtive, then
  \begin{enumerate}
  \item $\mu(E) = +\infty \Rightarrow \mu(F) = +\infty$.
  \item $\mu(E) < +\infty \Rightarrow \mu(F) = \mu(E) + \mu(F\setminus E)$.
  \end{enumerate}
  So, $\mu(E)\leq \mu(F)$ and when $\mu(E)$ is finite, then
  \[\mu(F\setminus E) = \mu(F) - \mu(E)\]
\end{observation}

\begin{define}[$\sigma$-addtive function]
  \index{$\sigma$-addtive function}
  For some $\mathcal{C}\subseteq\mathcal{P}(\Omega)$, where $\emptyset\in\mathcal{C}$. We could define a function $\mu:\mathcal{C}\to\mathbb{R}_+\cup\{+\infty\}$.
  Then $\mu$ is $\sigma$-additive when:
  \begin{enumerate}
  \item $\mu(\emptyset) = 0$.
  \item $\substack{E_j\in\mathcal{C}, j\geq 1\\E_j \cap E_k = \emptyset \\ E = \sum_{j\geq 1} E_j} \Rightarrow \mu(E) = \sum_{j\geq 1}\mu(E_j)$ \\
    (defined on countable disjoint union)
  \end{enumerate}
\end{define}

From the definition above, we could define what is a continuous measure (note that a measure is actually a function)

\begin{define}[continuous from below]
  \index{continuous from below}
  Suppose we have
  \[\substack{
      \mathcal{C}\subseteq\mathcal{P}(\Omega) \\
      \mu:\mathcal{C} \to \mathbb{R}_+\cup\{+\infty\} \\
      E\in\mathcal{C}
  }\],
  then when we say $\mu$ continuous from below at $E$, we means:
  \[\forall \{E_n\}_{n\geq 1} , \substack{E_n\in\mathcal{C}\\E_n\uparrow E} \Rightarrow \mu(E_n)\uparrow\mu(E)\] 
\end{define}
  \marginnote[-1cm]{
    Here, when we say $E_n\uparrow E$, we means $E_n\subseteq E_{n+1}$ and $\bigcup_{n\geq 1} E_n = E$.
  }

\begin{define}[continuous from above]
  Suppose we have
  \[\substack{
      \mathcal{C} \subseteq \mathcal{P}(\Omega) \\
      \mu : \mathcal{C} \to \mathbb{R}_+ \cup \{+\infty\} \\
      E \in \mathcal{C} 
    }\],
  then when we say $\mu$ continuous from above at $E$, we means:
  \[
    \forall \{E_n\}_{n\geq 1} , \substack{E_n \in\mathcal{C} \\ E_n\downarrow E \\ \exists n_0, \mu(n_0) \leq +\infty} \Rightarrow \mu(E_n)\downarrow\mu(E)
  \] \\
  (Note that here, there must exists some $n_0$ for $\mu(n_0)\leq +\infty$.
  In my mind, this is because, $+\infty$ is outside of $\mathbb{R}$, but we could only perform analysis inside $\mathbb{R}$.
  So, we need some tools to pull us back.)
\end{define}
\marginnote[-3cm]{
Here, when we say $E_n\downarrow E$, we means $E_n\supseteq E_{n+1}$ and $\bigcap_{n\geq 1} E_n = E$.
}

\begin{define}
  If a function continuous from both above and below, then we call it continuous.
\end{define}

\begin{lemma}
  Suppose we have \[\substack{
    \mathcal{A}\subseteq \mathcal{P}(\Omega), \mbox{ algebra} \\
    \mu : \mathcal{A} \to \mathbb{R}_+\cup\{+\infty\}, \mbox{ additive}
  }\],
  then we have:
  \begin{enumerate}
  \item $\mu$ is $\sigma$-additive $\Rightarrow$ $\mu$ is continuous at $E, \forall E\in \mathcal{A}$.
  \item $\mu$ is continuous from below $\Rightarrow$ $\mu$ is $\sigma$-additive.
  \item $\substack{
      \mbox{$\mu$ is continuous from above at $\emptyset$}\\
      \mbox{$\mu$ is finite ($\mu(\Omega) < +\infty$)}
    } \Rightarrow \mbox{$\mu$ is $\sigma$-additive.}$
  \end{enumerate}
\end{lemma}

\begin{proof}
  \emph{(1) ($\mu$ is $\sigma$-additive $\Rightarrow$ $\mu$ is continuous at $E, \forall E\in\mathcal{A}$ from \underline{below}):} \\
  Since we have $E_n\uparrow E$, then we define $F_n = E_n\setminus E_{n-1}$.
  Its clear that $F_k$ is disjoint and $\cup_{n\geq 1} E_n = \cup_{k\geq 1} F_k, \sum_{k=1}^n F_k = E_n$.
  So, using the $\sigma$-additive of $\mu$, we have
  \begin{align*}
    \mu(E) = \sum_{k\geq 1} F_k &= \lim_{n\to\infty} \sum_{k=1}^n \mu(F_k) \\
    &= \lim_n \mu(\sum_{k=1}^n F_k), \hspace{0.5cm} \mbox{using additive} \\
    &= \lim \mu(E_n)
  \end{align*}

  \emph{(1) ($\mu$ is $\sigma$-additive $\Rightarrow$ $\mu$ is continuous at $E, \forall E\in\mathcal{A}$ from \underline{above}):} \\
  Recall what we have: \[
    \left\{
      \begin{array}{l}
        E\in\mathcal{A} \\
        E_n \in \mathcal{A}, E_n\downarrow E \\
        \mu(E_{n_0}) < +\infty
      \end{array}
    \right.
  \]
  Lets define $G_k = E_{n_0}\setminus E_{n_0+k}$.
  Since $E_{n_0+1}, E_{n_0+2}, \cdots$ is getting smaller and smalller,
  we have $G_k\uparrow E_{n_0} \setminus E$.
  Thus, as we have proved above, $\mu(G_k)\uparrow \mu(E_{n_0}\setminus E)$.
  Since $\mu(E_{n_0})$ is finite, we have 
  \begin{align*}
    \mu(E_{n_0}\setminus E) &= \lim_k \mu(E_{n_0}\setminus E_{n_0+k}) \\
    \mu(E_{n_0}) - \mu(E) &= \lim_k (\mu(E_{n_0}) - \mu(E_{n_0+k})) \\
    \mu(E) &= \lim_k E_{n_0+k} \\
    \mu(E) &= \lim E_n
  \end{align*}

  \emph{(2) ($\mu$ is continuous from below $\Rightarrow$ $\mu$ is $\sigma$-additive):} \\
  Suppose we have $E = \sum_{k\geq 1} E_k (E, E_k\in\mathcal{A})$, we want to prove that $\mu(E) = \sum_{k\geq 1} \mu(E_k)$.
  \marginnote[-1cm]{
    Here, the notation \[\sum_{k\geq 1}\] could be seen as an abbreviate of
    \[\lim_{n\to+\infty}\sum_{k=1}^n\]
    (this is sometimes called infinity sum)
  }
  Let $F_n = \sum_{k=1}^n E_k$, then $F_n \uparrow E$.
  Since $\mu$ is continuous from below, we have $\mu(F_n)\uparrow \mu(E)$.
  And by additive $\sum_{k=1}^n \mu(E_k) \uparrow \mu(E)$.
  If we write this limit in infinity sum form we have
  \begin{align*}
    \lim_{n\to+\infty}\sum_{k=1}^n \mu(E_k) = \mu(E) \\
    \lim_{k\geq 1} \mu(E_k) = \mu(E)
  \end{align*}

  \emph{(3) (
    $\substack{
      \mbox{$\mu$ is continuous from above at $\emptyset$} \\
      \mbox{$\mu$ is finite, which means $\mu(\Omega)<+\infty$}
    } \Rightarrow \mbox{$\mu$ is $\sigma$-additive}
  $):} \\
  Suppose we have $E = \sum_{k\geq 1} E_k, (E, E_k\in\mathcal{A})$, we want to show that $\mu(E) = \sum_{k\geq 1} \mu(E_k)$.
  Let $F_n = \sum_{k\geq n} E_k = E\setminus \sum_{j=1}^{n-1}E_j \in \mathcal{A}$.
  So we have $F_n\downarrow \emptyset$.

  Since we have $\substack{F_n\in\mathcal{A} \\ F_n\downarrow\emptyset\\ \mu(F_n)<+\infty}$, by the definition of continuous from above at $\emptyset$,
  we have $\mu(F_n)\downarrow \mu(\emptyset) = 0$.
  Thus, we have
  \begin{align*}
    \mu(E) &= \mu(\sum_{k=1}^{n} E_k \cup \sum_{k > n} E_k) \\
    &= \sum_{k=1}^n \mu(E_k) + \mu(F_{n+1}), \hspace{0.5cm} \mbox{by additive} \\
    &= \lim_{n\to+\infty} (\sum_{k=1}^n \mu(E_k) + \mu(F_{n+1})), \hspace{0.5cm} \mbox{since the above equality holds for all $n$} \\
    &= \lim_{n\to+\infty} (\sum_{k=1}^n \mu(E_k)) + 0 \\
    &= \sum_{k\geq 1} \mu(E_k)
  \end{align*}
\end{proof}

Having these definitions and properties, we could start our extension.
\begin{theorem}[extension semi-algebra to algebra with ($\sigma$)-additive reserved]
  If $\mathcal{S}\subseteq\mathcal{P}(\Omega)$\\
  $\mu : \mathcal{S} \to \mathbb{R}_+\cup\{+\infty\}$, ($\sigma$)-additive.

  Then $\exists \nu : \mathcal{A}(\mathcal{S}) \to \mathbb{R}_+\cup\{+\infty\}$, such that:
  \begin{enumerate}
  \item $\nu$ is ($\sigma$)-additive.
  \item $\nu(A) = \mu(A), \forall A\in\mathcal{S}$.
  \item $\nu$ is unique. Which means:\\
  \( \substack{
      \nu_1, \nu_2 : \mathcal{A}(\mathcal{S}) \to \mathbb{R}_+\cup\{+\infty\} \\
      \nu_1(A) = \nu_2(A), \forall A\in \mathcal{S}
  } \Rightarrow \nu_1(E) = \nu_2(E), \forall E\in\mathcal{A}(\mathcal{S})\)
  \end{enumerate}
\end{theorem}
\begin{proof}
  Since any $A\in\mathcal{A}(\mathcal{S})$ could be written in the form
  \[A = \sum_{j=1}^n E_j, \hspace{0.5cm} E_j\in\mathcal{S}\]
  (note that this property only exists between semi-algebra and algebra)
  So, to remains additive, we could define
  \[\nu(A) \overset{add}{=\joinrel=} \sum_{j=1}^n \nu(E_j) \overset{ext}{=\joinrel=} \sum_{j=1}^n \mu(E_j)\]

  \emph{(1) ($\nu$ is well-defined):} \\
  Suppose $A$ has two representations
  \[A = \sum_{j=1}^n E_j = \sum_{k=1}^m F_k, \hspace{0.5cm} E_j, F_k \in\mathcal{S}\]
  By additive, we have
  \begin{align*}
    \sum_{j=1}^n E_j &= \sum_{j=1}^n E_j \cap A \\
    &= \sum_{j=1}^n E_j \cap (\sum_{k=1}^m F_k) \\
    &= \sum_{j=1}^n\sum_{k=1}^m E_j\cap F_k \\
    \sum_{j=1}^n \mu(E_j) &= \sum_{j=1}^n\sum_{k=1}^m \mu(E_j\cap F_k)
  \end{align*}
  Similarly,
  \[\sum_{k=1}^m \mu(F_k) = \sum_{j=1}^n\sum_{k=1}^m \mu(E_j\cap F_k)\]

  \emph{(2) ($\nu$ is additive):} \\
  Say we have \[
    \begin{array}{l}
      A = \sum_{j=1}^n E_j, \hspace{0.5cm} E_j\in \mathcal{S} \\
      B = \sum_{k=1}^m F_k, \hspace{0.5cm} F_k\in \mathcal{S}
    \end{array}
  \]
  and $A\cap B = \emptyset$.
  We want to prove that $\nu(A\cup B) = \nu(A) + \nu(B)$.
  Trivially,
  \begin{align*}
    A \cup B &= \sum_{j=1}^n E_j + \sum_{k=1}^m F_k \\
    \nu(A\cup B) &= \nu(\sum_{j=1}^nE_j) + \nu(\sum_{k=1}^m F_k) \\
    &= \nu(A) + \nu(B)
  \end{align*}
  
  \emph{(3) ($\nu$ is unique):}
  Recall that we have: \[
    \left\{
      \begin{array}{l}
        \nu_1, \nu_2 : \mathcal{A}(\mathcal{S}) \to \mathbb{R}_+\{+\infty\} \\
        \nu_1(A) = \nu_2(A), \hspace{0.5cm} \forall A \in \mathcal{S} \\
        \nu_1, \nu_2 \mbox{ additive}
      \end{array}
    \right.
  \], and we want to prove $\nu_1(B) = \nu_2(B)$ forall $B\in\mathcal{A}(\mathcal{S})$.
  But this is actually very trivial, since for all $B\in\mathcal{A}(\mathcal{S})$ we have $B = \sum_{j=1}^n E_j$ for some $E_j\in\mathcal{S}$.
  \[\nu_1(B) \overset{add}{=\joinrel=} \sum_{j=1}^n \mu(E_j) \overset{add}{=\joinrel=} \nu_2(B)\]
\end{proof}

%%% Local Variables:
%%% mode: latex
%%% TeX-master: "../../notebook"
%%% End:
