\section{Basic Notations and Ideas}

\subsection{Non-measurealbe Sets}
Measure is very important since the integrating and probability are built on top of it.
At the begining, the idea is very simple:
\paragraph{}\emph{
  The concept for the measure of length grows up naturally.
  For a set $(a, b]\in \mathbb{R}$, the length of it is $b - a$.
  As the interval $(a, b]$ could be seen as a subset of $\mathbb{R}$,
  we might ask, could we define a function $f:\mathcal{P}(\mathbb{R}) \to \mathbb{R}$
  to give a measure for all the subset of $\mathbb{R}$?
} 

\paragraph{}
At first, some people start to find such a function $f$ on $\mathcal{P}(\mathbb{R})$.
Ideally, if we want to define a function $\lambda$ like this, we'd like it satisfies some properties:
\begin{itemize}[itemsep=0mm]
\item (0) $\lambda: \mathcal{P}(\mathbb{R}) \to \mathbb{R}_{+}\cup\{+\infty\}$
\item (1) $\lambda((a, b]) = b - a$, since we want it to remain its nature meaning.
\item (2) $\lambda(A + x) = \lambda(A)$, since the shift should not change the volume.
\item (3) $\lambda(\cup_{j\geq 1} A_j) = \sum_{j\geq 1} \lambda(A_j)$ for coutable  disjoint $A_j$. Since we need to add the volume for calculation.
\end{itemize}

But soon, they find it impossible, as we could prove this function could not exists.
Otherwise, there must be a contradiction.
\begin{theorem}
  The function that satisfies the requirements above does not exist.
\end{theorem}
\begin{proof} 
  For $x, y\in\mathbb{R}$, let $x\sim y$ if $y - x\in\mathbb{Q}$.
  Let $[x] = \{y \in \mathbb{R} | y - x \in \mathbb{Q}\}$ be a equivalent class.
  Let $\Lambda = \mathbb{R}|\sim$, which means the set of equivalent classes.
  \marginnote{$\Lambda$ is uncountable, since $\mathbb{R}$ is uncountable, and any $[x]\in\Lambda$ is countable. We may use $\alpha, \beta$ to represent the point in $\Lambda$.}
  Now, we construct a set $\Omega\in\mathbb{R}$ be the set which contains one and only one points from each equivalent class. Thus we could assume that $\Omega\subseteq(0, 1)$.

\begin{claim}[For a set $\Omega\subseteq\mathbb{R}$, and $p, q\in\mathbb{R}$. Either we have $\Omega+p = \Omega+q$, eigher we have $\Omega+p$ disjoints with $\Omega+q$]
  \begin{proof}
    First, lets assume $(\Omega + p)\cap(\Omega+q)\not=\emptyset$.
    Then for some $x$ in it, we have
    \begin{align*}
      x &= \alpha + p, \hspace{0.5cm} \alpha\in\Omega \\
        &= \beta + q,  \hspace{0.5cm} \beta \in\Omega
    \end{align*}
    So we have $\alpha - \beta = q - p$.
    From the definition of $\Omega$, we know that $p = q$ and $\alpha = \beta$.
    So we have \[q\not= p \Rightarrow (\Omega + q)\cap(\Omega+p) = \emptyset\].
  \end{proof}
\end{claim}
  By this claim, we could define a set:
  \[ S = \sum_{\substack{q\in\mathbb{Q}\\-1 < q < 1}}(\Omega + q) \]
  Its easy to see that for all these $q$, $\Omega + q \subseteq (-1, 2)$.
  So we have $S\subseteq(-1, 2)$ and $\lambda(S) \leq \lambda((-1, 2)) = 3$.
  So by property (3), we have
  \[\sum_{\substack{q\in\mathbb{Q}\\-1<q<1}} \lambda(\Omega + q) \leq 3\]
  And by property (2), we have $\lambda(\Omega + q) = \lambda(\Omega) = 0$.
  \marginnote{
    Here, since we have infinity $q$s, $\lambda(\Omega + q) = 0$. Otherwise, the sum of them could not less than $3$.
  }
  By observing this, we have $\lambda(S) = 0$.
  \begin{claim}[$(0, 1) \subseteq S$]
    \begin{proof}
      Suppose we have $x\in(0,1)$ and $\alpha\in [x]\cap\Omega$, then we have $\alpha\in(0, 1)$ (by definition of $\Omega$). Since $\alpha$ belongs to the equivalent class of $[x]$, $\alpha - x = q\in\mathbb{Q}$. Its easy to see that $-1 < q < 1$. Which means $x=\alpha + q$ and thus $x\in\Omega + q$ for some $q\in(0,1)$. Which is exactly in $S$. So we have $(0, 1)\subseteq S$.
    \end{proof}
  \end{claim}
  So far, we have find a contradiction that there is no such function which could satisfies all the properties.
\end{proof}

What can we do now? Maybe we could relax some of these properties.
Maybe we could remove some this conditions.
Maybe we could accept the function which is not defined on all the subsets of $\mathbb{R}$. This means there are some kinds of subsets of $\mathcal{P}$ which we could not set a measure on it, these sets are called non-measurealbe sets.

\subsection{Classes of subsets, and set functions}
\begin{define}[semi-algebra]
  \index{semi-algebra}
  $\mathcal{S} \subseteq \mathcal{P}(\Omega)$, is a semi-algebra if:
  \begin{enumerate}[itemsep=0mm]
  \item $\Omega \in \mathcal{S}$.
  \item $A, B\in \mathcal{S} \Rightarrow A\cap B\in \mathcal{S}$. \\ (\emph{This means it is closed under finite intersections})
  \item $\forall A\in\mathcal{S} \Rightarrow$ $\exists E_1, E_2, \cdots, E_n\in\mathcal{S}$ such that $A^c = \cup_{j=1}^n E_j$.
  \end{enumerate}
\end{define}

\begin{define}[algebra]
  \index{algebra (in measure theory)}
  $\mathcal{A} \subseteq \mathcal{P}(\Omega)$, is an algebra if:
  \begin{enumerate}[itemsep=0mm]
  \item $\Omega \in \mathcal{A}$.
  \item $A, B\in \mathcal{A} \Rightarrow A\cap B\in \mathcal{A}$.
  \item $A\in\mathcal{A} \Rightarrow A^c\in\mathcal{A}$. \\ (\emph{Here is the difference between semi-algebra and algebra})
  \end{enumerate}
\end{define}

\begin{define}[$\sigma$-algebra]
  \index{$\sigma$-algebra}
  $\mathcal{F} \subseteq \mathcal{P}(\Omega)$, is an algebra if:
  \begin{enumerate}[itemsep=0mm]
  \item $\Omega\in\mathcal{F}$.
  \item $A_j\in\mathcal{F}$ for $j\geq 1$ $\Rightarrow$ $\cap_{j\geq 1} A_j \in \mathcal{F}$. \\ (\emph{closed under countable intersections})
  \end{enumerate}
\end{define}

\begin{remark}
  $\sigma$-algebra $\Rightarrow$ algebra $\Rightarrow$ seim-algebra.
  $\sigma$-algebra asks for more constraint than algebra, and thus it is bigger than an algebra (this means you could find some algebra in $\sigma$-algebra or $\sigma$-algebra is itself an algebra).
\end{remark}

\begin{observation}
  If $\mathcal{A}_{\alpha} \subseteq \mathcal{P}(\Omega)$ is ($\sigma$)-algebra for some $\alpha\in I$. Then we have $\mathcal{A} = \cap_{\alpha\in I} \mathcal{A}_\alpha$ is a ($\sigma$)-algebra.
\end{observation}
we can use these fact to introduce the notion of the algebra generated by the classes of sets.
\begin{define}
  The algebra generated by class $\mathcal{C}\subseteq\mathcal{P}(\Omega)$ is denoted by $\mathcal{A}(\mathcal{C})$. And $\mathcal{A}(\mathcal{C})$ satisfies:
  \begin{itemize}
  \item $\mathcal{C}\subseteq\mathcal{A}(\mathcal{C})$, $\mathcal{A}(\mathcal{C})$ is algebra.
  \item \[
      \left\{
        \begin{array}{l}
          \mathcal{C} \subseteq \mathcal{B} \\
          \mathcal{B} \mbox{ is algebra} 
        \end{array}
      \right. \Rightarrow \mathcal{A}(\mathcal{C})\subseteq \mathcal{B}
    \]
  \end{itemize}
\end{define}
Let $\mathcal{A}_\alpha$ be all the algebra that contains $\mathcal{C}$.
Then it is easy to verify that
\[\mathcal{A}(\mathcal{C}) = \bigcap_\alpha \mathcal{A}_\alpha\].

\begin{lemma}
  $\mathcal{S}\subseteq\mathcal{P}(\Omega)$ is a semi-algebra, $\mathcal{A}$ is the algebra generated by $\mathcal{S}$. Then
  \[A\in\mathcal{A} \Leftrightarrow \substack{\exists E_j, i\leq j\leq n, E_j \in \mathcal{S}\\A=\sum_{j=1}^n E_j}\]
\end{lemma}
\begin{proof}
  \emph{($\Leftarrow$):} This is obvious from the definition of $\mathcal{A}$.

  \emph{($\Rightarrow$):} Define a new class
  \[\mathcal{B} = \{\sum_{j=1}^n F_j, F_j\in\mathcal{S}\}\].
  We will prove that: \[
    \left\{
      \begin{array}{l}
        \mbox{(1) $\mathcal{B}$ is an algebra} \\
        \mbox{(2) }\mathcal{S} \subseteq \mathcal{B}
      \end{array}
    \right. \Rightarrow \mathcal{A} \subseteq \mathcal{B}
  \]
  Note that (2) is trivial since every element in $\mathcal{S}$ could be written in the form of $\sum_{j=1}^n F_j$.
  
  Now we prove the (1).
  \begin{enumerate}
  \item $\Omega\in\mathcal{B}$. (this is trivial, since $\Omega\in\mathcal{S}\subseteq\mathcal{B}$).
  \item $A,B\in\mathcal{B}\Rightarrow A\cap B\in\mathcal{B}$. (easy to verify)
  \item $A\in\mathcal{B}\Rightarrow A^c\in\mathcal{B}$. \\
    Let $A\in\mathcal{B}$ and $A = \sum_{j=1}^n E_j, E_j\in\mathcal{S}$.
    Then $A^c = (\sum_{j=1}^n E_j)^c = E_1^c \cap E_2^c \cap \cdots \cap E_n^c$.
    And $E_i^c = \sum_{k_i = 1}^{l_i} F_{i, k_i}, F_{i, j} \in \mathcal{S}$ (This is by the definition of semi-algebra).
    So,
    \begin{align*}
    A^c &= (\sum_{k_1=1}^{l_1}F_{1,k_1})\cap(\sum_{k_2=1}^{l_2}F_{2,k_2})\cap\cdots\cap(\sum_{k_n=1}^{l_n}F_{n,k_n}) \\
    &= \sum_{k_1=1}^{l_1}\sum_{k_2=1}^{l_2}\cdots\sum_{k_n=1}^{l_n} (F_{1,k_1}\cap F_{2,k_2}\cap\cdots\cap F_{n,k_n})
    \end{align*}
    So, $A^c\in\mathcal{B}$.
  \end{enumerate}
  Since $\mathcal{B}$ is an algebra, and it contains $\mathcal{S}$, so $\mathcal{A}\subseteq{S}$.
\end{proof}

Now we investigate functions defined on these sets.
\begin{define}[addtive function]
  \index{additive function}
  For some $\mathcal{C}\subseteq\mathcal{P}(\Omega)$ where $\emptyset\in\mathcal{C}$. we could define a function $\mu:\mathcal{C}\to\mathbb{R}_+\cup\{+\infty\}$.
  Then $\mu$ is additive when:
  \begin{enumerate}
  \item $\mu(\emptyset) = 0$.
  \item $\substack{E_1, E_2, \cdots E_n\in\mathcal{C}\\E = \sum_{j=1}^n E_jj} \Rightarrow \mu(E) = \sum_{j = 1}^n \mu(E_j)$ \\
    (defined on finite disjoint union) \\
  \end{enumerate}
\end{define}

\begin{observation}
  If $E\subseteq F\subseteq \mathcal{C}C$ and $\mu:\mathcal{C} \to \mathbb{R}_+\cup \{+\infty\}$ is addtive, then
  \begin{enumerate}
  \item $\mu(E) = +\infty \Rightarrow \mu(F) = +\infty$.
  \item $\mu(E) < +\infty \Rightarrow \mu(F) = \mu(E) + \mu(F\setminus E)$.
  \end{enumerate}
  So, $\mu(E)\leq \mu(F)$ and when $\mu(E)$ is finite, then
  \[\mu(F\setminus E) = \mu(F) - \mu(E)\]
\end{observation}

\begin{define}[$\sigma$-addtive function]
  \index{$\sigma$-addtive function}
  For some $\mathcal{C}\subseteq\mathcal{P}(\Omega)$, where $\emptyset\in\mathcal{C}$. We could define a function $\mu:\mathcal{C}\to\mathbb{R}_+\cup\{+\infty\}$.
  Then $\mu$ is $\sigma$-additive when:
  \begin{enumerate}
  \item $\mu(\emptyset) = 0$.
  \item $\substack{E_j\in\mathcal{C}, j\geq 1\\E_j \cap E_k = \emptyset \\ E = \sum_{j\geq 1} E_j} \Rightarrow \mu(E) = \sum_{j\geq 1}\mu(E_j)$ \\
    (defined on countable disjoint union)
  \end{enumerate}
\end{define}

From the definition above, we could define what is a continuous measure (note that a measure is actually a function)

\begin{define}[continuous from below]
  \index{continuous from below}
  Suppose we have
  \[\substack{
      \mathcal{C}\subseteq\mathcal{P}(\Omega) \\
      \mu:\mathcal{C} \to \mathbb{R}_+\cup\{+\infty\} \\
      E\in\mathcal{C}
  }\],
  then when we say $\mu$ continuous from below at $E$, we means:
  \[\forall \{E_n\}_{n\geq 1} , \substack{E_n\in\mathcal{C}\\E_n\uparrow E} \Rightarrow \mu(E_n)\uparrow\mu(E)\] 
\end{define}
  \marginnote[-1cm]{
    Here, when we say $E_n\uparrow E$, we means $E_n\subseteq E_{n+1}$ and $\bigcup_{n\geq 1} E_n = E$.
  }

\begin{define}[continuous from above]
  Suppose we have
  \[\substack{
      \mathcal{C} \subseteq \mathcal{P}(\Omega) \\
      \mu : \mathcal{C} \to \mathbb{R}_+ \cup \{+\infty\} \\
      E \in \mathcal{C} 
    }\],
  then when we say $\mu$ continuous from above at $E$, we means:
  \[
    \forall \{E_n\}_{n\geq 1} , \substack{E_n \in\mathcal{C} \\ E_n\downarrow E \\ \exists n_0, \mu(n_0) \leq +\infty} \Rightarrow \mu(E_n)\downarrow\mu(E)
  \] \\
  (Note that here, there must exists some $n_0$ for $\mu(n_0)\leq +\infty$.
  In my mind, this is because, $+\infty$ is outside of $\mathbb{R}$, but we could only perform analysis inside $\mathbb{R}$.
  So, we need some tools to pull us back.)
\end{define}
\marginnote[-3cm]{
Here, when we say $E_n\downarrow E$, we means $E_n\supseteq E_{n+1}$ and $\bigcap_{n\geq 1} E_n = E$.
}

\begin{define}
  If a function continuous from both above and below, then we call it continuous.
\end{define}

\begin{lemma}
  Suppose we have \[\substack{
    \mathcal{A}\subseteq \mathcal{P}(\Omega), \mbox{ algebra} \\
    \mu : \mathcal{A} \to \mathbb{R}_+\cup\{+\infty\}, \mbox{ additive}
  }\],
  then we have:
  \begin{enumerate}
  \item $\mu$ is $\sigma$-additive $\Rightarrow$ $\mu$ is continuous at $E, \forall E\in \mathcal{A}$.
  \item $\mu$ is continuous from below $\Rightarrow$ $\mu$ is $\sigma$-additive.
  \item $\substack{
      \mbox{$\mu$ is continuous from above at $\emptyset$}\\
      \mbox{$\mu$ is finite ($\mu(\Omega) < +\infty$)}
    } \Rightarrow \mbox{$\mu$ is $\sigma$-additive.}$
  \end{enumerate}
\end{lemma}

\begin{proof}
  \emph{(1) ($\mu$ is $\sigma$-additive $\Rightarrow$ $\mu$ is continuous at $E, \forall E\in\mathcal{A}$ from \underline{below}):} \\
  Since we have $E_n\uparrow E$, then we define $F_n = E_n\setminus E_{n-1}$.
  Its clear that $F_k$ is disjoint and $\cup_{n\geq 1} E_n = \cup_{k\geq 1} F_k, \sum_{k=1}^n F_k = E_n$.
  So, using the $\sigma$-additive of $\mu$, we have
  \begin{align*}
    \mu(E) = \sum_{k\geq 1} F_k &= \lim_{n\to\infty} \sum_{k=1}^n \mu(F_k) \\
    &= \lim_n \mu(\sum_{k=1}^n F_k), \hspace{0.5cm} \mbox{using additive} \\
    &= \lim \mu(E_n)
  \end{align*}

  \emph{(1) ($\mu$ is $\sigma$-additive $\Rightarrow$ $\mu$ is continuous at $E, \forall E\in\mathcal{A}$ from \underline{above}):} \\
  Recall what we have: \[
    \left\{
      \begin{array}{l}
        E\in\mathcal{A} \\
        E_n \in \mathcal{A}, E_n\downarrow E \\
        \mu(E_{n_0}) < +\infty
      \end{array}
    \right.
  \]
  Lets define $G_k = E_{n_0}\setminus E_{n_0+k}$.
  Since $E_{n_0+1}, E_{n_0+2}, \cdots$ is getting smaller and smalller,
  we have $G_k\uparrow E_{n_0} \setminus E$.
  Thus, as we have proved above, $\mu(G_k)\uparrow \mu(E_{n_0}\setminus E)$.
  Since $\mu(E_{n_0})$ is finite, we have 
  \begin{align*}
    \mu(E_{n_0}\setminus E) &= \lim_k \mu(E_{n_0}\setminus E_{n_0+k}) \\
    \mu(E_{n_0}) - \mu(E) &= \lim_k (\mu(E_{n_0}) - \mu(E_{n_0+k})) \\
    \mu(E) &= \lim_k E_{n_0+k} \\
    \mu(E) &= \lim E_n
  \end{align*}

  \emph{(2) ($\mu$ is continuous from below $\Rightarrow$ $\mu$ is $\sigma$-additive):} \\
  Suppose we have $E = \sum_{k\geq 1} E_k (E, E_k\in\mathcal{A})$, we want to prove that $\mu(E) = \sum_{k\geq 1} \mu(E_k)$.
  \marginnote[-1cm]{
    Here, the notation \[\sum_{k\geq 1}\] could be seen as an abbreviate of
    \[\lim_{n\to+\infty}\sum_{k=1}^n\]
    (this is sometimes called infinity sum)
  }
  Let $F_n = \sum_{k=1}^n E_k$, then $F_n \uparrow E$.
  Since $\mu$ is continuous from below, we have $\mu(F_n)\uparrow \mu(E)$.
  And by additive $\sum_{k=1}^n \mu(E_k) \uparrow \mu(E)$.
  If we write this limit in infinity sum form we have
  \begin{align*}
    \lim_{n\to+\infty}\sum_{k=1}^n \mu(E_k) = \mu(E) \\
    \lim_{k\geq 1} \mu(E_k) = \mu(E)
  \end{align*}

  \emph{(3) (
    $\substack{
      \mbox{$\mu$ is continuous from above at $\emptyset$} \\
      \mbox{$\mu$ is finite, which means $\mu(\Omega)<+\infty$}
    } \Rightarrow \mbox{$\mu$ is $\sigma$-additive}
  $):} \\
  Suppose we have $E = \sum_{k\geq 1} E_k, (E, E_k\in\mathcal{A})$, we want to show that $\mu(E) = \sum_{k\geq 1} \mu(E_k)$.
  Let $F_n = \sum_{k\geq n} E_k = E\setminus \sum_{j=1}^{n-1}E_j \in \mathcal{A}$.
  So we have $F_n\downarrow \emptyset$.

  Since we have $\substack{F_n\in\mathcal{A} \\ F_n\downarrow\emptyset\\ \mu(F_n)<+\infty}$, by the definition of continuous from above at $\emptyset$,
  we have $\mu(F_n)\downarrow \mu(\emptyset) = 0$.
  Thus, we have
  \begin{align*}
    \mu(E) &= \mu(\sum_{k=1}^{n} E_k \cup \sum_{k > n} E_k) \\
    &= \sum_{k=1}^n \mu(E_k) + \mu(F_{n+1}), \hspace{0.5cm} \mbox{by additive} \\
    &= \lim_{n\to+\infty} (\sum_{k=1}^n \mu(E_k) + \mu(F_{n+1})), \hspace{0.5cm} \mbox{since the above equality holds for all $n$} \\
    &= \lim_{n\to+\infty} (\sum_{k=1}^n \mu(E_k)) + 0 \\
    &= \sum_{k\geq 1} \mu(E_k)
  \end{align*}
\end{proof}

Having these definitions and properties, we could start our extension.
\begin{theorem}[extension semi-algebra to algebra with ($\sigma$)-additive reserved]
  If $\mathcal{S}\subseteq\mathcal{P}(\Omega)$\\
  $\mu : \mathcal{S} \to \mathbb{R}_+\cup\{+\infty\}$, ($\sigma$)-additive.

  Then $\exists \nu : \mathcal{A}(\mathcal{S}) \to \mathbb{R}_+\cup\{+\infty\}$, such that:
  \begin{enumerate}
  \item $\nu$ is ($\sigma$)-additive.
  \item $\nu(A) = \mu(A), \forall A\in\mathcal{S}$.
  \item $\nu$ is unique. Which means:\\
  \( \substack{
      \nu_1, \nu_2 : \mathcal{A}(\mathcal{S}) \to \mathbb{R}_+\cup\{+\infty\} \\
      \nu_1(A) = \nu_2(A), \forall A\in \mathcal{S}
  } \Rightarrow \nu_1(E) = \nu_2(E), \forall E\in\mathcal{A}(\mathcal{S})\)
  \end{enumerate}
\end{theorem}
\begin{proof}
  Since any $A\in\mathcal{A}(\mathcal{S})$ could be written in the form
  \[A = \sum_{j=1}^n E_j, \hspace{0.5cm} E_j\in\mathcal{S}\]
  (note that this property only exists between semi-algebra and algebra)
  So, to remains additive, we could define
  \[\nu(A) \overset{add}{=\joinrel=} \sum_{j=1}^n \nu(E_j) \overset{ext}{=\joinrel=} \sum_{j=1}^n \mu(E_j)\]

  \emph{(1) ($\nu$ is well-defined):} \\
  Suppose $A$ has two representations
  \[A = \sum_{j=1}^n E_j = \sum_{k=1}^m F_k, \hspace{0.5cm} E_j, F_k \in\mathcal{S}\]
  By additive, we have
  \begin{align*}
    \sum_{j=1}^n E_j &= \sum_{j=1}^n E_j \cap A \\
    &= \sum_{j=1}^n E_j \cap (\sum_{k=1}^m F_k) \\
    &= \sum_{j=1}^n\sum_{k=1}^m E_j\cap F_k \\
    \sum_{j=1}^n \mu(E_j) &= \sum_{j=1}^n\sum_{k=1}^m \mu(E_j\cap F_k)
  \end{align*}
  Similarly,
  \[\sum_{k=1}^m \mu(F_k) = \sum_{j=1}^n\sum_{k=1}^m \mu(E_j\cap F_k)\]

  \emph{(2) ($\nu$ is additive):} \\
  Say we have \[
    \begin{array}{l}
      A = \sum_{j=1}^n E_j, \hspace{0.5cm} E_j\in \mathcal{S} \\
      B = \sum_{k=1}^m F_k, \hspace{0.5cm} F_k\in \mathcal{S}
    \end{array}
  \]
  and $A\cap B = \emptyset$.
  We want to prove that $\nu(A\cup B) = \nu(A) + \nu(B)$.
  Trivially,
  \begin{align*}
    A \cup B &= \sum_{j=1}^n E_j + \sum_{k=1}^m F_k \\
    \nu(A\cup B) &= \nu(\sum_{j=1}^nE_j) + \nu(\sum_{k=1}^m F_k) \\
    &= \nu(A) + \nu(B)
  \end{align*}
  
  \emph{(3) ($\nu$ is unique):}
  Recall that we have: \[
    \left\{
      \begin{array}{l}
        \nu_1, \nu_2 : \mathcal{A}(\mathcal{S}) \to \mathbb{R}_+\{+\infty\} \\
        \nu_1(A) = \nu_2(A), \hspace{0.5cm} \forall A \in \mathcal{S} \\
        \nu_1, \nu_2 \mbox{ additive}
      \end{array}
    \right.
  \], and we want to prove $\nu_1(B) = \nu_2(B)$ forall $B\in\mathcal{A}(\mathcal{S})$.
  But this is actually very trivial, since for all $B\in\mathcal{A}(\mathcal{S})$ we have $B = \sum_{j=1}^n E_j$ for some $E_j\in\mathcal{S}$.
  \[\nu_1(B) \overset{add}{=\joinrel=} \sum_{j=1}^n \mu(E_j) \overset{add}{=\joinrel=} \nu_2(B)\]
\end{proof}

\begin{define}[infimum]
  \index{infimum}
  \label{define:infimum}
  The infimum of a subset $S$ of a partially ordered set $T$ is the greatest element in $T$ that is less than or equal to all elements of $S$, if such an element exists.
\end{define}

\begin{define}[outer measure]
  \index{outer measure}
  For \[\substack{
    \mathcal{C}\subseteq\mathcal{P}(\Omega) \\
    \emptyset \in \mathcal{C} \\
    \mu : \mathcal{C} \to \mathbb{R}_+\cup\{+\infty\}
  }\], then $\mu$ is an outer measure if
  \begin{enumerate}
  \item $\mu(\emptyset) = 0$.
  \item $E\subseteq F$ where $E, F\in \mathcal{C}$ $\Rightarrow$ $\mu(E)\leq\mu(F)$ (monotone).
  \item $E, E_i\in\mathcal{C}$, $E = \cup E_i$ $\Rightarrow$ $\mu(E) \leq \sum\mu(E_i)$ (sub-additive).
  \end{enumerate}
\end{define}

Now, we are going to extend the function on algebra to $\sigma$-algebra.
It turns out that if a functoin $\nu : \mathcal{A} \to \mathbb{R}_+\cup\{+\infty\}$ is $\sigma$-additive and $\sigma$-finite in algebra $\mathcal{A}$.
Then we could extend it to a unique $\sigma$-additive function $\pi : \mathcal{F}(\mathcal{A}) \to \mathbb{R}_+\cup\{+\infty\}$.
($\mathcal{F}(\mathcal{A})$ means the $\sigma$-algebra generated by $\mathcal{A}$)

There is the big picture of what we need to do to build this extension.
\begin{enumerate}
\item define $\pi^*$ on top of an algebra $\mathcal{A}$ with a $\sigma$-additive function $\nu$ on it.
\item prove $\pi^*$ is an outer meansure.
\item construct a measurealbe set $\mathcal{M}$, and prove that $\mathcal{F}(\mathcal{A})\subseteq \mathcal{M}$.
\item prove $\pi|_{\mathcal{M}}$ is $\sigma$-additive.
\item prove this extension is unique.
\end{enumerate}
This is called caratheodory theorem.

First we are going to define the $\pi^*$ on $\mathcal{P}(\Omega) \to \mathbb{R}_+\cup\{+\infty\}$.
\begin{define}[caratheodory theorem (STEP 1)]
  Suppose \[\substack{
    \mathcal{A}\subseteq\mathcal{P}(\Omega),\hspace{0.5cm} \mbox{algebra} \\
    \nu : \mathcal{A} \to \mathbb{R}_+\cup\{+\infty\}, \hspace{0.5cm} \mbox{$\sigma$-additive} 
  }\]
  , then we define $\pi^*$ as
  \[\pi^*(A) = \inf_{\{E_i\}} \sum\nu(E_i)\]
  where $\{E_i\}$ enumerates all the countable sequence where
  $E_i\in\mathcal{A}$ and $A\subseteq\cup E_i$.
\end{define}

\begin{lemma}[$\pi^*$ is an outer measure]
  Which means we need to prove that
  \begin{enumerate}
  \item $\pi^*(\emptyset) = 0$
  \item $E\subseteq F$ $\Rightarrow$ $\pi^*(E)\leq\pi^*(F)$ (monotone).
  \item $E\subseteq \cup E_i$ $\Rightarrow$ $\pi^*(E)\leq \sum_{i\geq 1} \pi^*(E_i)$ (sub-additive)
  \end{enumerate}
\end{lemma}
\begin{proof}
  \emph{(1) ($\pi^*(\emptyset) = 0$):} \\
  Since \[
    \substack{\{E_i\} \\ E_i = \emptyset} \Rightarrow
    \pi^*(\emptyset) \leq \sum_{i\geq 1} \nu(E_i) = 0
  \]
  and \[
    \substack{\{E_i\} \\ E_i\in\mathcal{A} \\ \emptyset\subseteq \cup E_i}
    \Rightarrow \sum_{i\geq} \nu(E_i) \geq 0
  \]
  So, we have $\pi^*(\emptyset) = 0$.

  \emph{(2) ($E\subseteq F$ $\Rightarrow$ $\pi^*(E)\leq\pi^*(F)$):} \\
  By definition:
  \[
    \begin{array}{l}
      \pi^*(E) = \inf_A \sum_{i\geq 1} \nu(E_i),\hspace{0.5cm}\mbox{$A = \{\{E_i\}|E \subseteq \cup E_i\}$} \\
      \pi^*(F) = \inf_B \sum_{i\geq 1} \nu(F_i),\hspace{0.5cm}\mbox{$B = \{\{F_i\}|F\subseteq \cup F_i\}$}
    \end{array}
  \]
  And we could see that since $E\subseteq F$, then $B\subseteq A$.
  So the infimum in $A$ should less or equal than the infimum in $B$, which means
  \[\pi^*(E) \leq \pi^*(F)\]

  \emph{(3) ($E\subseteq \cup E_i$ $\Rightarrow$ $\pi^*(E)\leq \sum_{i\geq 1} \pi^*(E_i)$):} \\
  First, we assume that $\pi^*(E_i) < +\infty, \forall i$, since if $\pi^*(E_i) = +\infty$ for some $i$, then the above inequality satisfies immediately.
  Lets fix some $\epsilon > 0$. And since by definition
  \[\pi^*(E_i) = \inf_{\{H_k\}} \sum_{k\geq 1} \nu(H_k) < +\infty\]
  There exists $\{H_{i,k}\}$ where $\substack{H_{i,k}\subseteq \mathcal{A} \\ E_i\subseteq \cup_{k\geq 1} H_{i,k}}$ and
  \[\pi^*(E_i) \leq \sum_{k\geq 1} \nu(H_{i,k}) \leq \pi^*(E_i) + \epsilon/2^i\]
  \marginnote[-3cm]{
    \begin{tcolorbox}[colframe={blue!20}, colback={blue!10}]
    \emph{Q: why we need to use a $\epsilon$ statement here?} \\
    \tcblower
    $\pi^*(E)$ is the infimum of $\{\sum_{i\geq 1} E_i | \{E_i\}\}$, thus, we may have the condition where
    \[\pi^*(E) \not\in \{\sum_{i\geq 1} E_i | \{E_i\}\}\]
    \end{tcolorbox}
  }
  So
  \begin{align*}
    \pi^*(E) &\leq \sum_{i,k} \nu(H_{i,k}) \\
             &\leq \sum_{i\geq 1} (\pi^*(E_i) + \epsilon/2^i) \\
             &= \sum_{i} pi^*(E_i) + \epsilon 
  \end{align*}
  Since this is true for all $\epsilon$, we have $\pi^*(E) \leq \sum_{i\geq 1} \pi^*(E_i)$.

  Now we define the measurealbe subset $\mathcal{M}$.
\end{proof}

\begin{define}[measurealbe set $\mathcal{M}$]
  $A\in\mathcal{M}$ if $\forall E\subseteq\Omega, \pi^*(E) = \pi^*(E\cap A) + \pi^*(E\cap A^c)$
\end{define}
\begin{observation}
  For any $A\subseteq\Omega$ we have $pi^*(E) \leq \pi^*(E\cap A) + \pi^*(E\cap A^c)$. \\
  So, if we want to prove $\pi^*(E) = \pi^*(E\cap A) + \pi^*(E\cap A^c)$ later,
  we only need to shouw that $\pi^*(E) \geq \pi^*(E\cap A) + \pi^*(E\cap A^c)$.
\end{observation}
\begin{fact}
  $\mathcal{A}\subseteq \mathcal{M}$.
\end{fact}
\begin{proof}
  For all the set $A\in\mathcal{A}$, we need to prove that
  for all $E\subseteq\Omega$, $\pi^*(E) \geq \pi^*(E\cap A) + \pi^*(E\cap A^c)$.

  Lets assume $\pi^*(E) \leq +\infty$.
  Then, for some fixed $\epsilon > 0$, there exists $\substack{\{E_i\}\\E_i\in\mathcal{A}\\E=\cup_{i\geq 1} E_i}$,
  such that $\pi^*(E) \leq \sum_{i\geq 1} \nu(E_i) \leq \pi^*(E) + \epsilon$.

  And we have
  \begin{align*}
    E\cap A \subseteq \cup_{i\geq 1} E_i\cap A &\Rightarrow \pi^*(E\cap A) \leq \sum_{i\geq 1} \nu(E_i\cap A) \\
    E\cap A^c \subseteq \cup_{i\geq 1} E_i\cap A^c &\Rightarrow \pi^*(E\cap A^c) \leq \sum_{i\geq 1} \nu(E_i\cap A^c) \\
    \pi^*(E\cap A) + \pi^*(E\cap A^c) &\leq \sum_{i\geq 1} (\nu(E_i\cap A) + \nu(E_i\cap A^c)) \\
                                               &= \sum_{i\geq 1} \nu(E_i), \hspace{0.5cm} \mbox{by $\sigma$-additive} \\
                                               &\leq \pi^*(E) + \epsilon
  \end{align*}
  Since this is true for all positive $\epsilon$, we could send $\epsilon$ to $0$.
  Then we have $\pi^*(E\cap A) + \pi^*(E\cap A^c) \leq \pi^*(E)$.
  \marginnote[-1cm]{
    Actually, we could not send $\epsilon$ to $0$ directly.
    Here, what we actually do is to calculate a infimum on all possible $\{E_i\}$,
    and thus $\epsilon$ could reach $0$.
  }
\end{proof}
\begin{fact}
  $\mathcal{M}$ is a $\sigma$-algebra.
\end{fact}
\begin{proof}
  \emph{(1) ($\Omega\in\mathcal{M}$):}\\
  \begin{align*}
    \pi^*(E\cap\Omega) + \pi^*(E\cap \Omega^c) &= \pi^*(E) + \pi^*(\emptyset) \\
                                               &= \pi^*(E)
  \end{align*}
  
  \emph{(2) ($A\in\mathcal{M}\Rightarrow A^c\in\mathcal{M}$):}\\
  This is clear, because the definition of $\mathcal{M}$ is symmetric with complement.
  
  \emph{(3) ($\substack{\{A_j\}\\A_j\in\mathcal{M}}\Rightarrow \cup_{j\geq 1} A_j \in \mathcal{M}$):}\\
  \emph{(3.1) ($\mathcal{M}$ is closed under finite union):}\\
  For all $A, B\in\mathcal{M}$, we want to show that $A\cup B\in \mathcal{M}$.
  First note that
  \begin{align*}
    \pi^*(E\setminus A) &= \pi^*((E\setminus)\cap B) + \pi^*((E\setminus A)\setminus B) \\
             &= \pi^*((E\setminus A)\cap B) + \pi^*(E\setminus (A\cup B)) \\
    \pi^*(E) &= \pi^*(E\cap A) + \pi^*(E\setminus A) \\
    &= \pi^*(E\cap A) + \pi^*((E\setminus A)\cap B) + \pi^*(E\setminus(A\cap B))
  \end{align*}
  Since $(E\cap A) \cup ((E\setminus A)\cap B) = E\cap(A\cup B)$,
  we have, by definition
  \[\pi^*(E\cap (A\cup B) \leq \pi^*(E\cap A) + \pi^*((E\setminus A)\cap B)\]
  So, we have
  \begin{align*}
    \pi^*(E) \geq \pi^*(E\cap(A\cup B)) + \pi^*(E\setminus(A\cup B))
  \end{align*}

  \emph{(3.2) ($\mathcal{M}$ is closed under countable union):} \\
  Suppose we have $\substack{\{A_j\}\\A_j\in\mathcal{M}\\A=\cup_{j\geq 1} A_j}$,
  we want to show that, $\forall E\in\mathcal{M}, \pi^*(E)\geq \pi^*(E\cap A) + \pi^*(E\cap A^c)$.
  Since we have proved that $\mathcal{M}$ is closed under finite union,
  for some fixed $n$, we have
  \begin{align*}
    \pi^*(E) &= \pi^*(E\cap \bigcup_{j=1}^n A_j) + \pi^*(E\setminus (\bigcup_{j=1}^n A_j)) \\
    &\geq \pi^*(E\cap \bigcup_{j=1}^n A_j) + \pi^*(E\setminus A), \hspace{0.5cm} \mbox{by monotone} 
  \end{align*}
  Now, lets define
  \begin{align*}
    F_1 &= A_1 \\
    F_2 &= A_2\setminus A_1 \\
        &\vdots\\
    F_n &= A_n\setminus(A_1\cup A_2 \cup \cdots \cup A_{n-1})
  \end{align*}
  Then we have $\cup_{j=1}^n A_j = \cup_{j=1}^n F_j$ and $F_i\cap F_j = \emptyset$.
  So, now we have
  \begin{align*}
    \pi^*(E) \geq \pi^*(E\cap \sum_{j=1}^n F_j) + \pi^*(E\setminus A)
  \end{align*}
  \begin{claim}[$\pi^*(E\cap \sum_{j=1}^n F_j) = \sum_{j=1}^n \pi^*(E\cap F_j)$\label{claim:additive}]
    \begin{proof}
      We prove this claim by induction.
      It is clear that when $n = 1$, we have noting to prove.
      Then, assume that \[\pi^*(E\cap \sum_{j=1}^n F_j) = \sum_{j=1}^n \pi^*(E\cap F_j)\] for some $n$.
      we want to show \[\pi^*(E\cap \sum_{j=1}^{n+1} F_j) = \sum_{j=1}^{n+1} \pi^*(E\cap F_j)\].
      Note that
      \begin{align*}
        \pi^*(E\cap \sum_{j=1}^{n+1} F_j) &= \pi^*(E\cap \sum_{j=1}^{n+1} F_j \cap F_{n+1}) + \pi^*(E\cap \sum_{j=1}^{n+1} F_j \cap F_{n+1}^c) \\
        &= \pi^*(E\cap F_{n+1}) + \pi^*(E\cap\sum_{j=1}^n F_j)
      \end{align*}
      The last step holds because $F_j$ are disjoint.
    \end{proof}
  \end{claim}
  Using this claim, we have
  \begin{align*}
    \pi^*(E) &\geq \sum_{j=1}^n \pi^*(E\cap F_j) + \pi^*(E\setminus A) \\
    \pi^*(E) &\geq \sum_{j\geq 1} \pi^*(E\cap F_j) + \pi^*(E\setminus A) \\
    &\geq \pi^*(E\cap A) + \pi^*(E\setminus A) \qedhere
  \end{align*}
\end{proof}

\begin{remark}
Since $\mathcal{M}$ is a $\sigma$-algebra, and it contains algebra $\mathcal{A}$,
we have $\mathcal{F}(\mathcal{A}) \subseteq \mathcal{M}$.
\end{remark}

\begin{lemma}
  \[\pi^*|_{\mathcal{M}} : \mathcal{M} \to \mathbb{R}_+\cup\{+\infty\}, \hspace{0.5cm} \mbox{is $\sigma$-additive}\]
  \[\pi^*(A) = \nu(A), \hspace{0.5cm} \forall A\in\mathcal{A}\]
\end{lemma}
\begin{proof}
  \emph{(1) ($\pi^*(A) = \nu(A), \forall A\in\mathcal{A}$):} \\
  \emph{(1.1) ($\pi^*(A) \leq \nu(A)$):} \\
  Let $E_1 = A, E_2 = \emptyset, E_3 = \emptyset, \cdots$, then we have
  \[\pi^*(A) \leq \sum_{j\geq 1} E_j = \nu(A)\]

  \emph{(1.2) ($\nu(A) \leq \sum_{j\geq 1} \nu(E_j)$ for all $\substack{E_j\in\mathcal{A}\\A\subseteq\cup_{j\geq 1} E_j}$):} \\
  Now, lets define
  \begin{align*}
    F_1 &= E_1 \\
    F_2 &= E_2\setminus E_1 \\
        &\vdots\\
    F_n &= E_n\setminus(E_1\cup E_2 \cup \cdots \cup E_{n-1})
  \end{align*}
  Then we have $\cup_{j=1}^n E_j = \cup_{j=1}^n F_j$ and $F_i\cap F_j = \emptyset$.
  So
  \begin{align*}
    A &\subseteq \cup_{j\geq 1} F_j \\
    A &= \sum_{j\geq 1} F_j\cap A \\
    \nu(A) &= \sum_{j\geq 1} \nu(F_j\cap A), \hspace{0.5cm} \mbox{$\nu$ is $\sigma$-additive on $\mathcal{A}$} \\
           &\leq \sum_{j\geq 1} \nu(E_j), \hspace{0.5cm} \mbox{since $F_j\cap A\subseteq E_j$}
  \end{align*}
  Since we consider this for all $\{E_j\}$, we have:
  \begin{align*}
    \nu(A) &\leq \inf_{\{E_j\}} \sum_{j\geq 1} \nu(E_j), \hspace{0.5cm} \{\{E_j\} | \substack{E_j\in\mathcal{A}, A\subseteq \cup_{j\geq 1} E_j}\} \\
             &= \pi^*(A) 
  \end{align*}

  \emph{(2) ($\pi^*|_{\mathcal{M}}$ is $\sigma$-additive):} \\
  Recall that we want to prove
  \[\substack{A_j\in\mathcal{M}\\A_j\cap A_k = \emptyset} \Rightarrow \pi^*(\sum_{j\geq 1} A_j) = \sum_{j\geq 1} \pi^*(A_j)\]
  First, by sub-additive of $\pi^*$, we know that
  \[\pi^*(\sum_{j\geq 1} A_j) \leq \sum_{j\geq 1} \pi^*(A_j)\]
  On the other hand, we have
  \begin{align*}
    \pi^*(\sum_{j\geq 1} A_j) &\geq \pi^*(\sum_{j=1}^n A_j) \hspace{0.5cm} \mbox{monotone} \\
    &= \sum_{j=1}^n \pi^*(A_j) \hspace{0.5cm} \mbox{additive proved by Claim \ref{claim:additive}} \\
    \pi^*(\sum_{j\geq 1} A_j) &\geq \sum_{j\geq 1} \pi^*(A_j)
  \end{align*}
  So, we have 
  \[\pi^*(\sum_{j\geq 1} A_j) = \sum_{j\geq 1} \pi^*(A_j) \qedhere\]
\end{proof}

Now we want to show that $\pi^*$ is unique, according to $\nu$.
At first, we need to introduce some tools.
\begin{define}[monotone class]
  \index{monotone class}
  For $\mathcal{G}\subseteq \mathcal{P}(\Omega)$,
  $\mathcal{G}$ is a monotone class if
  \begin{enumerate}
  \item $\substack{\{A_j\}\\A_j\in\mathcal{G}\\A_j\subseteq A_{j+1}} \Rightarrow A = \cup_{j\geq 1} A_j \in \mathcal{G}$
  \item $\substack{\{B_j\}\\B_j\in\mathcal{G}\\B_j\supseteq B_{j+1}} \Rightarrow B = \cap_{j\geq 1} B_j \in \mathcal{G}$
  \end{enumerate}
\end{define}
\begin{remark}
  Since it is easy to verify that the intersection of two monotone class is also a monotone class,
  we could define the monotone class generated by class $\mathcal{C}$ by $\mathcal{M}(\mathcal{C})$.
\end{remark}

Now we give a lemma that will be proved in the latter sections.
\begin{lemma}
  If $\mathcal{A} \subseteq \mathcal{P}(\Omega)$ is an algebra,
  then $\mathcal{M}(\mathcal{A}) = \mathcal{F}(\mathcal{A})$.
\end{lemma}

\begin{define}[$\sigma$-finite]
  \index{$\sigma$-finite}
  If we say $\Omega$ is $\sigma$-finite on $\mu$, then we means: \\
  $\substack{\{E_j\}\\E_j\subseteq\Omega\\E_j\uparrow \Omega} \Rightarrow \mu(E_j) < +\infty, \forall j$.
\end{define}

\begin{remark}
  Under this setting, $\mu(\Omega)$ could be $+\infty$, but any sequence converges to $\Omega$ should be finite.
\end{remark}

\begin{lemma}[uniqueness]
  We require that
  \[\substack{\mu_1, \mu_2 : \mathcal{F}(\mathcal{A}) \to \mathbb{R}_+\cup\{+\infty\}, \hspace{0.5cm} \mbox{$\sigma$-additive} \\
      \mu_1|_{\mathcal{A}} = \mu_2|_{\mathcal{A}} \\
      \
    }\]
  and $\mu_1$ is $\sigma$-finite by only consider the sequences $\substack{\{E_j\}\\E_j\in\mathcal{A}\\E_j\uparrow\Omega}$ (This also means that $\mu_2$ is $\sigma$-finite because $\mu_1$ and $\mu_2$ are coincident in $\mathcal{A}$).
  Then $\mu_1 = \mu_2$.
\end{lemma}
\begin{proof}
  Take a sequence which satisfies $\substack{\{E_j\}\\E_j\in\mathcal{A}\\ \Omega = \cup_{j\geq 1} E_j \\ \mu_1(E_j) < +\infty}$,
  We could define a sequence of sets on top of it.
  \marginnote[-1cm]{
    Actually, by $\sigma$-finite, all the sequence $\{E_j\}$ such that $E_j\uparrow \Omega$ have the property
    that $\mu_1(E_j)\leq +\infty$.
    Here, the $\Omega\in\mathcal{F}(\mathcal{A})$, so that we could add the restriction $E_j\in\mathcal{A}$ and
    still have $\Omega = \cup_{j\geq 1} E_j$.
  }
  \[\mathcal{B}_n = \{E\in\mathcal{F}(\mathcal{A}) | \mu_1(E\cap E_n) = \mu_2(E\cap E_n)\}\].

  \emph{(1) ($\mathcal{B}_n \supseteq \mathcal{A}$):} \\
  For $E\in\mathcal{A}$, we have $E\cap E_n\subseteq A$,
  so, by definition \[\mu_1(E\cap E_n) = \mu_2(E\cap E_n)\].

  \emph{(2) ($\mathcal{B}_n$ is a monotone class):} \\
  \emph{(2.1) ($\substack{\{A_j\}\\A_j\in \mathcal{B}_n\\A_j\subseteq A_{j+1} \\ A = \cup_{j\geq 1} A_j} \Rightarrow A \in \mathcal{B}_n $):}\\
  By definition, $\mu_1(A_j\cap E_n) = \mu_2(A_j\cap E_n)$.
  And, since $\mu_1, \mu_2$ are $\sigma$-additive, they are continue from below, which means
  \begin{align*}
    A_j\cap E_n \uparrow A\cap E_n &\Rightarrow \mu_1(A_j\cap E_n) \uparrow \mu_1(A\cap E_n) \\
                                   &\Rightarrow \mu_2(A_j\cap E_n) \uparrow \mu_2(A\cap E_n) \\
  \end{align*}
  And since in each $j$, $\mu_1(A_j\cap E_n) = \mu_2(A_j\cap E_n)$, so their limit are also equal.
  Thus
  \[\mu_1(A\cap E_n) = \mu_2(A\cap E_n)\]
  So, $A\in \mathcal{B}_n$.
  
  \emph{(2.2) ($\substack{\{B_j\}\\B_j\in \mathcal{B}_n\\B_j\supseteq B_{j+1}\\ B = \cap_{j\geq 1} B_j} \Rightarrow B\in \mathcal{B}_n$):}\\
  By definition, $\mu_1(B_j\cap E_n) = \mu_2(B_j\cap E_n)$.
  Since $\mu_1(E_n)$ is finite, so we have
  \begin{align*}
    B_j\cap E_n \downarrow B\cap E_n &\Rightarrow \mu_1(B_j\cap E_n) \downarrow \mu_1(B\cap E_n) \\
                                   &\Rightarrow \mu_2(B_j\cap E_n) \downarrow \mu_2(B\cap E_n) \\
  \end{align*}
  Bnd since in each $j$, $\mu_1(B_j\cap E_n) = \mu_2(B_j\cap E_n)$, so their limit are also equal.
  Thus
  \[\mu_1(B\cap E_n) = \mu_2(B\cap E_n)\]
  So, $B\in \mathcal{B}_n$.
  
  Put them together, we have $\mathcal{B}_n$ is a monotone class.
  So we have $\mathcal{B}_n \supseteq \mathcal{M}(\mathcal{A}) = \mathcal{F}(\mathcal{A})$.
  But by definition, $\mathcal{B}_n$ is also contained in $\mathcal{F}(\mathcal{A})$, so
  $\mathcal{B}_n = \mathcal{F}(\mathcal{A})$.

  \emph{(3) ($\mu_1(A) = \mu_2(A)$ for all $A\in\mathcal{F}(\mathcal{A})$):}\\
  Since $A\in\mathcal{F}(\mathcal{A})$, we have $A\in\mathcal{B}_n$.
  So $\mu_1(A\cap E_n) = \mu_2(A\cap E_n)$.
  Since $\mu_1$ and $\mu_2$ are $\sigma$-additive, they continuous from below,
  So we have $\mu_1(A\cap \Omega) = \mu_2(A\cap \Omega)$, so $\mu_1(A) = \mu_2(A)$.
\end{proof}

%%% Local Variables:
%%% mode: latex
%%% TeX-master: "../../notebook"
%%% End:
