%\documentclass{tufte-handout}
\documentclass{article}

\usepackage{amsmath}
\usepackage{amsthm}
\usepackage{tikz}
\usepackage{mathrsfs}
\usetikzlibrary{arrows}
\usepackage{pgffor}
\usepackage{luacode}
\usepackage{enumitem}
\usepackage{tcolorbox}

\title{Exercise of Chapter 1}
\author{Xiaoyu Chen}
\date{}

\newtcolorbox{mybox}[3][]
{
  colframe = #2!25,
  colback  = #2!10,
  coltitle = #2!20!black,  
  title    = {#3},
  #1,
}

\begin{document}
\maketitle
\section{Exercise 1.15}
In the physics community, perfect matchings are sometimes known as ``dimer covers.'' It is of some interest to know the number of dimer covers of a graph $G$ when $G$ has a regular structure that models, for example, a crystal lattice. Let $\Lambda$ to be the $L\times L$ square lattice, with vertex set $V(\Lambda) = \{(i, j) : 0 \leq i, j < L\}$ and edge set $E(\Lambda) = \{(i, j), (i', j')\} : |i - i'| + |j - j'| = 1$. Exhibit a (nicely structured!) Pfaffian orientation of $\Lambda$.
\subsection{solution}
\begin{luacode*}
  function LLGrid(L)
    for x = 1, L do
      for y = 1, L do
        tex.print(string.format('\\node (p%d%d) at (%d, %d) [point] {};', x, y, x, y))
      end
    end
    for x = 1, L do
      for y = 1, L-1 do
        tex.print(string.format('\\draw [->, semithick] (p%d%d) -- (p%d%d);', x, y+1, x, y))
      end
    end
    for x = 1, L-1 do
      for y = 1, L do
        s = '->'
        if y % 2 == 0 then s = '<-' end
        tex.print(string.format('\\draw [%s, semithick] (p%d%d) -- (p%d%d);', s, x, y, x+1, y))
      end
    end
  end
\end{luacode*}
The Pfaffian orientation of $\Lambda$ could be constructed using the following rules:
\begin{mybox}{blue}{}
\begin{enumerate}[nosep]
    \item Make all the edges on the column point down.
    \item Make all the edges on the odd row point right.
    \item Make all the edges on the even row point left.
\end{enumerate}
\begin{center}
  \begin{tikzpicture}
  [point/.style={shape=circle, minimum size=1pt, inner sep=0pt, minimum size=5pt, fill}]  
  \luadirect{LLGrid(5)}
  \end{tikzpicture}
\end{center}
\end{mybox}
It is easy to see that all the faces in this grid has odd number of edges oriented clockwise.
So, this orientation is a Pfaffian orientation.

\section{Exercise 1.16}
Exhibit a non-planar graph that admits a Pfaffian orientation.
\subsection{solution}

\begin{luacode*}
    function triangle() 
        angle = - math.pi * 2 / 3
        cnt = 2
        for i = 1, 3 do
            angle = angle + math.pi * 2 / 3
            cnt = cnt + 1
            tex.print(string.format('\\node (p%d) at (%.2f, %.2f) [point] {%d};', cnt, math.cos(angle), math.sin(angle), cnt))
            print(string.format('\\node (p%d) at (%.2f, %.2f) [point] {};', cnt, math.cos(angle), math.sin(angle)))
            cnt = cnt + 1
            tex.print(string.format('\\node (p%d) at (%.2f, %.2f) [point] {%d};', cnt, 4 + math.cos(angle + math.pi), math.sin(angle + math.pi), cnt))
            print(string.format('\\node (p%d) at (%.2f, %.2f) [point] {};', cnt, 4 + math.cos(angle), math.sin(angle)))
        end
        for i = 1, 3 do
            s = '<-'
            is = '->'
            tex.print(string.format('\\draw [%s, semithick] (p1) -- (p%d);', s, 2*i+1))
            tex.print(string.format('\\draw [%s, semithick] (p2) -- (p%d);', is, 2*i+2))
        end
    end
\end{luacode*}

\begin{center}
\begin{tikzpicture}
  [point/.style={shape=circle, minimum size=1pt, inner sep=0pt, minimum size=5pt}]  
  \node (p1) at (0, 0) [point] {1};
  \node (p2) at (4, 0) [point] {2};
  \luadirect{triangle()}
  \draw [->, semithick] (p1) to [bend left=45] (p2);
  \draw [->, semithick] (p3) to (p4);
  \draw [->, semithick] (p5) to [bend left=45] (p8);
  \draw [->, semithick] (p7) to [bend right=45] (p6);
  \draw [->, semithick] (p3) to [bend right = 45] (p5);
  \draw [->, semithick] (p5) to [bend right = 45] (p7);
  \draw [->, semithick] (p7) to [bend right = 45] (p3);
  \draw [->, semithick] (p4) to [bend right = 45] (p6);
  \draw [->, semithick] (p6) to [bend right = 45] (p8);
  \draw [->, semithick] (p8) to [bend right = 45] (p4);
\end{tikzpicture}
\end{center}
As you can see, if we shrink $3, 5, 7$ into a single vertex, we will get a $K_5$, so this graph is a non-planar graph.
\textcolor[rgb]{0.5,0,1.0}{I don't know how to proof that this orientation is a Pfaffian orientation. I just can't find any counter-example.}

\section{Exercise 1.17}
Exhibit a (necessarily non-planar) graph that does not admit a Pfaffian orientation.
\subsection{solution}
The main idea of solving this problem is to assume that the Pfaffian orientation always exists and then construct a conflict.

We could define a variable for each edge (with a direction), and then we could construct an expression to represent an odd orientation for each even cycle.
For example, for edge $\overrightarrow{AF}$, we could use $\overrightarrow{AF}$ to represent its direction and use $1- \overrightarrow{AF}$ to represent the direction of $\overrightarrow{FA}$.
 Then, if we have some even cycle $AFBE$, we could represent an odd orientation on cycle $AFBE$ by using the following equation:
     \[
         \overrightarrow{AF} + \overrightarrow{FB} + \overrightarrow{BE} + \overrightarrow{EA} \equiv 1 (\mbox{mod } 2)
     \]
And after carefully considering the case of clique, I have the following result:


\begin{mybox}{blue}{
All the even circle of a clique (even number of vertices) could be constructed by the union of two perfect matches.
}
\begin{proof}
    Suppose that we want to construct an even circle $C$ in a clique $G$ (with even number of vertices) using perfect match $M$ and $M'$. We could distribute the edges on the circle to each perfect match alternately. By doing this, the circle $C$ must appear in $M\cup M'$. (Note that $G\setminus C$ is also a clique, so there is always a match on it.)
\end{proof}
\end{mybox}

\definecolor{uuuuuu}{rgb}{0.26666666666666666,0.26666666666666666,0.26666666666666666}
\begin{center}
\begin{tikzpicture}[line cap=round,line join=round,>=triangle 45,x=1.0cm,y=1.0cm]
%\clip(-2.904763467029151,-4.170426945416111) rectangle (9.702326843195893,6.023927115783117);
\draw [line width=1.2pt] (1.,3.)-- (2.,3.);
\draw [line width=1.2pt] (2.,3.)-- (3.,3.);
\draw [line width=1.2pt] (3.,3.)-- (3.,2.);
\draw [line width=1.2pt] (3.,2.)-- (3.,1.);
\draw [line width=1.2pt] (3.,1.)-- (2.,1.);
\draw [line width=1.2pt] (2.,1.)-- (1.,1.);
\draw [line width=1.2pt] (1.,1.)-- (1.,2.);
\draw [line width=1.2pt] (1.,2.)-- (1.,3.);
\draw [line width=1.2pt] (1.,2.)-- (2.,2.);
\draw [line width=1.2pt] (2.,2.)-- (3.,2.);
\begin{scriptsize}
\draw [fill=uuuuuu] (1.,1.) circle (2.0pt);
\draw[color=uuuuuu] (1.1599363398882336,1.3614773372602293) node {$A$};
\draw [fill=uuuuuu] (2.,1.) circle (2.0pt);
\draw[color=uuuuuu] (2.159809019664703,1.3614773372602293) node {$B$};
\draw [fill=uuuuuu] (3.,1.) circle (2.0pt);
\draw[color=uuuuuu] (3.159681699441172,1.3614773372602293) node {$C$};
\draw [fill=uuuuuu] (1.,2.) circle (2.0pt);
\draw[color=uuuuuu] (1.1599363398882336,2.361350017036699) node {$D$};
\draw [fill=uuuuuu] (2.,2.) circle (2.0pt);
\draw[color=uuuuuu] (2.159809019664703,2.361350017036699) node {$E$};
\draw [fill=uuuuuu] (3.,2.) circle (2.0pt);
\draw[color=uuuuuu] (3.159681699441172,2.361350017036699) node {$F$};
\draw [fill=uuuuuu] (1.,3.) circle (2.0pt);
\draw[color=uuuuuu] (1.1599363398882336,3.3612226968131695) node {$G$};
\draw [fill=uuuuuu] (2.,3.) circle (2.0pt);
\draw[color=uuuuuu] (2.159809019664703,3.3612226968131695) node {$H$};
\draw [fill=uuuuuu] (3.,3.) circle (2.0pt);
\draw[color=uuuuuu] (3.159681699441172,3.3612226968131695) node {$I$};
\draw [fill=uuuuuu] (4.,2.) circle (2.0pt);
\draw[color=uuuuuu] (4.15955437921764,2.361350017036699) node {$J$};
\end{scriptsize}
\end{tikzpicture}
\end{center}

After many experiments, I find a nice structure of $K_{10}$. Using the result above, we know that circle $GHIFCBAD$, $GHIFED$, and $DEFCBA$ could be constructed by two perfect matches. So, if $K_{10}$ has a Pfaffian orientation, then we have the following equations:
\def\ORA{\overrightarrow}

For circle $GHIFCBAD$, we have:
\begin{align*}
    S_1 = \ORA{GH} + \ORA{HI} + \ORA{IF} + \ORA{FC} + \ORA{CB} + \ORA{BA} + \ORA{AD} + \ORA{DG} &\equiv 1 (\mbox{mod } 2) 
\end{align*}

For circle $GHIFED$, we have:
\begin{align*}
    S_2 = \ORA{GH} + \ORA{HI} + \ORA{IF} + \ORA{FE} + \ORA{ED} + \ORA{DG} &\equiv 1 (\mbox{mod } 2) 
\end{align*}

For circle $DEFCBA$, we have:
\begin{align*}
    \ORA{DE} + \ORA{EF} + \ORA{FC} + \ORA{CB} + \ORA{BA} + \ORA{AD} &\equiv 1 (\mbox{mod } 2) \\
    (1 - \ORA{ED}) + (1 - \ORA{FE}) + \ORA{FC} + \ORA{CB} + \ORA{BA} + \ORA{AD} &\equiv 1 (\mbox{mod } 2) \\
    S_3 = - \ORA{DE} - \ORA{FE} + \ORA{FC} + \ORA{CB} + \ORA{BA} + \ORA{AD} &\equiv 1 (\mbox{mod } 2)  
\end{align*}

Looking at the left-hand side of these equations, we have:
\begin{equation}
    S_1 + S_2 + S_3 \equiv 0 (\mbox{mod } 2)
\end{equation}

But when we look at the right-hand side of these equations, we have:
\begin{equation}
    S_1 + S_2 + S_3 \equiv 1 (\mbox{mod } 2)
\end{equation}
So, there is a contradiction between equation (1) and (2). And we could draw a conclusion that $K_{10}$ could not have any Pfaffian orientation.
\begin{mybox}{red}{
    A sad story
}    
    I have wasted lots of my time trying $K_{3,3}$, $K_{4,4}$ and even some planar graphs (they looks really like a non-planar graph). I am not brave enough to try solving complete graphs at first because they seem to be very complex.
\end{mybox}

\section{Exercise 1.18}
The dimer model is one model from statistical physics; another is the Ising model. Computing the “partition function” of an Ising system with underlying graph $G$ in the absence of an external field is essentially equivalent to counting ``closed subgraphs'' of $G$: subgraphs $(V,A\subseteq E)$ such that the degree of every vertex $i \in V$ in $(V, A)$ is even (possibly zero). Show that the problem of counting closed subgraphs in a planar graph is efficiently reducible to counting perfect matchings (or dimer covers) in a derived planar graph. The bottom line is that the Ising model for planar systems with no applied field is computationally feasible.

\subsection{solution}
This question is very interesting, though it is very confusing at first. To solve this problem, we need to construct some kinds of structures to check the parity of the degree of each vertex. Luckily, I have found one:
\begin{mybox}{blue}{
    We could simulate a XOR gate using matching
}
\begin{center}
\begin{tikzpicture}[line cap=round,line join=round,>=triangle 45,x=1.0cm,y=1.0cm]
%\clip(-3.9481088720132926,-3.474863342093349) rectangle (8.65898143821175,6.719490719105878);
\draw [line width=1pt] (3.3, 4.5) -- (3.3, -0.5);
\draw [line width=1.2pt] (2.,2.)-- (1.,1.);
\draw [line width=1.2pt] (1.,1.)-- (3.,1.);
\draw [line width=1.2pt] (3.,1.)-- (2.,2.);
\draw [line width=1.2pt] (2.,2.)-- (2.,3.);
\draw [line width=1.2pt] (2.,4.) node [above = 2pt] {$out$}-- (2.,3.);
\draw [line width=1.2pt] (1.,1.)-- (1.,0.) node [below = 2pt] {$in_1$};
\draw [line width=1.2pt] (3.,1.)-- (3.,0.) node [below = 2pt] {$in_2$};
\begin{scriptsize}
\draw [fill=uuuuuu] (1.,1.) circle (2.0pt);
\draw [fill=uuuuuu] (3.,1.) circle (2.0pt);
\draw [fill=uuuuuu] (2.,2.) circle (2.0pt);
\draw [fill=uuuuuu] (2.,3.) circle (2.0pt);
\end{scriptsize}
\end{tikzpicture}
\begin{tikzpicture}[line cap=round,line join=round,>=triangle 45,x=1.0cm,y=1.0cm]
%\clip(-3.948108872013293,-3.47486334209335) rectangle (8.65898143821175,6.7194907191058775);
\draw [line width=1.2pt] (2.,2.)-- (1.,1.);
\draw [line width=1.2pt] (1.,1.)-- (3.,1.);
\draw [line width=1.2pt] (3.,1.)-- (2.,2.);
\draw [line width=1.2pt] (2.,2.)-- (2.,3.);
\draw [line width=1.2pt] (2.,4.) node [above = 2pt] {$0$} -- (2.,3.);
\draw [line width=1.2pt] (1.,1.)-- (1.,0.) node [below = 2pt] {$0$};
\draw [line width=1.2pt] (3.,1.)-- (3.,0.) node [below = 2pt] {$0$};
\draw [rotate around={0.:(2.,1.)},line width=1.2pt] (2.,1.) ellipse (1.1320505420979492cm and 0.530601950490445cm);
\draw [rotate around={90.:(2.,2.5)},line width=1.2pt] (2.,2.5) ellipse (0.7779201445888927cm and 0.5959528096730526cm);
\begin{scriptsize}
\draw [fill=uuuuuu] (1.,1.) circle (2.0pt);
\draw [fill=uuuuuu] (3.,1.) circle (2.0pt);
\draw [fill=uuuuuu] (2.,2.) circle (2.0pt);
\draw [fill=uuuuuu] (2.,3.) circle (2.0pt);
\end{scriptsize}
\end{tikzpicture}
\begin{tikzpicture}[line cap=round,line join=round,>=triangle 45,x=1.0cm,y=1.0cm]
%\clip(-3.948108872013293,-3.47486334209335) rectangle (8.65898143821175,6.7194907191058775);
\draw [line width=1.2pt] (2.,2.)-- (1.,1.);
\draw [line width=1.2pt] (1.,1.)-- (3.,1.);
\draw [line width=1.2pt] (3.,1.)-- (2.,2.);
\draw [line width=1.2pt] (2.,2.)-- (2.,3.);
\draw [line width=1.2pt] (2.,4.) node [above = 2pt] {$1$} -- (2.,3.);
\draw [line width=1.2pt] (1.,1.)-- (1.,0.) node [below = 2pt] {$0$};
\draw [line width=1.2pt] (3.,1.)-- (3.,0.) node [below = 2pt] {$1$};
\draw [rotate around={90.:(2.,3.5)},line width=1.2pt] (2.,3.5) ellipse (0.6274504638410235cm and 0.37907002595076883cm);
\draw [rotate around={45.:(1.5,1.5)},line width=1.2pt] (1.5,1.5) ellipse (0.846341420728762cm and 0.46507397308511983cm);
\draw [rotate around={90.:(3.,0.5014773372602293)},line width=1.2pt] (3.,0.5014773372602293) ellipse (0.625958968347376cm and 0.3785495803581805cm);
\begin{scriptsize}
\draw [fill=uuuuuu] (1.,1.) circle (2.0pt);
\draw [fill=uuuuuu] (3.,1.) circle (2.0pt);
\draw [fill=uuuuuu] (2.,2.) circle (2.0pt);
\draw [fill=uuuuuu] (2.,3.) circle (2.0pt);
\end{scriptsize}
\end{tikzpicture}
\begin{tikzpicture}[line cap=round,line join=round,>=triangle 45,x=1.0cm,y=1.0cm]
%\clip(-3.948108872013293,-3.47486334209335) rectangle (8.65898143821175,6.7194907191058775);
\draw [line width=1.2pt] (2.,2.)-- (1.,1.);
\draw [line width=1.2pt] (1.,1.)-- (3.,1.);
\draw [line width=1.2pt] (3.,1.)-- (2.,2.);
\draw [line width=1.2pt] (2.,2.)-- (2.,3.);
\draw [line width=1.2pt] (2.,4.) node [above = 2pt] {$0$} -- (2.,3.);
\draw [line width=1.2pt] (1.,1.)-- (1.,0.) node [below = 2pt] {$1$};
\draw [line width=1.2pt] (3.,1.)-- (3.,0.) node [below = 2pt] {$1$};
\draw [rotate around={90.:(1.,0.5123455185621477)},line width=1.2pt] (1.,0.5123455185621477) ellipse (0.6168687392078445cm and 0.37778320257186493cm);
\draw [rotate around={90.:(3.,0.5014773372602294)},line width=1.2pt] (3.,0.5014773372602294) ellipse (0.6390150421890468cm and 0.39976915698777105cm);
\draw [rotate around={90.:(2.,2.5)},line width=1.2pt] (2.,2.5) ellipse (0.6606769975086546cm and 0.43184962086014655cm);
\begin{scriptsize}
\draw [fill=uuuuuu] (1.,1.) circle (2.0pt);
\draw [fill=uuuuuu] (3.,1.) circle (2.0pt);
\draw [fill=uuuuuu] (2.,2.) circle (2.0pt);
\draw [fill=uuuuuu] (2.,3.) circle (2.0pt);
\draw[color=black] (1.159936339888233,0.40507738269143057);
\draw[color=black] (3.159681699441171,0.3833410200875943);
\draw[color=black] (2.11633629445703,2.361350017036698);
\end{scriptsize}
\end{tikzpicture}    
\end{center}
\end{mybox}
For convenience, we could write this XOR gate as:
\[ out = in_1 \oplus in_2 \]
So, if there is a vertex $v$ whose edges are $\{e_1, e_2, \cdots, e_m\}$, once some of its edges are chosen to be in the closed subgraph, we could check the parity of its degree by:
    \[ out = e_1 \oplus e_2 \oplus \cdots \oplus e_m \]
The magic here is that not only we could check the parity of degree by our gate, but also we could force it to be a certain value.
If we hang the output port of the outer most XOR gate, we could force the output of all these XOR gates to be zero (see figure below).

\begin{center}
\begin{tikzpicture}[line cap=round,line join=round,>=triangle 45,x=1.0cm,y=1.0cm]
%\clip(-2.72355627904485,-2.393143872256391) rectangle (7.810934677369749,6.125263470085766);
\draw [line width=1.2pt] (2.,2.)-- (1.,1.);
\draw [line width=1.2pt] (1.,1.)-- (3.,1.);
\draw [line width=1.2pt] (3.,1.)-- (2.,2.);
\draw [line width=1.2pt] (2.,2.)-- (2.,3.);
\draw [line width=1.2pt] (1.,1.)-- (1.,0.);
\draw [line width=1.2pt] (3.,1.)-- (3.,0.);
\begin{scriptsize}
\draw [fill=uuuuuu] (1.,1.) circle (2.0pt);
\draw [fill=uuuuuu] (3.,1.) circle (2.0pt);
\draw [fill=uuuuuu] (2.,2.) circle (2.0pt) node [right = 2pt] {$u$};
\draw [fill=uuuuuu] (2.,3.) circle (2.0pt) node [above = 2pt] {$v$};
\end{scriptsize}
\end{tikzpicture}
\end{center}
Note that $deg(v) = 1$, its clear that $\{v, u\}$ should contained by all the perfect match


So, by converting all the vertices in the original graph $G$ into the form of XOR gate, we have a new graph $G'$. It is clear that:
    \[ \mbox{\# of perfect matches on }G' = \mbox{\# of closed subgraphs on }G \]

\end{document}
