\section{Brunn-Minkowski Theorem}
\marginnote{
  The material of this topic is mainly found in a course named \href{https://ocw.mit.edu/courses/mathematics/18-409-topics-in-theoretical-computer-science-an-algorithmists-toolkit-fall-2009/lecture-notes/}{An Algorithmist's Toolkit} at \textbf{MIT OCW}. They are really good at explaing this.
}
\marginnote[0.5cm]{
  Note that this definition has some good properties only when $A$ and $B$ are convex.
}
\begin{define}[The Notation of Minkowski Sum]
  \index{minkowski sum}
  Let $A$ and $B$ be sets of points and $\lambda$ be a real number. A point $p$ is represented by the vector pointing from $0$ to $p$. Then the Minkowshi Sum notation could be defined as below:
  \begin{align*}
    A + B &= \{a + b| a\in A, b\in B\} \\
    \lambda A &= \{\lambda a | a\in A\}
  \end{align*}
\end{define}
Fist, we provide some trivial facts about $\mathrm{Vol}_n$ and Minkowshi Sum.
\begin{fact}
  \[
    \mathrm{Vol}_n (A + B) \geq \max\{\mathrm{Vol}_n(A), \mathrm{Vol}_n(B)\}
  \]
\end{fact}
\begin{proof}
  First, if $A = \emptyset$ or $B = \emptyset$, then this fact is true. Else there exists $a\in A$, and clearly we have $\{a\} + B \subseteq A + B$. Hence,
  \[\mathrm{Vol}_n(B)\leq \mathrm{Vol}_n(A+B)\]
  . Similarly,
  \[\mathrm{Vol}_n(A) \leq \mathrm{Vol}_n (A + B) \qedhere\].
\end{proof}
\marginnote[0.5cm]{
  The bound given by this fact is loose, since we have $\mathrm{Vol}_n(2A) = 2^n\mathrm{Vol}_n(A)$
}
\begin{fact}
  \[
    \mathrm{Vol}_n(A + B) \geq \mathrm{Vol}_n(A) + \mathrm{Vol}_n(B)
  \]
\end{fact}
\begin{proof}
  Assume $A$ and $B$ are all positive. Then we could choose two points $a\in A$ and $b\in B$ with both has the maximum value in direction $e_1$ (e.g. $x$ direction). Then its clear that $\{a\} + B$ is disjoint with $\{b\} + A$. So we have:
  \begin{align*}
    \{a\} + B + \{b\} + A &\subseteq A + B \\
    \mathrm{Vol}_n(\{a\} + B) + \mathrm{Vol}_n(\{b\} + A) &\leq \mathrm{Vol}_n(A + B) \qedhere
  \end{align*}
\end{proof}
\begin{fact}
  Move $A$ and $B$ will not change the volume of $A$, $B$, or $A + B$.
\end{fact}
\begin{proof}
  Suppose we want to move $A$ by a vector $a$ and move $B$ by a vector $b$. Then we have:
  \begin{align*}
    A + a &= \{x + a | x \in A\} \\
    B + b &= \{x + b | x \in B\} \\
    (A + a) + (B + b) &= \{x + (a + b)| x \in A + B\}
  \end{align*}
  Clearly, these three are bijection betweent $A$ and $A+a$, $B$ and $B+b$, $A + B$ and $A + B + a + b$.
  So, the move of these point sets will not change their volume.
\end{proof}
Now we are going to get a tight bound on this:
\begin{lemma}[Burnn-Minkowshi Theorem on Box]
  Let $A$ and $B$ be box in $\mathbb{R}^n$. Then:
  \[
    \mathrm{Vol}_n(A + B)^{1/n} \geq \mathrm{Vol}_n(A)^{1/n} + \mathrm{Vol}_n(B)^{1/n}
  \]
\end{lemma}
\begin{proof}
  Let $A$ have sides of length $a_1, a_2, \cdots, a_n$, $B$ have sides of length $b_1, b_2, \cdots, b_n$. Then the Minkowshi Sum $A + B$ has sides of length $a_1 + b_1, a_2 + b_2, \cdots, a_n + b_n$. Then:
  \begin{align*}
    \frac{\mathrm{Vol}_n(A)^{1/n} + \mathrm{Vol}_n(B)^{1/n}}{\mathrm{Vol}_n(A + B)^{1/n}} &= \frac{(\prod_{i=1}^n a_i)^{1/n} + (\prod_{i=1}^n b_i)^{1/n}}{(\prod_{i=1}^n (a_i + b_i))^{1/n}} \\
    &= \prod_{i=1}^n (\frac{a_i}{a_i + b_i})^{1/n} + \prod_{i=1}^n (\frac{b_i}{a_i + b_i})^{1/n} \\
    &\leq \frac{1}{n}\sum_{i=1}^n \frac{a_i}{a_i + b_i} + \frac{1}{n}\sum_{i=1}^n \frac{b_i}{a_i + b_i} \\
    &\leq 1 \qedhere
  \end{align*}
\end{proof}
Now, we want to generize the result to any convexi point sets $A$, and $B$.
\begin{theorem}[Brunn-Minkowski Theorem]
  Let $A$ and $B$ be convex point sets (or measurable point sets). Then:
  \[
    \mathrm{Vol}_n(A + B)^{1/n} \geq \mathrm{Vol}_n(A)^{1/n} + \mathrm{Vol}_n(B)^{1/n}
  \]
\end{theorem}
\begin{proof}
  We prove this by induction. We pick the base case to be two boxes $A$ and $B$, which is proved in the lemma above.

  In the inductive case, $A$ and $B$ are made up by finite number of disjoint boxes. As the number of boxes converge to $\infty$, we could simulate any convex point set.

  Define the following subsets of $\mathbb{R}^n$:
  \begin{align*}
    A^+ = A\cap \{x\in\mathbb{R}^n|x_n\geq 0\} &,\hspace{0.5cm} A^- = A\cap \{x\in\mathbb{R}^n|x_n\leq 0\} \\
    B^+ = B\cap \{x\in\mathbb{R}^n|x_n\geq 0\} &,\hspace{0.5cm} B^- = B\cap \{x\in\mathbb{R}^n|x_n\leq 0\} \\
  \end{align*}
  Move $A$ and $B$ such that the following conditions holds:
  \begin{enumerate}
  \item A has some pair of boxes separated by the hyperplane $\{x \in \mathbb{R}^n|x_1 = 0\}$. i.e. there exists a box that lies completely in the halfspace $\{x \in \mathbb{R}^n|x_1 \geq 0\}$ and there is some other box that lies in its complement half-space.
  \item It holds that
    \[\frac{\mathrm{Vol}_n(A^+)}{\mathrm{Vol}_n(A)} = \frac{\mathrm{Vol}_n(B^+)}{\mathrm{Vol}_n(B)}\]
  \end{enumerate}
  Actually, we could split $A$ into two equal volume parts and split each part recurrently. We could do the same operation on $B$. This operation could stop when the volume of each part is small enough. In this way, the above two requirement could always be achieved.

  Then because we achieved condition 1, we could find that $A^+$ and $A^-$ are strictly subset of $A$. So $A^+\cup B^+$ and $A^- \cup B^-$ have fewer boxes than $A\cup B$ and the inductive hypothesis is ture on them. Moreover, $A^+\cup B^+$ and $A^- and B^-$ are disjoint because they have different sign of the $x_1$ coordinate. Hence we have:
  \begin{align*}
    \mathrm{Vol}_n&(A + B) \geq \mathrm{Vol}_n(A^+ + B^+) + \mathrm{Vol}_n(A^- + B^-) \\
    &\geq (\mathrm{Vol}_n(A^+)^{1/n} + \mathrm{Vol}_n(B^+)^{1/n})^n + (\mathrm{Vol}_n(A^-)^{1/n} + \mathrm{Vol}_n(B^-)^{1/n})^n, \hspace{0.5cm} \mbox{by inductive hypothesis}\\
    &= \mathrm{Vol}_n(A^+)(1 + \frac{\mathrm{Vol}_n(B^+)^{1/n}}{\mathrm{Vol}_n(A^+)^{1/n}})^n + \mathrm{Vol}_n(A^-)(1 + \frac{\mathrm{Vol}_n(B^-)^{1/n}}{\mathrm{Vol}_n(A^-)^{1/n}})^n \\
    &= \mathrm{Vol}_n(A^+)(1 + \frac{\mathrm{Vol}_n(B)^{1/n}}{\mathrm{Vol}_n(A)^{1/n}})^n + \mathrm{Vol}_n(A^-)(1 + \frac{\mathrm{Vol}_n(B)^{1/n}}{\mathrm{Vol}_n(A)^{1/n}})^n, \hspace{0.5cm} \mbox{by condition 2} \\
    &= (\mathrm{Vol}_n(A^+) + \mathrm{Vol}_n(A^-))(1 + \frac{\mathrm{Vol}_n(B)^{1/n}}{\mathrm{Vol}_n(A)^{1/n}})^n \\
    &= \mathrm{Vol}_n(A)(1 + \frac{\mathrm{Vol}_n(B)^{1/n}}{\mathrm{Vol}_n(A)^{1/n}})^n \\
    &= (\mathrm{Vol}_n(A)^{1/n} + \mathrm{Vol}_n(B)^{1/n})^n \qedhere
  \end{align*}
\end{proof}

Here we have an equivalent for Brunn-Minkowshi Theorem:
\begin{corollary}
  $\mathrm{Vol}_n (\lambda A + (1-\lambda) B)^{1/n} \geq \lambda\mathrm{Vol}_n(A)^{1/n} + (1-\lambda)\mathrm{Vol}_n (B)^{1/n}$
\end{corollary}
\begin{proof}
  \begin{align*}
    \mathrm{Vol}_n(\lambda A + (1-\lambda)B)^{1/n} &\geq \mathrm{Vol}_n(\lambda A)^{1/n} + \mathrm{Vol}_n((1-\lambda)B)^{1/n} \\
    &= (\lambda^n\mathrm{Vol}_n(A))^{1/n} + ((1-\lambda)^n\mathrm{Vol}_n((1-\lambda)B))^{1/n} \\
    &= \lambda\mathrm{Vol}_n(A)^{1/n} + (1-\lambda)\mathrm{Vol}_n((1-\lambda)B)^{1/n} \\
  \end{align*}
\end{proof}

%%% Local Variables:
%%% mode: latex
%%% TeX-master: "../../notebook"
%%% End:
